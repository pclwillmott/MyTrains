% Activate the following line by filling in the right side. If for example the name of the root file is Main.tex, write
% "...root = Main.tex" if the chapter file is in the same directory, and "...root = ../Main.tex" if the chapter is in a subdirectory.
 
%!TEX root =  

\chapter[Opcodes]{Loconet Commands}

\section{Introduction}

\rule{15.1cm}{0.4pt}
\subsubsection{OPC\_IDLE}
\underline{Operation:} Force idle state and broadcast emergency stop.

\underline{Group:} \hspace{0.5cm} 2-Byte Message

\underline{Direction:} \hspace{0.05cm} $\rightarrow$ Command Station

\underline{Encoding:} 

\begin{tabular}{l l l} 
\begin{tabular}{|p{0.3cm}|p{0.3cm}|p{0.3cm}|p{0.3cm}|p{0.3cm}|p{0.3cm}|p{0.3cm}|p{0.3cm}|}
\hline
1 & 1 & 1 & 0 & 1 & 1 & 0 & 1\\
\hline
\end{tabular}

& 0xED & $<$opCode$>$\\
& \\
\begin{tabular}{|p{0.3cm}|p{0.3cm}|p{0.3cm}|p{0.3cm}|p{0.3cm}|p{0.3cm}|p{0.3cm}|p{0.3cm}|}
\hline
0 & 1 & s & s & 1 & 0 & 1 & 0\\
\hline
\end{tabular}
& 0x00 & $<$checkSum$>$
\end{tabular}

\underline{Description:}

This command forces Loconet into the idle state and broadcasts an emergency stop.

\underline{Response:} 

None

