% Activate the following line by filling in the right side. If for example the name of the root file is Main.tex, write
% "...root = Main.tex" if the chapter file is in the same directory, and "...root = ../Main.tex" if the chapter is in a subdirectory.
 
%!TEX root =  

\chapter[Opcodes]{Loconet Opcodes}

\section{Introduction}

\rule{15.1cm}{0.4pt}
\subsubsection{ADC HL, ss}
\underline{Operation:} $ HL \leftarrow HL + ss + CY $

\underline{Group:} \hspace{0.5cm} 16-Bit Arithmetic

\underline{Format:} 

\begin{tabular}{l l}
\underline{Opcode} & \underline{Operands} \\
ADC & HL, ss\\
\end{tabular}

\begin{tabular}{l l l}
ADC HL, ss &
\begin{tabular}{|p{0.3cm}|p{0.3cm}|p{0.3cm}|p{0.3cm}|p{0.3cm}|p{0.3cm}|p{0.3cm}|p{0.3cm}|}
\hline
1 & 1 & 1 & 0 & 1 & 1 & 0 & 1\\
\hline
\end{tabular}
& 0EDH\\
& & \\
&
\begin{tabular}{|p{0.3cm}|p{0.3cm}|p{0.3cm}|p{0.3cm}|p{0.3cm}|p{0.3cm}|p{0.3cm}|p{0.3cm}|}
\hline
0 & 1 & s & s & 1 & 0 & 1 & 0\\
\hline
\end{tabular}
& \\
\end{tabular}

\underline{Description:}

The contents of register pair ss (any of register pairs BC, DE, HL or SP) are added with CY, the Carry Flag (C flag in the F register), to the contents of register pair HL, and the result is stored in HL. Operand ss is specified as follows in the assembled object code: 

\begin{tabular}{c c}
\underline{Register} &\\
\underline{Pair} & \underline{ss}\\
BC & 00 \\
DE & 01 \\
HL & 10 \\
SP & 11 \\
\end{tabular}

M CYCLES: 4 T STATES: 15(4,4,4,3) 

\underline{Condition Bits Affected:} 

\begin{tabular}{r l}
S: & Set if result is negative; reset otherwise \\
Z: & Set if result is zero; reset otherwise \\
H: & Set if carry from Bit 11; reset otherwise \\ 
P/V: & Set if overflow; reset otherwise \\
N: & Reset \\
C: & Set if carry from Bit 15; reset otherwise \\
\end{tabular}

\underline{Example:} 

If the register pair BC contains 2222H, register pair HL contains 5437H and the Carry Flag is set, after the execution of

\begin{verbatim}
ADC HL, BC 
\end{verbatim}

the contents of HL will be 765AH. 

