\newpage
\section{Ack}\index{Ack}

\underline{Description:}

This message provides a response code from a \gls{Command}. This is the generic form of this message type.

\underline{Group:}

4-Byte Message

\underline{Opcode:}

OPC\_LONG\_ACK

\underline{Type:}

\gls{Response}

\underline{Encoding:} 

Byte 0:

\begin{tabular}{p{0.4\linewidth} p{0.15\linewidth} p{0.38\linewidth}} 

\begin{tabular}{|p{0.3cm}|p{0.3cm}|p{0.3cm}|p{0.3cm}|p{0.3cm}|p{0.3cm}|p{0.3cm}|p{0.3cm}|}
\hline
1 & 0 & 1 & 1 & 0 & 1 & 0 & 0\\
\hline
\end{tabular}
& 0xB4 & Opcode.\\
\end{tabular}

Byte 1:

\begin{tabular}{p{0.4\linewidth} p{0.15\linewidth} p{0.38\linewidth}} 

\begin{tabular}{|p{0.3cm}|p{0.3cm}|p{0.3cm}|p{0.3cm}|p{0.3cm}|p{0.3cm}|p{0.3cm}|p{0.3cm}|}
\hline
0 & n & n & n & n & n & n & n\\
\hline
\end{tabular}
& $<$LOPC$>$ & Opcode of the \gls{Command} that this message is a response to with the most significant bit set to 0.\\
\end{tabular}

Byte 2:

\begin{tabular}{p{0.4\linewidth} p{0.15\linewidth} p{0.38\linewidth}} 

\begin{tabular}{|p{0.3cm}|p{0.3cm}|p{0.3cm}|p{0.3cm}|p{0.3cm}|p{0.3cm}|p{0.3cm}|p{0.3cm}|}
\hline
0 & n & n & n & n & n & n & n\\
\hline
\end{tabular}
& $<$ACK1$>$ & Response code. This is usually 0 to indicate the \gls{Command} failed, and 127 (0x7F) if it was successful. Other values are possible to indicate other conditions or states.\\
\end{tabular}

Byte 3:

\begin{tabular}{p{0.4\linewidth} p{0.15\linewidth} p{0.38\linewidth}} 

\begin{tabular}{|p{0.3cm}|p{0.3cm}|p{0.3cm}|p{0.3cm}|p{0.3cm}|p{0.3cm}|p{0.3cm}|p{0.3cm}|}
\hline
0 & n & n & n & n & n & n & n\\
\hline
\end{tabular}
& $<$CHK$>$ & Checksum.
\end{tabular}

\underline{Response:} 

None.

\underline{Signature:}

Byte 0:

\begin{tabular}{p{0.4\linewidth} p{0.38\linewidth}} 

\begin{tabular}{|p{0.3cm}|p{0.3cm}|p{0.3cm}|p{0.3cm}|p{0.3cm}|p{0.3cm}|p{0.3cm}|p{0.3cm}|}
\hline
1 & 0 & 1 & 1 & 0 & 1 & 0 & 0\\
\hline
\end{tabular}
& 0xB4\\
\end{tabular}

\underline{Notes:} 

None.

\rule{15.1cm}{0.4pt}
