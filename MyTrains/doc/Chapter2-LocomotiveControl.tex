% Activate the following line by filling in the right side. If for example the name of the root file is Main.tex, write
% "...root = Main.tex" if the chapter file is in the same directory, and "...root = ../Main.tex" if the chapter is in a subdirectory.
 
%!TEX root =  

\chapter[Locomotive Control]{Locomotive Control}

\section{Introduction}

Initially all locomotive slots are empty and are said to be \gls{Free}. A Free slot does not have a locomotive address loaded and no DCC commands are generated by the command station for it. To control a locomotive a \gls{throttle} must request a slot from the command station and in the case of an expanded slot take ownership of it.

\subsection{Slot State}\index{Slot State}

A locomotive slot's \gls{slot state} is determined by bits d5 and d4 of the \gls{Slot Status 1} byte of the applicable \textbf{LocoSlotDataP1} or \textbf{LocoSlotDataP2} response and whether the locomotive's address has been loaded. The slot state determines whether DCC commands are generated for it and if throttles can take control of it.

\begin{tabular}{l l l l l l}
\underline{Slot State} & \underline{d5} & \underline{d4} & \underline{Address Loaded} & \underline{Decoder Refreshed} & \underline{Any Throttle}\\
Free & 0 & 0 & No & No & Yes\\
New & 0 & 0 & Yes & No & Yes\\
Common & 0 & 1 & Yes & Yes & Yes\\
Idle & 1 & 0 & Yes & No & Yes\\
In-Use & 1 & 1 & Yes & Yes & No\\
\end{tabular}

\subsection{Throttle ID}\index{Throttle ID}

The \gls{Throttle ID} for a \gls{physical throttle} is derived from the throttle's serial number. Digitrax serial numbers are 16-bit numbers. The Throttle ID is split into two parts consisting of the least significant bits of the low and high bytes of the serial number respectively. For example a physical throttle with the serial number of 0xFFFE would have a Throttle ID of 0x7E 0x7F with 0x7E being the low byte. The low byte of the Throttle ID is required by some of the protocol 2 commands to ensure that only the throttle that has ownership of the locomotive slot is the one that updates the slot. A \gls{software throttle} should choose a Throttle ID that does not clash with that of a physical throttle.

\subsection{Protocol 1}

\begin{enumerate}
\item The throttle requests a slot for the locomotive \gls{address} by sending either a \textbf{GetLocoSlotDataSAdrP1} or \textbf{GetLocoSlotDataLAdrP1} \gls{Command} to the command station. Which one depends on what type of address the locomotive's decoder is programmed to use. 
\item If a slot has been previously loaded with the locomotive's address, then the command station will return a \textbf{LocoSlotDataP1} \gls{Response}.
\item If the locomotive's address is not currently in a slot, then the command station will load the new locomotive address into a Free slot, with speed equal to zero, direction forwards, functions off and default decoder mode, and return a \textbf{LocoSlotDataP1} \gls{Response}. The default decoder mode is determined by the command station's OpSw21-OpSw23 settings.\index{decoder mode}
\item If there are no Free slots to load the new locomotive address into, the command station with return a \textbf{NoFreeSlotsP1} \gls{Response} and this procedure is terminated.
\item The throttle must then examine the slot data bytes to work out how to process the command station response.
\item If the slot state is New, Common or Idle then the throttle requests a ``null move" operation by sending the command station a \textbf{MoveSlotsP1} \gls{Command}. The command station returns a \textbf{LocoSlotDataP1} \gls{Response}. 
\item The \textbf{SetLocoSlotDataP1} \gls{Command} can be used at this time to change the decoder mode from that of the default.
\item The throttle will then be able to update speed, direction and function information. Whenever slot information is changed in an active slot, the slot is flagged to be updated as the next DCC packet sent to the track. 
\end{enumerate}

\subsection{Protocol 2}

\begin{enumerate}
\item The throttle requests a slot for the locomotive \gls{address} by sending either a \textbf{GetLocoSlotDataSAdrP2} or \textbf{GetLocoSlotDataLAdrP2} \gls{Command} to the command station. Which one depends on what type of address the locomotive's decoder is programmed to use. 
\item If a slot has been previously loaded with the locomotive's address, then the command station will return a \textbf{LocoSlotDataP2} \gls{Response}.
\item If the locomotive's address is not currently in a slot, then the command station will load the new locomotive address into a Free slot, with speed equal to zero, direction forwards, functions off and default decoder mode, and return a \textbf{LocoSlotDataP2} \gls{Response}. The default decoder mode is determined by the command station's OpSw21-OpSw23 settings.
\item If there are no Free slots to load the new locomotive address into, the command station with return a \textbf{NoFreeSlotsP2} \gls{Response} and this procedure is terminated.
\item The throttle must then examine the slot data bytes to work out how to process the command station response.
\item If the slot state is New, Common or Idle then the throttle requests a ``null move" operation by sending the command station a \textbf{MoveSlotsP2} \gls{Command}. The command station returns a \textbf{LocoSlotDataP2} \gls{Response}. 
\item If the slot state is In-Use and the slot's \gls{Throttle ID} does not match that of the throttle then the throttle should ask the user if they wish to ``steal?" the slot. If the answer is no then this procedure is terminated.
\item The throttle now takes ownership of the slot by updating the slot's Throttle ID to that of the throttle and writing the updated slot data to the command station by sending a \textbf{SetLocoSlotDataP2} \gls{Command}. If the request is successful then the command station will return a \textbf{setSlotDataOKP2} \gls{Response}. The \textbf{SetLocoSlotDataP2} can also be used to change the decoder mode from that of the default.
\item The throttle will then be able to update speed, direction and function information. Whenever slot information is changed in an active slot, the slot is flagged to be updated as the next DCC packet sent to the track. If the slot was stolen from another throttle then the other throttle will no longer be able to command the locomotive.
\end{enumerate}

Example:

\begin{verbatim}
getLocoSlotDataSAdrP2
     0xbe 0x00 0x17 0x56 
     
locoSlotDataP2
     0xe6 0x15 0x01 0x05 0x03 0x17 0x00 0x47 0x00 0x00 
     0x00 0x00 0x00 0x00 0x00 0x00 0x00 0x00 0x00 0x00 0x5b 

moveSlotsP2
     0xd4 0x39 0x05 0x01 0x05 0x13 

locoSlotDataP2
     0xe6 0x15 0x01 0x05 0x33 0x17 0x00 0x47 0x00 0x00 
     0x00 0x00 0x00 0x00 0x00 0x00 0x00 0x00 0x00 0x00 0x6b 

setLocoSlotDataP2
     0xee 0x15 0x01 0x05 0x33 0x17 0x00 0x47 0x00 0x00 
     0x00 0x00 0x00 0x00 0x00 0x00 0x00 0x00 0x6d 0x52 0x5c 

setSlotDataOKP2
     0xb4 0x6e 0x7f 0x5a 
\end{verbatim}
\normalsize

\subsection{Purging}\index{Purging}
If a device disconnects from the network and so does not access or reference a slot within the system purge time, the command station will force the un-accessed slot to \gls{Common} status so other system devices can use the slot. The typical purge time of a command station is about 200 seconds. A good ``ping" or slot update activity is about every 100 seconds, i.e. if a user makes no change to a throttle/slot within 100 seconds, the throttle/device should automatically send another speed update at the current speed to reset the purge timeout for that slot. Purging behaviour can be modified by adjusting the command station's OpSw13-OpSw15 settings.

