% Activate the following line by filling in the right side. If for example the name of the root file is Main.tex, write
% "...root = Main.tex" if the chapter file is in the same directory, and "...root = ../Main.tex" if the chapter is in a subdirectory.
 
%!TEX root =  

\chapter[Network Protocol]{The Network Protocol}.  

\section{Overview}

Loconet is a peer to peer distributed network system on which all devices can monitor the network data flow. The network is event driven by different devices in time, and is not polled by a centralised controller in normal operation. The normal network state is idle, with no data traffic unless a device has information to send. With no traffic flow, the network is quiet.

The network data is sent in asynchronous format using 1 start bit, 8 data bits and 1 stop bit. The 8 bit data is transmitted least significant bit first. The bit times are 60.0 $\mu$S or 16,660 baud +/- 1.5\%. A computer can connect to a Digitrax USB interface at higher baud rates and the device will make the necessary conversion. Bytes may be transmitted back-to-back, with a start bit immediately following the stop bit of the previous character. 

Any message that has format or framing errors, data errors or is a fragment caused by noise glitches and does not completely follow the message format will be ignored by all receivers, and a new opcode will be scanned for re-synchronisation.

The \textbf{Busy} message is included to allow the command station to keep the network active whilst it is performing a task that requires a response, and entails a significant processing delay, i.e. it can ensure no new requests are started until it has responded to the last message. The \textbf{Busy} message should be simply stripped and ignored.

If a device disconnects from the network and so does not access or reference a slot within the system purge time, the command station will force the un-accessed slot to common status so other system devices can use the slot. The typical purge time of a command station is about 200 seconds. A good ``ping" or slot update activity is about every 100 seconds, i.e. if a user makes no change to a throttle/slot within 100 seconds, the throttle/device should automatically send another speed update at the current speed to reset the purge timeout for that slot.

\section{Message Format}

All the network communications are via multi-byte messages. The command station is defined as the device that is maintaining the refresh stack for DCC packet generation and is actively generating the DCC track data. Refresh of information is typically only performed for mobile decoders. Stationary type decoders are not refreshed and individual immediate commands are sent out to the track as requested.

The command station is only privileged in respect to performing the task of maintaining the locomotive refresh stack and generating DCC packets. In this way other network transactions may occur that the command station does not need to be involved with or understand, as long as they follow the message protocol and timing requirements. i.e. other devices may have a dialog on the network without disturbing or involving the command station. Devices on the network monitor the messages, check for format and data integrity and parse good messages to decode if action is required in the context. Devices such as throttles, input sensors, computer interfaces and control panels may generate the network messages without needing prompting or polling by a central controller.

Devices frequently will be added and removed from an operating the network. The devices and protocol are tolerant of electrical and data transients. The format chosen gives a good degree of data integrity, guaranteed quick network-state synchronisation, high data throughput, good distribution of access to many competing devices and low event latency. Also, the devices may be operated without need for unique ID or other requirements that can make network administration awkward.

The data bytes on the network are defined as 8 bit data with the most significant bit as an opcode flag bit. If the most significant bit, d7, is 1 then the 7 least significant bits are interpreted as a network opcode . The opcode byte may only occur once in a valid message and is the first byte of a message. The opcode does not necessarily uniquely identify a message type. Sometimes the opcode must be used in combination of other bits or bytes in the message. All the remaining bytes in the message must have a most significant bit of 0, including the last checksum byte. The checksum is the 1's complement of the byte wise exclusive or of all the bytes in the message, except the checksum itself. To validate data accuracy, all the bytes in a correctly formatted message are exclusive or'ed. If this resulting byte value is 0xFF, then the message data is accepted as good.

The opcodes may be examined to determine message length and if subsequent response message is required. Data bits d6 and d5 encode the message length. The message length includes the opcode and the checksum bytes. The bit d3 = 1 implies that a follow-on message or reply is expected. For variable byte messages  The byte following the opcode in the message is a 7 bit byte count.

\begin{tabular}{p{0.05\linewidth} p{0.05\linewidth}  p{0.05\linewidth}  p{0.05\linewidth}  p{0.05\linewidth}  p{0.05\linewidth}  p{0.05\linewidth}  p{0.05\linewidth} p{0.36\linewidth}} 
\underline{d7} & \underline{d6} & \underline{d5} & \underline{d4} & \underline{d3} & \underline{d2} & \underline{d1} & \underline{d0} & \\
1 & 0 & 0 & E & D & C & B & A & 2 byte message\\
1 & 0 & 1 & E & D & C & B & A & 4 byte message\\
1 & 1 & 0 & E & D & C & B & A & 6 byte message\\
1 & 1 & 1 & E & D & C & B & A & Variable length message.\\
\end{tabular}

The A,B,C,D,E are bits available to encode 32 opcodes per message length.

\section{Refresh Slots}

The command station's refresh stack is used to control the locomotives. The refresh stack is an array of read/write refresh slots. There are two protocols for manipulating the refresh slots. Protocol 1 allows up to 120 locomotive slots and each slot contains 10 bytes of data relating to the locomotive. Protocol 2 allows up to 960 locomotive slots and each slot contains 15 bytes of data relating to the locomotive. Not all command stations implement both protocols. A command station may also not implement the maximum number of locomotive slots for the protocols it supports. Where a command station implements both protocols messages from both protocols can be freely mixed. The user should check the TRK status bits to determine if protocol 1 and/or 2 are supported. In this document message mnemonics that are suffixed ``P1" belong to protocol 1 and those suffixed ``P2" belong to protocol 2. The slot number is a principal component of the protocol and is similar to a file handle. In addition to the locomotive slots there are slots reserved for system and command station control. These slots are numbered 120 to 127 (0x78 to 0x7F) and are encoded differently from the locomotive slots. Slot 124 (0x7C) is allocated for read/write access to the programming track and slot 127 (0x7F) contains the command station configuration settings.

\section{Standard Address Selection}

To request a mobile or locomotive decoder task in the refresh stack, a throttle device requests a slot for the locomotive address by sending either the \textbf{getLocoSlotDataSAdr} or \textbf{getLocoSlotDataLAdr} commands. Which one depends on what type of decoder address you are using - short 2 digit or long 4 diigit. The command station responds with \textbf{LocoSlotData} messaage that contains this locomotive address and all of its state information. If the address is currently not in any slot, the command station will load this new locomotive address into a new slot (speed=0, direction forwards, functions off and 128 step mode) and return this as a \textbf{LocoSlotData}. If no inactive slots are free to load the new locomotive address, the response will be the \textbf{Ack} with a fail code 0x00.

The throttle/computer must then examine the slot data bytes to work out how to process the command station response. If the slot status 1 byte shows the slot to be ``common'', ``idle" or ``new" the throttle may change the slot to ``in use" by performing a null move instruction on this slot (see \textbf{MoveSlots}). This activation mechanism is used to guarantee proper slot usage interlocking in a multi-user asynchronous environment.

If the slot return information shows the locomotive requested is ``in use" or up-consisted (i.e. the SL\_CONUP, bit 6 of slot status 1 = 1) the user should not use the slot. Any up-consisted locomotives must be unlinked before usage. Always process the result from the \textbf{LinkSlots} and \textbf{UnlinkSlots} commands, since the command station reserves the right to change the reply slot number and can reject the linking tasks under several circumstances. Verify the reply slot number and the link UP/DN bits in slot status 1 are as you expected.

The throttle will then be able to update speed, direction and function information. Whenever slot information is changed in an active slot , the slot is flagged to be updated as the next DCC packet sent to the track. If the slot is part of linked consist slots the whole consist chain is updated consecutively.

If a throttle is disconnected from the the Network, upon reconnection (if the throttle retains the slot state from before disconnection) it will request the full status of the slot it was previously using. If the reported status and speed, function data etc., from the command station exactly matches the remembered slot state the throttle will continue using the slot. If the slot data does not match, the throttle will assume the slot was purged free by the system and will go through the setup log on procedure again.

With this procedure the throttle does not need to have a unique ID number. slot addresses do not imply they contain any particular locomotive address. The system can be mapped such that the slot address matches the locomotive address within, if the user directly reads and writes to slots without using the command station to allocate locomotive addresses.

\section{Messages}

The following information is provided for each of the messages:

\underline{Description:}

Description of the message's function.

\underline{Protocol:}

Which protocol the message belongs to.

\underline{Group:}

Which message size group the message belongs to.

\underline{Opcode:}

The opcode mnemonic.

\underline{Type:}

The message type - broadcast, command, response, or message.

\underline{Encoding:} 

How the message is encoded byte by byte.

\underline{Response:} 

The response expected from a command message, if applicable.

\underline{Signature:}

The bits and bytes that must be tested to determine the message's unique type.

\underline{Notes:} 

Any notes.

\newpage
\subsection{Ack}

\underline{Description:}

This message provides a response code from a command.

\underline{Protocol:}

1

\underline{Group:}

4-Byte Message

\underline{Opcode:}

OPC\_LONG\_ACK

\underline{Type:}

Response

\underline{Encoding:} 

Byte 0:

\begin{tabular}{p{0.4\linewidth} p{0.15\linewidth} p{0.38\linewidth}} 

\begin{tabular}{|p{0.3cm}|p{0.3cm}|p{0.3cm}|p{0.3cm}|p{0.3cm}|p{0.3cm}|p{0.3cm}|p{0.3cm}|}
\hline
1 & 0 & 1 & 1 & 0 & 1 & 0 & 0\\
\hline
\end{tabular}
& 0xB4 & Opcode.\\
\end{tabular}

Byte 1:

\begin{tabular}{p{0.4\linewidth} p{0.15\linewidth} p{0.38\linewidth}} 

\begin{tabular}{|p{0.3cm}|p{0.3cm}|p{0.3cm}|p{0.3cm}|p{0.3cm}|p{0.3cm}|p{0.3cm}|p{0.3cm}|}
\hline
0 & n & n & n & n & n & n & n\\
\hline
\end{tabular}
& $<$LOPC$>$ & Opcode of the command that this message is a response to with the most significant bit set to 0.\\
\end{tabular}

Byte 2:

\begin{tabular}{p{0.4\linewidth} p{0.15\linewidth} p{0.38\linewidth}} 

\begin{tabular}{|p{0.3cm}|p{0.3cm}|p{0.3cm}|p{0.3cm}|p{0.3cm}|p{0.3cm}|p{0.3cm}|p{0.3cm}|}
\hline
0 & n & n & n & n & n & n & n\\
\hline
\end{tabular}
& $<$ACK1$>$ & Response code.\\
\end{tabular}

Byte 3:

\begin{tabular}{p{0.4\linewidth} p{0.15\linewidth} p{0.38\linewidth}} 

\begin{tabular}{|p{0.3cm}|p{0.3cm}|p{0.3cm}|p{0.3cm}|p{0.3cm}|p{0.3cm}|p{0.3cm}|p{0.3cm}|}
\hline
0 & n & n & n & n & n & n & n\\
\hline
\end{tabular}
& $<$CHK$>$ & Checksum.
\end{tabular}

\underline{Response:} 

None.

\underline{Signature:}

Byte 0:

\begin{tabular}{p{0.4\linewidth} p{0.38\linewidth}} 

\begin{tabular}{|p{0.3cm}|p{0.3cm}|p{0.3cm}|p{0.3cm}|p{0.3cm}|p{0.3cm}|p{0.3cm}|p{0.3cm}|}
\hline
1 & 0 & 1 & 1 & 0 & 1 & 0 & 0\\
\hline
\end{tabular}
& 0xB4\\
\end{tabular}

\underline{Notes:} 

\begin{tabular}{p{0.3\linewidth} p{0.10\linewidth} p{0.10\linewidth} p{0.40\linewidth}}
\underline{Command} & \underline{$<$LOPC$>$} & \underline{$<$ACK1$>$} & \underline{Meaning}\\
OPC\_SW\_ACK & 0x3D & 0x00 & DCS100 FIFO is full, command rejected.\\
OPC\_SW\_ACK & 0x3D & 0x7F &  DCS100 command accepted.\\
MoveSlotsP1 & 0x3A & 0x00 & Illegal move.\\
OPC\_LINK\_SLOTS & 0x39 & 0x00 & Invalid link, link failed.\\
OPC\_SW\_REQ & 0x30 & 0x00 & Command failed.\\
GetLocoSlotDataSAdrP1 & 0x3F & 0x00 & No free slot, command failed.\\
GetLocoSlotDataLAdrP1 & 0x3F & 0x00 & No free slot, command failed.\\
OPC\_IMM\_PACKET & 0x7D & 0x7F & Command OK, if not limited master.\\
OPC\_IMM\_PACKET & 0x7E & $<$lim address$>$ & Command OK, if limited master.\\
OPC\_IMM\_PACKET & 0x7D & 0x00 & Internal buffer busy or full.\\
OPC\_WR\_SL\_DATA\_V2 & 0x6E & 0x7F & Command OK.\\
\end{tabular}

\rule{15.1cm}{0.4pt}


\newpage
\subsection{Busy}

\rule{15.1cm}{0.4pt}

\underline{Description:}

The \textbf{Busy} broadcast message allows the command station to keep the network active whilst it is performing a task that requires a response, and entails a significant processing delay, i.e. it can ensure no new requests are started until it has responded to the last message. The \textbf{Busy} message should be simply stripped and ignored.

\underline{Group:}

2-Byte Message

\underline{Opcode:}

OPC\_BUSY

\underline{Type:}

Broadcast

\underline{Encoding:} 

Byte 0:

\begin{tabular}{p{0.4\linewidth} p{0.15\linewidth} p{0.38\linewidth}} 

\begin{tabular}{|p{0.3cm}|p{0.3cm}|p{0.3cm}|p{0.3cm}|p{0.3cm}|p{0.3cm}|p{0.3cm}|p{0.3cm}|}
\hline
1 & 0 & 0 & 0 & 0 & 0 & 0 & 1\\
\hline
\end{tabular}
& 0x81 & Opcode.\\
\end{tabular}

Byte 1:

\begin{tabular}{p{0.4\linewidth} p{0.15\linewidth} p{0.38\linewidth}} 

\begin{tabular}{|p{0.3cm}|p{0.3cm}|p{0.3cm}|p{0.3cm}|p{0.3cm}|p{0.3cm}|p{0.3cm}|p{0.3cm}|}
\hline
0 & 1 & 1 & 1 & 1 & 1 & 1 & 0\\
\hline
\end{tabular}
& 0x7E & Checksum.
\end{tabular}

\underline{Response:} 

None

\underline{Signature:}

Byte 0:

\begin{tabular}{p{0.4\linewidth} p{0.15\linewidth} p{0.38\linewidth}} 

\begin{tabular}{|p{0.3cm}|p{0.3cm}|p{0.3cm}|p{0.3cm}|p{0.3cm}|p{0.3cm}|p{0.3cm}|p{0.3cm}|}
\hline
1 & 0 & 0 & 0 & 0 & 0 & 0 & 1\\
\hline
\end{tabular}
& 0x81 & \\
\end{tabular}

\underline{Notes:} 

None.

\rule{15.1cm}{0.4pt}


\newpage
\section{CfgSlotDataP1}\index{CfgSlotDataP1}

\rule{15.1cm}{0.4pt}

\underline{Description:}

This \gls{Response} provides the current command station configuration slot data. It is returned by the command station in response to the \textbf{GetCfgSlotDataP1} \gls{Command}.

\underline{Protocol:}

1

\underline{Group:}

Variable-Byte Message

\underline{Opcode:}

OPC\_SL\_RD\_DATA

\underline{Type:}

\gls{Response}

\underline{Encoding:} 

Byte 0:

\begin{tabular}{p{0.4\linewidth} p{0.15\linewidth} p{0.38\linewidth}} 

\begin{tabular}{|p{0.3cm}|p{0.3cm}|p{0.3cm}|p{0.3cm}|p{0.3cm}|p{0.3cm}|p{0.3cm}|p{0.3cm}|}
\hline
1 & 1 & 1 & 0 & 0 & 1 & 1 & 1\\
\hline
\end{tabular}
& 0xE7 & Opcode.\\
\end{tabular}

Byte 1:

\begin{tabular}{p{0.4\linewidth} p{0.15\linewidth} p{0.38\linewidth}} 

\begin{tabular}{|p{0.3cm}|p{0.3cm}|p{0.3cm}|p{0.3cm}|p{0.3cm}|p{0.3cm}|p{0.3cm}|p{0.3cm}|}
\hline
0 & 0 & 0 & 0 & 1 & 1 & 1 & 0\\
\hline
\end{tabular}
& 0x0E & Message length (14 bytes).\\
\end{tabular}


OST1 to OST6 encode the command station's option switch table. The narrative is based upon information in the the DCS210 and DCS240 user manuals. A bit value of 1 means that the switch is closed and a value of 0 means that a switch is thrown. OpSw 8, OpSw 16, OpSw 24, OpSw 32 and OpSw 40 cannot be read due to bit 7 being cleared in the message format. The manual shows these switches as defaulting to thrown, i.e. 0, and are flagged in all cases except OpSw 40 as ``do not change".

Byte 2:

\begin{tabular}{p{0.4\linewidth} p{0.15\linewidth} p{0.376\linewidth}} 

\begin{tabular}{|p{0.3cm}|p{0.3cm}|p{0.3cm}|p{0.3cm}|p{0.3cm}|p{0.3cm}|p{0.3cm}|p{0.3cm}|}
\hline
0 & 1 & 1 & 1 & 1 & 1 & 1 & 1\\
\hline
\end{tabular}
& 0x7F & Configuration slot number.\\
\end{tabular}

Byte 3:

\begin{tabular}{p{0.4\linewidth} p{0.15\linewidth} p{0.38\linewidth}} 

\begin{tabular}{|p{0.3cm}|p{0.3cm}|p{0.3cm}|p{0.3cm}|p{0.3cm}|p{0.3cm}|p{0.3cm}|p{0.3cm}|}
\hline
0 & d6 & d5 & d4 & d3 & d2 & d1 & d0\\
\hline
\end{tabular}
& $<$OST1$>$ & Option switch table byte 1.\\
\end{tabular}

\begin{tabular}{p{0.05\linewidth} p{0.125\linewidth} p{0.1\linewidth} p{0.625\linewidth}} 
\underline{Bit} & \underline{Switch \#} & \underline{Default} & \underline{Effect on system operation}\\
d6 & OpSw 07 & t & do not change\\
d5 & OpSw 06 & t & t = check for decoder before programming\\
& & & c = program without checking for device\\ 
d4 & OpSw 05 & t & do not change\\
d3 & OpSw 04 & t & do not change\\
d2 & OpSw 03 & t & t = command station's booster normal\\
& & & c = command station's booster is auto reversing\\
d1 & OpSw 02 & t & t = command station mode\\
& & & c = booster only mode.\\
d0 & OpSw 01 & t & do not change.\\
\end{tabular}

Byte 4:

\begin{tabular}{p{0.4\linewidth} p{0.15\linewidth} p{0.38\linewidth}} 

\begin{tabular}{|p{0.3cm}|p{0.3cm}|p{0.3cm}|p{0.3cm}|p{0.3cm}|p{0.3cm}|p{0.3cm}|p{0.3cm}|}
\hline
0 & d6 & d5 & d4 & d3 & d2 & d1 & d0\\
\hline
\end{tabular}
& $<$OST2$>$ & Option switch table byte 2.\\
\end{tabular}

\begin{tabular}{p{0.05\linewidth} p{0.125\linewidth} p{0.1\linewidth} p{0.625\linewidth}} 
\underline{Bit} & \underline{Switch \#} & \underline{Default} & \underline{Effect on system operation}\\
d6 & OpSw 15 & t & t = purging will not change loco speed\\
&&& c = purging will force a loco to 0 speed\\
d5 & OpSw 14 & t & t = loco address purging enabled\\
&&& c = loco address purging disabled\\
d4 & OpSw 13 & t & t = loco address purge time 200 seconds\\
& & & c = loco address purge time 600 seconds\\
d3 & OpSw 12 & t & do not change\\
d2 & OpSw 11 & t & do not change\\
d1 & OpSw 10 & c & do not change\\
d0 & OpSw 09 & c & do not change\\
\end{tabular}

Byte 5:

\begin{tabular}{p{0.4\linewidth} p{0.15\linewidth} p{0.38\linewidth}} 

\begin{tabular}{|p{0.3cm}|p{0.3cm}|p{0.3cm}|p{0.3cm}|p{0.3cm}|p{0.3cm}|p{0.3cm}|p{0.3cm}|}
\hline
0 & d6 & d5 & d4 & d3 & d2 & d1 & d0\\
\hline
\end{tabular}
& $<$OST3$>$ & Option switch table byte 3.\\
\end{tabular}

\begin{tabular}{p{0.05\linewidth} p{0.125\linewidth} p{0.1\linewidth} p{0.625\linewidth}} 
\underline{Bit} & \underline{Switch \#} & \underline{Default} & \underline{Effect on system operation}\\
d6 & OpSw 23 & t & SW23\\
d5 & OpSw 22 & c & SW22\\ 
d4 & OpSw 21 & c & SW21\\
d3 & OpSw 20 & t & t = enable address 0x00 or analog stretching for conventional locos\\
&&&. c = disable address 0x00 or analog stretching for conventional locos\\
d2 & OpSw 19 & t & do not change\\
d1 & OpSw 18 & t & t = normal command station booster short circuit shutdown time\\
&&& c = extended command station booster short circuit shutdown time\\
d0 & OpSw 17 & t & t = automatic advanced decode (FX) consists are enabled\\
&&& c = automatic advanced decode (FX) consists are disabled\\
\end{tabular}

\begin{tabular}{p{0.05\linewidth} p{0.05\linewidth} p{0.05\linewidth} p{0.5\linewidth}} 
\underline{SW21} & \underline{SW22} & \underline{SW23} & \underline{Global system default type for new locos}\\
t & t & t & 28 step mode\\
t & t & c & reserved\\
t & c & t & 14 step mode\\
t & c & c & reserved\\
c & t & t & reserved\\
c & t & c & reserved\\
c & c & t & 128 step mode\\
c & c & c & 128 step FX mode\\
\end{tabular}

Byte 6:

\begin{tabular}{p{0.4\linewidth} p{0.15\linewidth} p{0.38\linewidth}} 

\begin{tabular}{|p{0.3cm}|p{0.3cm}|p{0.3cm}|p{0.3cm}|p{0.3cm}|p{0.3cm}|p{0.3cm}|p{0.3cm}|}
\hline
0 & d6 & d5 & d4 & d3 & d2 & d1 & d0\\
\hline
\end{tabular}
& $<$OST4$>$ & Option switch table byte 4.\\
\end{tabular}

\begin{tabular}{p{0.05\linewidth} p{0.125\linewidth} p{0.1\linewidth} p{0.625\linewidth}} 
\underline{Bit} & \underline{Switch \#} & \underline{Default} & \underline{Effect on system operation}\\
d6 & OpSw 31 & t & t = normal route/switch output rate when not trinary\\
&&& c = fast route/switch output rate when not trinary\\ 
d5 & OpSw 30 & t & do not change\\ 
d4 & OpSw 29 & t & do not change\\
d3 & OpSw 28 & t & t = enable interrogate commands at power on\\
&&& c = disable interrogate commands at power on\\
d2 & OpSw 27 & t & t = enable normal switch commands, a.k.a. the ``Bushby bit"\\
&&& c = disable normal switch commands, a.k.a. the ``Bushby bit" (allows attached computer to handle switch control logic)\\
d1 & OpSw 26 & c & t = disable routes\\
&&& c = enable routes\\
d0 & OpSw 25 & t & t = enable route echo over the Network\\
&&& c = disbale route echo over the Network\\
\end{tabular}

Byte 7:

\begin{tabular}{p{0.4\linewidth} p{0.15\linewidth} p{0.38\linewidth}} 

\begin{tabular}{|p{0.3cm}|p{0.3cm}|p{0.3cm}|p{0.3cm}|p{0.3cm}|p{0.3cm}|p{0.3cm}|p{0.3cm}|}
\hline
0 & d6 & d5 & d4 & d3 & d2 & d1 & d0\\
\hline
\end{tabular}
& $<$TRK$>$ & Global system track status.\\
\end{tabular}

\begin{tabular}{p{0.05\linewidth} p{0.6\linewidth}} 
d6 & 1 means this command station implements version 2 slot commands. This can be turned off on the DCS240 by setting the OpSw 44 to be closed.\\
d5 & Reserved. Set to 0.\\
d4 & Reserved. Set to 0.\\
d3 & 1 means the programming track is busy.\\
d2 & 1 means this master implements the Network version 1.1 capability,  0 means the master is a DT200.\\
d1 & 0 means the track is paused, broadcast an emergency stop.\\
d0 & 1 means the DCC packets are on in the master, global power up.\\
\end{tabular}

Byte 8:

\begin{tabular}{p{0.4\linewidth} p{0.15\linewidth} p{0.38\linewidth}} 

\begin{tabular}{|p{0.3cm}|p{0.3cm}|p{0.3cm}|p{0.3cm}|p{0.3cm}|p{0.3cm}|p{0.3cm}|p{0.3cm}|}
\hline
0 & d6 & d5 & d4 & d3 & d2 & d1 & d0\\
\hline
\end{tabular}
& $<$OST5$>$ & Option switch table byte 5.\\
\end{tabular}

\begin{tabular}{p{0.05\linewidth} p{0.125\linewidth} p{0.1\linewidth} p{0.625\linewidth}} 
\underline{Bit} & \underline{Switch \#} & \underline{Default} & \underline{Effect on system operation}\\
d6 & OpSw 39 & t & c = clear all internal memory states, including OpSw 36 and 37\\
d5 & OpSw 38 & t & t = loco reset button activates OpSw 39\\
&&& c = loco reset activates slot zero\\ 
d4 & OpSw 37 & t & c = clears all routes\\
d3 & OpSw 36 & t & c = clears all mobile decoder info and consists\\
d2 & OpSw 35 & t & t = enables loco reset buttone\\
&&& c = disable loco reset button\\
d1 & OpSw 34 & t & t = disallow track to power up to run state, if set to run prior to power up\\
&& & c = allow track to power up to run state, if set to run prior to power up\\
d0 & OpSw 33 & c & t = track power off at power on\\
&&& c = allow track power to restore to prior state at power on\\
\end{tabular}

Byte 9:

\begin{tabular}{p{0.4\linewidth} p{0.15\linewidth} p{0.38\linewidth}} 

\begin{tabular}{|p{0.3cm}|p{0.3cm}|p{0.3cm}|p{0.3cm}|p{0.3cm}|p{0.3cm}|p{0.3cm}|p{0.3cm}|}
\hline
0 & d6 & d5 & d4 & d3 & d2 & d1 & d0\\
\hline
\end{tabular}
& $<$OST6$>$ & Option switch table byte 6.\\
\end{tabular}

\begin{tabular}{p{0.05\linewidth} p{0.125\linewidth} p{0.1\linewidth} p{0.625\linewidth}} 
\underline{Bit} & \underline{Switch \#} & \underline{Default} & \underline{Effect on system operation}\\
d6 & OpSw 47 & t & t = normal program track setting\\
&&& c = program track is brake generator when not programming. Braking is DCC set to speed 0 (not emergency stop) for address 0, light on, broadcast to all addresses.\\
d5 & OpSw 46 & t & do not change\\ 
d4 & OpSw 45 & t & t = enable reply for switch state request\\
&&& c = disable reply for switch state request\\
d3 & OpSw 44 & t & do not change (DCS210)\\
    & OpSw 44 & t & maximum slots to 400 (DCS240) and enable protocol 2 support\\
    & OpSw 44 & c & maximum slots to 120 (DCS240) and disable protocol 2 support\\
d2 & OpSw 43 & t & t = enable the Network update of command station's track status\\
&&& c = disable the Network update of command station's track status\\
d1 & OpSw 42 & t & t = enable 2 short beeps when loco address purged\\
&&& c = disable 2 short beeps when loco address purged\\
d0 & OpSw 41 & t & t = diagnostic click disabled\\
&&& c = diagnostic click when valid the Network commands incoming and routes being output\\
\end{tabular}

Byte 10:

\begin{tabular}{p{0.4\linewidth} p{0.15\linewidth} p{0.38\linewidth}} 

\begin{tabular}{|p{0.3cm}|p{0.3cm}|p{0.3cm}|p{0.3cm}|p{0.3cm}|p{0.3cm}|p{0.3cm}|p{0.3cm}|}
\hline
0 & d6 & d5 & d4 & d3 & d2 & d1 & d0\\
\hline
\end{tabular}
&  & Unknown.\\
\end{tabular}

\begin{tabular}{p{0.05\linewidth} p{0.6\linewidth}} 
d6 & \\
d5 & \\
d4 & \\
d3 & \\
d2 & \\
d1 & \\
d0 & \\
\end{tabular}

Byte 11:

\begin{tabular}{p{0.4\linewidth} p{0.15\linewidth} p{0.38\linewidth}} 

\begin{tabular}{|p{0.3cm}|p{0.3cm}|p{0.3cm}|p{0.3cm}|p{0.3cm}|p{0.3cm}|p{0.3cm}|p{0.3cm}|}
\hline
0 & n & n & n & n & n & n & n\\
\hline
\end{tabular}
& $<$CSM$>$ & Product code.\\
\end{tabular}

\begin{tabular}{p{0.2\linewidth} p{0.6\linewidth}} 
\underline{Product Code} & \underline{Model}\\
0x1B & DCS210\\
0x1C & DCS240\\
\end{tabular}

Byte 12:

\begin{tabular}{p{0.4\linewidth} p{0.15\linewidth} p{0.38\linewidth}} 

\begin{tabular}{|p{0.3cm}|p{0.3cm}|p{0.3cm}|p{0.3cm}|p{0.3cm}|p{0.3cm}|p{0.3cm}|p{0.3cm}|}
\hline
0 & d6 & d5 & d4 & d3 & d2 & d1 & d0\\
\hline
\end{tabular}
&  & Unknown.\\
\end{tabular}

\begin{tabular}{p{0.05\linewidth} p{0.6\linewidth}} 
d6 & \\
d5 & \\
d4 & \\
d3 & \\
d2 & \\
d1 & \\
d0 & \\
\end{tabular}



Byte 13:

\begin{tabular}{p{0.4\linewidth} p{0.15\linewidth} p{0.37\linewidth}} 

\begin{tabular}{|p{0.3cm}|p{0.3cm}|p{0.3cm}|p{0.3cm}|p{0.3cm}|p{0.3cm}|p{0.3cm}|p{0.3cm}|}
\hline
0 & n & n & n & n & n & n & n\\
\hline
\end{tabular}
& $<$CHK$>$ & Checksum.\\
\end{tabular}

\underline{Response:} 

None.

\underline{Signature:}

Byte 0:

\begin{tabular}{p{0.4\linewidth} p{0.38\linewidth}} 

\begin{tabular}{|p{0.3cm}|p{0.3cm}|p{0.3cm}|p{0.3cm}|p{0.3cm}|p{0.3cm}|p{0.3cm}|p{0.3cm}|}
\hline
1 & 1 & 1 & 0 & 0 & 1 & 1 & 1\\
\hline
\end{tabular}
& 0xE7 \\
\end{tabular}

Byte 1:

\begin{tabular}{p{0.4\linewidth} p{0.38\linewidth}} 

\begin{tabular}{|p{0.3cm}|p{0.3cm}|p{0.3cm}|p{0.3cm}|p{0.3cm}|p{0.3cm}|p{0.3cm}|p{0.3cm}|}
\hline
0 & 0 & 0 & 0 & 1 & 1 & 1 & 0\\
\hline
\end{tabular}
& 0x0E \\
\end{tabular}

Byte 2:

\begin{tabular}{p{0.4\linewidth} p{0.376\linewidth}} 

\begin{tabular}{|p{0.3cm}|p{0.3cm}|p{0.3cm}|p{0.3cm}|p{0.3cm}|p{0.3cm}|p{0.3cm}|p{0.3cm}|}
\hline
0 & 1 & 1 & 1 & 1 & 1 & 1 & 1\\
\hline
\end{tabular}
& 0x7F \\
\end{tabular}

Byte 7:

\begin{tabular}{p{0.4\linewidth} p{0.376\linewidth}} 

\begin{tabular}{|p{0.3cm}|p{0.3cm}|p{0.3cm}|p{0.3cm}|p{0.3cm}|p{0.3cm}|p{0.3cm}|p{0.3cm}|}
\hline
0 & $\times$ & 0 & 0 & $\times$ & $\times$ & $\times$ & $\times$\\
\hline
\end{tabular}
&  \\
\end{tabular}

\underline{Notes:} 

None.

\rule{15.1cm}{0.4pt}

\input{Chapter1-ConsistDirF0F4}
\input{Chapter1-GetBrdOpSw}
\newpage
\subsection{GetCfgSlotData}

\underline{Description:}

This command requests the configuration slot data. The command station responds with a \textbf{CfgSlotDataP1} message.

\underline{Protocol:}

1

\underline{Group:}

4-Byte Message

\underline{Opcode:}

OPC\_RQ\_SL\_DATA

\underline{Type:}

Command

\underline{Encoding:} 

Byte 0:

\begin{tabular}{p{0.4\linewidth} p{0.15\linewidth} p{0.38\linewidth}} 

\begin{tabular}{|p{0.3cm}|p{0.3cm}|p{0.3cm}|p{0.3cm}|p{0.3cm}|p{0.3cm}|p{0.3cm}|p{0.3cm}|}
\hline
1 & 0 & 1 & 1 & 1 & 0 & 1 & 1\\
\hline
\end{tabular}
& 0xBB & Opcode.\\
\end{tabular}

Byte 1:

\begin{tabular}{p{0.4\linewidth} p{0.15\linewidth} p{0.38\linewidth}} 

\begin{tabular}{|p{0.3cm}|p{0.3cm}|p{0.3cm}|p{0.3cm}|p{0.3cm}|p{0.3cm}|p{0.3cm}|p{0.3cm}|}
\hline
0 & 1 & 1 & 1 & 1 & 1 & 1 & 1\\
\hline
\end{tabular}
& 0x7F & \\
\end{tabular}

Byte 2:

\begin{tabular}{p{0.4\linewidth} p{0.15\linewidth} p{0.38\linewidth}} 

\begin{tabular}{|p{0.3cm}|p{0.3cm}|p{0.3cm}|p{0.3cm}|p{0.3cm}|p{0.3cm}|p{0.3cm}|p{0.3cm}|}
\hline
0 & 0 & 0 & 0 & 0 & 0 & 0 & 0\\
\hline
\end{tabular}
& 0x00 & \\
\end{tabular}

Byte 3:

\begin{tabular}{p{0.4\linewidth} p{0.15\linewidth} p{0.38\linewidth}} 

\begin{tabular}{|p{0.3cm}|p{0.3cm}|p{0.3cm}|p{0.3cm}|p{0.3cm}|p{0.3cm}|p{0.3cm}|p{0.3cm}|}
\hline
0 & n & n & n & n & n & n & n\\
\hline
\end{tabular}
& $<$CHK$>$ & Checksum.

\end{tabular}

\underline{Response:} 

\textbf{CfgSlotDataP1}

\underline{Signature:}

Byte 0:

\begin{tabular}{p{0.4\linewidth} p{0.15\linewidth} p{0.38\linewidth}} 

\begin{tabular}{|p{0.3cm}|p{0.3cm}|p{0.3cm}|p{0.3cm}|p{0.3cm}|p{0.3cm}|p{0.3cm}|p{0.3cm}|}
\hline
1 & 0 & 1 & 1 & 1 & 0 & 1 & 1\\
\hline
\end{tabular}
& 0xBB & \\
\end{tabular}

Byte 1:

\begin{tabular}{p{0.4\linewidth} p{0.15\linewidth} p{0.38\linewidth}} 

\begin{tabular}{|p{0.3cm}|p{0.3cm}|p{0.3cm}|p{0.3cm}|p{0.3cm}|p{0.3cm}|p{0.3cm}|p{0.3cm}|}
\hline
0 & 1 & 1 & 1 & 1 & 1 & 1 & 1\\
\hline
\end{tabular}
& 0x7F & \\
\end{tabular}

Byte 2:

\begin{tabular}{p{0.4\linewidth} p{0.15\linewidth} p{0.38\linewidth}} 

\begin{tabular}{|p{0.3cm}|p{0.3cm}|p{0.3cm}|p{0.3cm}|p{0.3cm}|p{0.3cm}|p{0.3cm}|p{0.3cm}|}
\hline
0 & 0 & 0 & 0 & 0 & 0 & 0 & 0\\
\hline
\end{tabular}
& 0x00 & \\
\end{tabular}

\underline{Notes:} 

None.

\rule{15.1cm}{0.4pt}

\newpage
\subsection{GetInterfaceData}\index{GetInterfaceData}

\rule{15.1cm}{0.4pt}

\underline{Description:}

This \gls{Command} is sent by a computer to request an \textbf{InterfaceData} \gls{Response} from the attached network interface device. 

\underline{Group:}

2-Byte Message

\underline{Opcode:}

OPC\_BUSY

\underline{Type:}

Command

\underline{Applicable Hardware}:

Digitrax PR4 and DCS240.

\underline{Encoding:} 

Byte 0:

\begin{tabular}{p{0.4\linewidth} p{0.15\linewidth} p{0.38\linewidth}} 

\begin{tabular}{|p{0.3cm}|p{0.3cm}|p{0.3cm}|p{0.3cm}|p{0.3cm}|p{0.3cm}|p{0.3cm}|p{0.3cm}|}
\hline
1 & 0 & 0 & 0 & 0 & 0 & 0 & 1\\
\hline
\end{tabular}
& 0x81 & Opcode.\\
\end{tabular}

Byte 1:

\begin{tabular}{p{0.4\linewidth} p{0.15\linewidth} p{0.38\linewidth}} 

\begin{tabular}{|p{0.3cm}|p{0.3cm}|p{0.3cm}|p{0.3cm}|p{0.3cm}|p{0.3cm}|p{0.3cm}|p{0.3cm}|}
\hline
0 & 1 & 1 & 1 & 1 & 1 & 1 & 0\\
\hline
\end{tabular}
& 0x7E & Checksum.\\
\end{tabular}

\underline{Response:} 

Interface device returns an \textbf{InterfaceData} response.

\underline{Signature:}

\underline{Notes:} 

None.

\rule{15.1cm}{0.4pt}


\newpage
\subsection{GetLocoSlotDataLAdrP1}\index{GetLocoSlotDataLAdrP1}

\rule{15.1cm}{0.4pt}

\underline{Description:}

This \gls{Command} requests a slot for the selected locomotive address. If the locomotive \gls{address} is found in the slot table then the command station returns an \textbf{LocoSlotDataP1} \gls{Response} with the slot information. If it is not found then the command station will put the locomotive \gls{address} into a free slot and then return an \textbf{LocoSlotDataP1} \gls{Response} with the slot information. If there are no free slots then the command station returns a \textbf{NoFreeSlotsP1} \gls{Response}.

The command station will generate \gls{NMRA} 14 bit or long address packets for the locomotive. The \gls{address} must be in the range 128 to 9983.

\underline{Protocol:}

1

\underline{Group:} 

4-Byte Message

\underline{Opcode:}

OPC\_LOCO\_ADR

\underline{Type:}

\gls{Command}

\underline{Encoding:} 

Byte 0:

\begin{tabular}{p{0.4\linewidth} p{0.15\linewidth} p{0.38\linewidth}} 

\begin{tabular}{|p{0.3cm}|p{0.3cm}|p{0.3cm}|p{0.3cm}|p{0.3cm}|p{0.3cm}|p{0.3cm}|p{0.3cm}|}
\hline
1 & 0 & 1 & 1 & 1 & 1 & 1 & 1\\
\hline
\end{tabular}
& 0xBF & Opcode.\\
\end{tabular}

Byte 1:

\begin{tabular}{p{0.4\linewidth} p{0.15\linewidth} p{0.38\linewidth}} 

\begin{tabular}{|p{0.3cm}|p{0.3cm}|p{0.3cm}|p{0.3cm}|p{0.3cm}|p{0.3cm}|p{0.3cm}|p{0.3cm}|}
\hline
0 & n & n & n & n & n & n & n\\
\hline
\end{tabular}
& $<$ADR2$>$ & Address high 7 bits.\\
\end{tabular}

Byte 2:

\begin{tabular}{p{0.4\linewidth} p{0.15\linewidth} p{0.38\linewidth}} 

\begin{tabular}{|p{0.3cm}|p{0.3cm}|p{0.3cm}|p{0.3cm}|p{0.3cm}|p{0.3cm}|p{0.3cm}|p{0.3cm}|}
\hline
0 & n & n & n & n & n & n & n\\
\hline
\end{tabular}
& $<$ADR$>$ & Address low 7 bits.\\
\end{tabular}

Byte 3:

\begin{tabular}{p{0.4\linewidth} p{0.15\linewidth} p{0.38\linewidth}} 

\begin{tabular}{|p{0.3cm}|p{0.3cm}|p{0.3cm}|p{0.3cm}|p{0.3cm}|p{0.3cm}|p{0.3cm}|p{0.3cm}|}
\hline
0 & n & n & n & n & n & n & n\\
\hline
\end{tabular}
& $<$CHK$>$ & Checksum.\\

\end{tabular}

\underline{Response:} 

\textbf{LocoSlotDataP1} if success, otherwise \textbf{NoFreeSlotsP1}

\underline{Signature:}

Byte 0:

\begin{tabular}{p{0.4\linewidth} p{0.38\linewidth}} 

\begin{tabular}{|p{0.3cm}|p{0.3cm}|p{0.3cm}|p{0.3cm}|p{0.3cm}|p{0.3cm}|p{0.3cm}|p{0.3cm}|}
\hline
1 & 0 & 1 & 1 & 1 & 1 & 1 & 1\\
\hline
\end{tabular}
& 0xBF \\
\end{tabular}

Byte 1:

\begin{tabular}{p{0.4\linewidth} p{0.38\linewidth}} 

\begin{tabular}{|p{0.3cm}|p{0.3cm}|p{0.3cm}|p{0.3cm}|p{0.3cm}|p{0.3cm}|p{0.3cm}|p{0.3cm}|}
\hline
0 & n & n & n & n & n & n & n\\
\hline
\end{tabular}
&  not equal to 0\\
\end{tabular}

\underline{Notes:} 

This command is not supported by the Digitrax DT200 command station.

\rule{15.1cm}{0.4pt}

\newpage
\section{GetLocoSlotDataLAdrP2}\index{GetLocoSlotDataLAdrP2}

\rule{15.1cm}{0.4pt}

\underline{Description:}

This \gls{Command} requests a slot\index{Slot} for the selected locomotive \gls{address}. If the locomotive address is found in the slot table then the command station returns a \textbf{LocoSlotDataP2} \gls{Response} with the slot information. If it is not found then the command station will put the locomotive address into a \gls{Free} slot and then return an \textbf{LocoSlotDataP2} \gls{Response} with the slot information. If there are no free slots then the command station returns a \textbf{NoFreeSlotsP2} \gls{Response}.

The command station will generate \gls{NMRA} 14 bit or long address packets for the locomotive. The \gls{address} must be in the range 128 to 9983.

\underline{Protocol:}

2

\underline{Group:} 

4-Byte Message

\underline{Opcode:}

OPC\_LOCO\_ADR\_P2 (unofficial mnemonic)

\underline{Type:}

\gls{Command}

\underline{Encoding:} 

Byte 0:

\begin{tabular}{p{0.4\linewidth} p{0.15\linewidth} p{0.38\linewidth}} 

\begin{tabular}{|p{0.3cm}|p{0.3cm}|p{0.3cm}|p{0.3cm}|p{0.3cm}|p{0.3cm}|p{0.3cm}|p{0.3cm}|}
\hline
1 & 0 & 1 & 1 & 1 & 1 & 1 & 0\\
\hline
\end{tabular}
& 0xBE & Opcode.\\
\end{tabular}

Byte 1:

\begin{tabular}{p{0.4\linewidth} p{0.15\linewidth} p{0.38\linewidth}} 

\begin{tabular}{|p{0.3cm}|p{0.3cm}|p{0.3cm}|p{0.3cm}|p{0.3cm}|p{0.3cm}|p{0.3cm}|p{0.3cm}|}
\hline
0 & n & n & n & n & n & n & n\\
\hline
\end{tabular}
& $<$ADR2$>$ & Address high 7 bits.\\
\end{tabular}

Byte 2:

\begin{tabular}{p{0.4\linewidth} p{0.15\linewidth} p{0.38\linewidth}} 

\begin{tabular}{|p{0.3cm}|p{0.3cm}|p{0.3cm}|p{0.3cm}|p{0.3cm}|p{0.3cm}|p{0.3cm}|p{0.3cm}|}
\hline
0 & n & n & n & n & n & n & n\\
\hline
\end{tabular}
& $<$ADR$>$ & Address low 7 bits.\\
\end{tabular}

Byte 3:

\begin{tabular}{p{0.4\linewidth} p{0.15\linewidth} p{0.38\linewidth}} 

\begin{tabular}{|p{0.3cm}|p{0.3cm}|p{0.3cm}|p{0.3cm}|p{0.3cm}|p{0.3cm}|p{0.3cm}|p{0.3cm}|}
\hline
0 & n & n & n & n & n & n & n\\
\hline
\end{tabular}
& $<$CHK$>$ & Checksum.\\

\end{tabular}

\underline{Response:} 

\textbf{LocoSlotDataP2} if success, otherwise \textbf{NoFreeSlotsP2}.

\underline{Signature:}

Byte 0:

\begin{tabular}{p{0.4\linewidth} p{0.38\linewidth}} 

\begin{tabular}{|p{0.3cm}|p{0.3cm}|p{0.3cm}|p{0.3cm}|p{0.3cm}|p{0.3cm}|p{0.3cm}|p{0.3cm}|}
\hline
1 & 0 & 1 & 1 & 1 & 1 & 1 & 0\\
\hline
\end{tabular}
& 0xBE\\
\end{tabular}

Byte 1:

\begin{tabular}{p{0.4\linewidth} p{0.38\linewidth}} 

\begin{tabular}{|p{0.3cm}|p{0.3cm}|p{0.3cm}|p{0.3cm}|p{0.3cm}|p{0.3cm}|p{0.3cm}|p{0.3cm}|}
\hline
0 & n & n & n & n & n & n & n\\
\hline
\end{tabular}
& not equal to 0\\
\end{tabular}

\underline{Notes:} 

This \gls{Command} can be disabled by the command station OpSw66.

\rule{15.1cm}{0.4pt}

\newpage
\section{GetLocoSlotDataP1}\index{GetLocoSlotDataP1}\index{Slot}

\underline{Description:}

This \gls{Command} requests the locomotive slot data for the specified slot. The command station responds with a \textbf{LocoSlotDataP1} \gls{Response}.

\underline{Protocol:}

1

\underline{Group:}

4-Byte Message

\underline{Opcode:}

OPC\_RQ\_SL\_DATA

\underline{Type:}

\gls{Command}

\underline{Encoding:} 

Byte 0:

\begin{tabular}{p{0.4\linewidth} p{0.15\linewidth} p{0.38\linewidth}} 

\begin{tabular}{|p{0.3cm}|p{0.3cm}|p{0.3cm}|p{0.3cm}|p{0.3cm}|p{0.3cm}|p{0.3cm}|p{0.3cm}|}
\hline
1 & 0 & 1 & 1 & 1 & 0 & 1 & 1\\
\hline
\end{tabular}
& 0xBB & Opcode.\\
\end{tabular}

Byte 1:

\begin{tabular}{p{0.4\linewidth} p{0.15\linewidth} p{0.38\linewidth}} 

\begin{tabular}{|p{0.3cm}|p{0.3cm}|p{0.3cm}|p{0.3cm}|p{0.3cm}|p{0.3cm}|p{0.3cm}|p{0.3cm}|}
\hline
0 & n & n & n & n & n & n & n\\
\hline
\end{tabular}
& $<$SLOT\#$>$ & Slot number in the range 0x00 to 0x77.\\
\end{tabular}

Byte 2:

\begin{tabular}{p{0.4\linewidth} p{0.15\linewidth} p{0.38\linewidth}} 

\begin{tabular}{|p{0.3cm}|p{0.3cm}|p{0.3cm}|p{0.3cm}|p{0.3cm}|p{0.3cm}|p{0.3cm}|p{0.3cm}|}
\hline
0 & 0 & 0 & 0 & 0 & 0 & 0 & d0\\
\hline
\end{tabular}
& 0x00 & \\
\end{tabular}

Byte 3:

\begin{tabular}{p{0.4\linewidth} p{0.15\linewidth} p{0.38\linewidth}} 

\begin{tabular}{|p{0.3cm}|p{0.3cm}|p{0.3cm}|p{0.3cm}|p{0.3cm}|p{0.3cm}|p{0.3cm}|p{0.3cm}|}
\hline
0 & n & n & n & n & n & n & n\\
\hline
\end{tabular}
& $<$CHK$>$ & Checksum.\\

\end{tabular}

\underline{Response:} 

\textbf{LocoSlotDataP1} or \textbf{SlotNotImplemented}

\underline{Signature:}

Byte 0:

\begin{tabular}{p{0.4\linewidth} p{0.15\linewidth} p{0.38\linewidth}} 

\begin{tabular}{|p{0.3cm}|p{0.3cm}|p{0.3cm}|p{0.3cm}|p{0.3cm}|p{0.3cm}|p{0.3cm}|p{0.3cm}|}
\hline
1 & 0 & 1 & 1 & 1 & 0 & 1 & 1\\
\hline
\end{tabular}
& 0xBB & \\
\end{tabular}

Byte 1:

\begin{tabular}{p{0.4\linewidth} p{0.38\linewidth}} 

\begin{tabular}{|p{0.3cm}|p{0.3cm}|p{0.3cm}|p{0.3cm}|p{0.3cm}|p{0.3cm}|p{0.3cm}|p{0.3cm}|}
\hline
0 & n & n & n & n & n & n & n\\
\hline
\end{tabular}
& less than 0x78\\
\end{tabular}

Byte 2:

\begin{tabular}{p{0.4\linewidth} p{0.15\linewidth} p{0.38\linewidth}} 

\begin{tabular}{|p{0.3cm}|p{0.3cm}|p{0.3cm}|p{0.3cm}|p{0.3cm}|p{0.3cm}|p{0.3cm}|p{0.3cm}|}
\hline
0 & 0 & 0 & 0 & 0 & 0 & 0 & d0\\
\hline
\end{tabular}
& 0x00 & \\
\end{tabular}

\underline{Notes:} 

None.

\rule{15.1cm}{0.4pt}

\newpage
\subsection{GetLocoSlotDataP2}

\underline{Description:}

This command requests the locomotive slot data for the specified slot number. The command station responds with a \textbf{LocoSlotDataP2} message.

\underline{Protocol:}

2

\underline{Group:}

4-Byte Message

\underline{Direction:} \hspace{0.05cm} $\rightarrow$ Switch

\underline{Opcode:}

OPC\_RQ\_SL\_DATA

\underline{Type:}

Command

\underline{Encoding:} 

Byte 0:

\begin{tabular}{p{0.4\linewidth} p{0.15\linewidth} p{0.38\linewidth}} 

\begin{tabular}{|p{0.3cm}|p{0.3cm}|p{0.3cm}|p{0.3cm}|p{0.3cm}|p{0.3cm}|p{0.3cm}|p{0.3cm}|}
\hline
1 & 0 & 1 & 1 & 1 & 0 & 1 & 1\\
\hline
\end{tabular}
& 0xBB & Opcode.\\
\end{tabular}

Byte 1:

\begin{tabular}{p{0.4\linewidth} p{0.15\linewidth} p{0.38\linewidth}} 

\begin{tabular}{|p{0.3cm}|p{0.3cm}|p{0.3cm}|p{0.3cm}|p{0.3cm}|p{0.3cm}|p{0.3cm}|p{0.3cm}|}
\hline
0 & n & n & n & n & n & n & n\\
\hline
\end{tabular}
& $<$SLOT\#$>$ & Slot number in the range 0x00 to 0x77.\\
\end{tabular}

Byte 2:

\begin{tabular}{p{0.4\linewidth} p{0.15\linewidth} p{0.38\linewidth}} 

\begin{tabular}{|p{0.3cm}|p{0.3cm}|p{0.3cm}|p{0.3cm}|p{0.3cm}|p{0.3cm}|p{0.3cm}|p{0.3cm}|}
\hline
0 & 1 & 0 & 0 & d3 & d2 & d1 & d0\\
\hline
\end{tabular}
& $<$SLOTP$>$ & Bits d2 to d0 contain the slot page number in the range 0x0 to 0x7. The bit d3 does something but its function is not yet known.\\
\end{tabular}

Byte 3:

\begin{tabular}{p{0.4\linewidth} p{0.15\linewidth} p{0.38\linewidth}} 

\begin{tabular}{|p{0.3cm}|p{0.3cm}|p{0.3cm}|p{0.3cm}|p{0.3cm}|p{0.3cm}|p{0.3cm}|p{0.3cm}|}
\hline
0 & n & n & n & n & n & n & n\\
\hline
\end{tabular}
& $<$CHK$>$ & Checksum.

\end{tabular}

\underline{Response:} 

\textbf{LocoSlotDataP2}

\underline{Signature:}

Byte 0:

\begin{tabular}{p{0.4\linewidth} p{0.15\linewidth} p{0.38\linewidth}} 

\begin{tabular}{|p{0.3cm}|p{0.3cm}|p{0.3cm}|p{0.3cm}|p{0.3cm}|p{0.3cm}|p{0.3cm}|p{0.3cm}|}
\hline
1 & 0 & 1 & 1 & 1 & 0 & 1 & 1\\
\hline
\end{tabular}
& 0xBB &\\
\end{tabular}

Byte 1:

\begin{tabular}{p{0.4\linewidth} p{0.38\linewidth}} 

\begin{tabular}{|p{0.3cm}|p{0.3cm}|p{0.3cm}|p{0.3cm}|p{0.3cm}|p{0.3cm}|p{0.3cm}|p{0.3cm}|}
\hline
0 & n & n & n & n & n & n & n\\
\hline
\end{tabular}
& less than 0x78\\
\end{tabular}

Byte 2:

\begin{tabular}{p{0.4\linewidth} p{0.15\linewidth} p{0.38\linewidth}} 

\begin{tabular}{|p{0.3cm}|p{0.3cm}|p{0.3cm}|p{0.3cm}|p{0.3cm}|p{0.3cm}|p{0.3cm}|p{0.3cm}|}
\hline
0 & 1 & 0 & 0 & $\times$ & $\times$ & $\times$ & $\times$\\
\hline
\end{tabular}
& \\
\end{tabular}

\underline{Notes:} 

None.

\rule{15.1cm}{0.4pt}

\newpage
\subsection{GetLocoSlotDataSAdrP1}

\rule{15.1cm}{0.4pt}

\underline{Description:}

This command requests the slot number for the selected locomotive address. If the locomotive is found in the slot table then the command station returns an \textbf{LocoSlotDataP1} message with the slot information. If it is not found then the command station will put the locomotive into a free slot and then return an \textbf{LocoSlotDataP1} message with the slot information. If there are no free slots then the command station returns an \textbf{Ack} containing a response code.

The command station will generate NMRA 7 bit or short address packets for the locomotive. The address has the range 0 to 127. The analog locomotive is selected with address 0.  

\underline{Protocol:}

1

\underline{Group:} 

4-Byte Message

\underline{Opcode:}

OPC\_LOCO\_ADR

\underline{Type:}

Command

\underline{Encoding:} 

Byte 0:

\begin{tabular}{p{0.4\linewidth} p{0.15\linewidth} p{0.38\linewidth}} 

\begin{tabular}{|p{0.3cm}|p{0.3cm}|p{0.3cm}|p{0.3cm}|p{0.3cm}|p{0.3cm}|p{0.3cm}|p{0.3cm}|}
\hline
1 & 0 & 1 & 1 & 1 & 1 & 1 & 1\\
\hline
\end{tabular}
& 0xBF & Opcode.\\
\end{tabular}

Byte 1:

\begin{tabular}{p{0.4\linewidth} p{0.15\linewidth} p{0.38\linewidth}} 

\begin{tabular}{|p{0.3cm}|p{0.3cm}|p{0.3cm}|p{0.3cm}|p{0.3cm}|p{0.3cm}|p{0.3cm}|p{0.3cm}|}
\hline
0 & 0 & 0 & 0 & 0 & 0 & 0 & 0\\
\hline
\end{tabular}
& 0x00 &\\
\end{tabular}

Byte 2:

\begin{tabular}{p{0.4\linewidth} p{0.15\linewidth} p{0.38\linewidth}} 

\begin{tabular}{|p{0.3cm}|p{0.3cm}|p{0.3cm}|p{0.3cm}|p{0.3cm}|p{0.3cm}|p{0.3cm}|p{0.3cm}|}
\hline
0 & n & n & n & n & n & n & n\\
\hline
\end{tabular}
& $<$ADR$>$ & Short address in the range 0 to 127.\\
\end{tabular}

Byte 3:

\begin{tabular}{p{0.4\linewidth} p{0.15\linewidth} p{0.38\linewidth}} 

\begin{tabular}{|p{0.3cm}|p{0.3cm}|p{0.3cm}|p{0.3cm}|p{0.3cm}|p{0.3cm}|p{0.3cm}|p{0.3cm}|}
\hline
0 & n & n & n & n & n & n & n\\
\hline
\end{tabular}
& $<$CHK$>$ & Checksum.

\end{tabular}

\underline{Response:} 

\textbf{LocoSlotDataP1} if success, otherwise 

\textbf{Ack}

\begin{tabular}{p{0.10\linewidth} p{0.10\linewidth} p{0.40\linewidth}}
\underline{$<$LOPC$>$} & \underline{$<$ACK1$>$} & \underline{Meaning}\\
0x3F & 0x00 & No free slot, command failed.\\
\end{tabular}

\underline{Signature:}

Byte 0:

\begin{tabular}{p{0.4\linewidth} p{0.38\linewidth}} 

\begin{tabular}{|p{0.3cm}|p{0.3cm}|p{0.3cm}|p{0.3cm}|p{0.3cm}|p{0.3cm}|p{0.3cm}|p{0.3cm}|}
\hline
1 & 0 & 1 & 1 & 1 & 1 & 1 & 1\\
\hline
\end{tabular}
& 0xBF\\
\end{tabular}

Byte 1:

\begin{tabular}{p{0.4\linewidth} p{0.38\linewidth}} 

\begin{tabular}{|p{0.3cm}|p{0.3cm}|p{0.3cm}|p{0.3cm}|p{0.3cm}|p{0.3cm}|p{0.3cm}|p{0.3cm}|}
\hline
0 & 0 & 0 & 0 & 0 & 0 & 0 & 0\\
\hline
\end{tabular}
& 0x00 \\
\end{tabular}

\underline{Notes:} 

None.

\rule{15.1cm}{0.4pt}

\newpage
\section{GetLocoSlotDataSAdrP2}\index{GetLocoSlotDataSAdrP2}

\rule{15.1cm}{0.4pt}

\underline{Description:}

This \gls{Command} requests a slot\index{Slot} for the selected locomotive \gls{address}. If the locomotive address is found in the slot table then the command station returns a \textbf{LocoSlotDataP2} \gls{Response} with the slot information. If it is not found then the command station will put the locomotive address into a \gls{Free} slot and then return a \textbf{LocoSlotDataP2} \gls{Response} with the slot information. If there are no free slots then the command station returns a \textbf{NoFreeSlotsP2} \gls{Response}.

The command station will generate \gls{NMRA} 7 bit or short address packets for the locomotive. The \gls{address} has the range 0 to 127. The analog locomotive is selected with address 0.  

\underline{Protocol:}

2

\underline{Group:} 

4-Byte Message

\underline{Opcode:}

OPC\_LOCO\_ADR\_P2 (unofficial mnemonic)

\underline{Type:}

\gls{Command}

\underline{Encoding:} 

Byte 0:

\begin{tabular}{p{0.4\linewidth} p{0.15\linewidth} p{0.38\linewidth}} 

\begin{tabular}{|p{0.3cm}|p{0.3cm}|p{0.3cm}|p{0.3cm}|p{0.3cm}|p{0.3cm}|p{0.3cm}|p{0.3cm}|}
\hline
1 & 0 & 1 & 1 & 1 & 1 & 1 & 0\\
\hline
\end{tabular}
& 0xBE & Opcode.\\
\end{tabular}

Byte 1:

\begin{tabular}{p{0.4\linewidth} p{0.15\linewidth} p{0.38\linewidth}} 

\begin{tabular}{|p{0.3cm}|p{0.3cm}|p{0.3cm}|p{0.3cm}|p{0.3cm}|p{0.3cm}|p{0.3cm}|p{0.3cm}|}
\hline
0 & 0 & 0 & 0 & 0 & 0 & 0 & 0\\
\hline
\end{tabular}
& 0x00 &\\
\end{tabular}

Byte 2:

\begin{tabular}{p{0.4\linewidth} p{0.15\linewidth} p{0.38\linewidth}} 

\begin{tabular}{|p{0.3cm}|p{0.3cm}|p{0.3cm}|p{0.3cm}|p{0.3cm}|p{0.3cm}|p{0.3cm}|p{0.3cm}|}
\hline
0 & n & n & n & n & n & n & n\\
\hline
\end{tabular}
& $<$ADR$>$ & Short address in the range 0 to 127.\\
\end{tabular}

Byte 3:

\begin{tabular}{p{0.4\linewidth} p{0.15\linewidth} p{0.38\linewidth}} 

\begin{tabular}{|p{0.3cm}|p{0.3cm}|p{0.3cm}|p{0.3cm}|p{0.3cm}|p{0.3cm}|p{0.3cm}|p{0.3cm}|}
\hline
0 & n & n & n & n & n & n & n\\
\hline
\end{tabular}
& $<$CHK$>$ & Checksum.\\

\end{tabular}

\underline{Response:} 

\textbf{LocoSlotDataP2} if success, otherwise \textbf{NoFreeSlotsP2}

\underline{Signature:}

Byte 0:

\begin{tabular}{p{0.4\linewidth} p{0.38\linewidth}} 

\begin{tabular}{|p{0.3cm}|p{0.3cm}|p{0.3cm}|p{0.3cm}|p{0.3cm}|p{0.3cm}|p{0.3cm}|p{0.3cm}|}
\hline
1 & 0 & 1 & 1 & 1 & 1 & 1 & 0\\
\hline
\end{tabular}
& 0xBE \\
\end{tabular}

Byte 1:

\begin{tabular}{p{0.4\linewidth} p{0.38\linewidth}} 

\begin{tabular}{|p{0.3cm}|p{0.3cm}|p{0.3cm}|p{0.3cm}|p{0.3cm}|p{0.3cm}|p{0.3cm}|p{0.3cm}|}
\hline
0 & 0 & 0 & 0 & 0 & 0 & 0 & 0\\
\hline
\end{tabular}
& 0x00 \\
\end{tabular}

\underline{Notes:} 

This \gls{Command} can be disabled by the command station's OpSw66.

\rule{15.1cm}{0.4pt}

\newpage
\subsection{IMMPacket}

\underline{Description:}

Send n-byte DCC immediate packet.

\underline{Group:}

Variable-Byte Message

\underline{Opcode:}

OPC\_IMM\_PACKET

\underline{Type:}

Command

\underline{Encoding:} 

Byte 0:

\begin{tabular}{p{0.4\linewidth} p{0.15\linewidth} p{0.38\linewidth}} 

\begin{tabular}{|p{0.3cm}|p{0.3cm}|p{0.3cm}|p{0.3cm}|p{0.3cm}|p{0.3cm}|p{0.3cm}|p{0.3cm}|}
\hline
1 & 1 & 1 & 0 & 1 & 1 & 0 & 1\\
\hline
\end{tabular}
& 0xED & Opcode.\\
\end{tabular}

Byte 1:

\begin{tabular}{p{0.4\linewidth} p{0.15\linewidth} p{0.38\linewidth}} 

\begin{tabular}{|p{0.3cm}|p{0.3cm}|p{0.3cm}|p{0.3cm}|p{0.3cm}|p{0.3cm}|p{0.3cm}|p{0.3cm}|}
\hline
0 & 0 & 0 & 1 & 0 & 0 & 0 & 0\\
\hline
\end{tabular}
& 0x0B & Message length (11 bytes).\\
\end{tabular}

Byte 2:

\begin{tabular}{p{0.4\linewidth} p{0.15\linewidth} p{0.38\linewidth}} 

\begin{tabular}{|p{0.3cm}|p{0.3cm}|p{0.3cm}|p{0.3cm}|p{0.3cm}|p{0.3cm}|p{0.3cm}|p{0.3cm}|}
\hline
0 & 1 & 1 & 1 & 1 & 1 & 1 & 1\\
\hline
\end{tabular}
& 0x7F & \\
\end{tabular}

Byte 3:

\begin{tabular}{p{0.4\linewidth} p{0.15\linewidth} p{0.38\linewidth}} 

\begin{tabular}{|p{0.3cm}|p{0.3cm}|p{0.3cm}|p{0.3cm}|p{0.3cm}|p{0.3cm}|p{0.3cm}|p{0.3cm}|}
\hline
0 & d6 & d5 & d4 & 0 & d2 & d1 & d0\\
\hline
\end{tabular}
& $<$REPS$>$ & Number of immediate bytes and repeat count.
\end{tabular}

\begin{tabular}{p{0.05\linewidth} p{0.6\linewidth}} 
d6 & N2. Number of immediate bytes.\\
d5 & N1. Number of immediate bytes.\\
d4 & N0. Number of immediate bytes.\\
d2 & R2. Repeat count.\\
d1 & R1. Repeat count.\\
d0 & R0. Repeat count.\\
\end{tabular}

Byte 4:

\begin{tabular}{p{0.4\linewidth} p{0.15\linewidth} p{0.38\linewidth}} 

\begin{tabular}{|p{0.3cm}|p{0.3cm}|p{0.3cm}|p{0.3cm}|p{0.3cm}|p{0.3cm}|p{0.3cm}|p{0.3cm}|}
\hline
0 & 0 & 1 & d4 & d3 & d2 & d1 & d0\\
\hline
\end{tabular}
& $<$DHII$>$ & High bits of IM1 to IM5.
\end{tabular}

\begin{tabular}{p{0.05\linewidth} p{0.6\linewidth}} 
d4 & IM5.7. High bit.\\
d3 & IM4.7. High bit.\\
d2 & IM3.7. High bit.\\
d1 & IM2.7. High bit.\\
d0 & IM1.7. High bit.\\
\end{tabular}

Byte 5:

\begin{tabular}{p{0.4\linewidth} p{0.15\linewidth} p{0.38\linewidth}} 

\begin{tabular}{|p{0.3cm}|p{0.3cm}|p{0.3cm}|p{0.3cm}|p{0.3cm}|p{0.3cm}|p{0.3cm}|p{0.3cm}|}
\hline
0 & d6 & d5 & d4 & d3 & d2 & d1 & d0\\
\hline
\end{tabular}
& $<$IM1$>$ & Data item 1 low 7 bits.
\end{tabular}

Byte 6:

\begin{tabular}{p{0.4\linewidth} p{0.15\linewidth} p{0.38\linewidth}} 

\begin{tabular}{|p{0.3cm}|p{0.3cm}|p{0.3cm}|p{0.3cm}|p{0.3cm}|p{0.3cm}|p{0.3cm}|p{0.3cm}|}
\hline
0 & d6 & d5 & d4 & d3 & d2 & d1 & d0\\
\hline
\end{tabular}
& $<$IM2$>$ & Data item 2 low 7 bits.
\end{tabular}

Byte 7:

\begin{tabular}{p{0.4\linewidth} p{0.15\linewidth} p{0.38\linewidth}} 

\begin{tabular}{|p{0.3cm}|p{0.3cm}|p{0.3cm}|p{0.3cm}|p{0.3cm}|p{0.3cm}|p{0.3cm}|p{0.3cm}|}
\hline
0 & d6 & d5 & d4 & d3 & d2 & d1 & d0\\
\hline
\end{tabular}
& $<$IM3$>$ & Data item 3 low 7 bits.
\end{tabular}

Byte 8:

\begin{tabular}{p{0.4\linewidth} p{0.15\linewidth} p{0.38\linewidth}} 

\begin{tabular}{|p{0.3cm}|p{0.3cm}|p{0.3cm}|p{0.3cm}|p{0.3cm}|p{0.3cm}|p{0.3cm}|p{0.3cm}|}
\hline
0 & d6 & d5 & d4 & d3 & d2 & d1 & d0\\
\hline
\end{tabular}
& $<$IM4$>$ & Data item 4 low 7 bits.
\end{tabular}

Byte 9:

\begin{tabular}{p{0.4\linewidth} p{0.15\linewidth} p{0.38\linewidth}} 

\begin{tabular}{|p{0.3cm}|p{0.3cm}|p{0.3cm}|p{0.3cm}|p{0.3cm}|p{0.3cm}|p{0.3cm}|p{0.3cm}|}
\hline
0 & d6 & d5 & d4 & d3 & d2 & d1 & d0\\
\hline
\end{tabular}
& $<$IM5$>$ & Data item 5 low 7 bits.
\end{tabular}

Byte 10:

\begin{tabular}{p{0.4\linewidth} p{0.15\linewidth} p{0.38\linewidth}} 

\begin{tabular}{|p{0.3cm}|p{0.3cm}|p{0.3cm}|p{0.3cm}|p{0.3cm}|p{0.3cm}|p{0.3cm}|p{0.3cm}|}
\hline
0 & n & n & n & n & n & n & n\\
\hline
\end{tabular}
& $<$CHK$>$ & Checksum.

\end{tabular}

\underline{Response:} 

\textbf{Ack}.

\begin{tabular}{p{0.10\linewidth} p{0.20\linewidth} p{0.40\linewidth}}
\underline{$<$LOPC$>$} & \underline{$<$ACK1$>$} & \underline{Meaning}\\
0x7D & 0x7F & Command OK, if command station.\\
0x7E & $<$lim address$>$ & Command OK, if limited master.\\
0x7D & 0x00 & Internal buffer busy or full.\\
\end{tabular}

\underline{Signature:}

Byte 0:

\begin{tabular}{p{0.4\linewidth} p{0.38\linewidth}} 

\begin{tabular}{|p{0.3cm}|p{0.3cm}|p{0.3cm}|p{0.3cm}|p{0.3cm}|p{0.3cm}|p{0.3cm}|p{0.3cm}|}
\hline
1 & 1 & 1 & 0 & 1 & 1 & 0 & 1\\
\hline
\end{tabular}
& 0xED \\
\end{tabular}

Byte 1:

\begin{tabular}{p{0.4\linewidth} p{0.38\linewidth}} 

\begin{tabular}{|p{0.3cm}|p{0.3cm}|p{0.3cm}|p{0.3cm}|p{0.3cm}|p{0.3cm}|p{0.3cm}|p{0.3cm}|}
\hline
0 & 0 & 0 & 1 & 0 & 0 & 0 & 0\\
\hline
\end{tabular}
& 0x0B \\
\end{tabular}

Byte 2:

\begin{tabular}{p{0.4\linewidth} p{0.38\linewidth}} 

\begin{tabular}{|p{0.3cm}|p{0.3cm}|p{0.3cm}|p{0.3cm}|p{0.3cm}|p{0.3cm}|p{0.3cm}|p{0.3cm}|}
\hline
0 & 1 & 1 & 1 & 1 & 1 & 1 & 1\\
\hline
\end{tabular}
& 0x7F \\
\end{tabular}

Byte 3:

\begin{tabular}{p{0.4\linewidth} p{0.38\linewidth}} 

\begin{tabular}{|p{0.3cm}|p{0.3cm}|p{0.3cm}|p{0.3cm}|p{0.3cm}|p{0.3cm}|p{0.3cm}|p{0.3cm}|}
\hline
0 & $\times$ & $\times$ & $\times$ & 0 & $\times$ & $\times$ & $\times$\\
\hline
\end{tabular}
& \\
\end{tabular}

Byte 4:

\begin{tabular}{p{0.4\linewidth} p{0.38\linewidth}} 

\begin{tabular}{|p{0.3cm}|p{0.3cm}|p{0.3cm}|p{0.3cm}|p{0.3cm}|p{0.3cm}|p{0.3cm}|p{0.3cm}|}
\hline
0 & 0 & 1 & $\times$ & $\times$ & $\times$ & $\times$ & $\times$\\
\hline
\end{tabular}
& \\
\end{tabular}

\underline{Notes:} 

None.

\rule{15.1cm}{0.4pt}

\newpage
\subsection{InterfaceData}

\rule{15.1cm}{0.4pt}

\underline{Description:}

This is sent by an interface device in response to a \textbf{getInterfaceData} command.

\underline{Group:}

Variable-Byte Message

\underline{Opcode:}

OPC\_PEER\_XFER

\underline{Type:}

Response

\underline{Applicable Hardware}:

Digitrax PR4 and DCS240.

\underline{Encoding:} 

Byte 0:

\begin{tabular}{p{0.4\linewidth} p{0.15\linewidth} p{0.38\linewidth}} 

\begin{tabular}{|p{0.3cm}|p{0.3cm}|p{0.3cm}|p{0.3cm}|p{0.3cm}|p{0.3cm}|p{0.3cm}|p{0.3cm}|}
\hline
1 & 1 & 1 & 0 & 0 & 1 & 0 & 1\\
\hline
\end{tabular}
& 0xE5 & Opcode.\\
\end{tabular}

Byte 1:

\begin{tabular}{p{0.4\linewidth} p{0.15\linewidth} p{0.38\linewidth}} 

\begin{tabular}{|p{0.3cm}|p{0.3cm}|p{0.3cm}|p{0.3cm}|p{0.3cm}|p{0.3cm}|p{0.3cm}|p{0.3cm}|}
\hline
0 & 0 & 0 & 1 & 0 & 0 & 0 & 0\\
\hline
\end{tabular}
& 0x10 & Message length (16 bytes).\\
\end{tabular}

Byte 2:

\begin{tabular}{p{0.4\linewidth} p{0.15\linewidth} p{0.38\linewidth}} 

\begin{tabular}{|p{0.3cm}|p{0.3cm}|p{0.3cm}|p{0.3cm}|p{0.3cm}|p{0.3cm}|p{0.3cm}|p{0.3cm}|}
\hline
0 & 0 & 1 & 0 & 0 & 0 & 1 & 0\\
\hline
\end{tabular}
& 0x22 & \\
\end{tabular}

Byte 3:

\begin{tabular}{p{0.4\linewidth} p{0.15\linewidth} p{0.38\linewidth}} 

\begin{tabular}{|p{0.3cm}|p{0.3cm}|p{0.3cm}|p{0.3cm}|p{0.3cm}|p{0.3cm}|p{0.3cm}|p{0.3cm}|}
\hline
0 & 0 & 1 & 0 & 0 & 0 & 1 & 0\\
\hline
\end{tabular}
& 0x22 & \\
\end{tabular}

Byte 4:

\begin{tabular}{p{0.4\linewidth} p{0.15\linewidth} p{0.38\linewidth}} 

\begin{tabular}{|p{0.3cm}|p{0.3cm}|p{0.3cm}|p{0.3cm}|p{0.3cm}|p{0.3cm}|p{0.3cm}|p{0.3cm}|}
\hline
0 & 0 & 0 & 0 & 0 & 0 & 0 & 1\\
\hline
\end{tabular}
& 0x01 & \\
\end{tabular}

Byte 5:

\begin{tabular}{p{0.4\linewidth} p{0.15\linewidth} p{0.38\linewidth}} 

\begin{tabular}{|p{0.3cm}|p{0.3cm}|p{0.3cm}|p{0.3cm}|p{0.3cm}|p{0.3cm}|p{0.3cm}|p{0.3cm}|}
\hline
0 & 0 & 0 & 0 & 0 & 0 & 0 & 0\\
\hline
\end{tabular}
& 0x00 & \\
\end{tabular}

Byte 6:

\begin{tabular}{p{0.4\linewidth} p{0.15\linewidth} p{0.38\linewidth}} 

\begin{tabular}{|p{0.3cm}|p{0.3cm}|p{0.3cm}|p{0.3cm}|p{0.3cm}|p{0.3cm}|p{0.3cm}|p{0.3cm}|}
\hline
0 & n & n & n & n & n & n & n\\
\hline
\end{tabular}
& $<$D1$>$ & Serial Number low byte low 7 bits.\\
\end{tabular}

Byte 7:

\begin{tabular}{p{0.4\linewidth} p{0.15\linewidth} p{0.38\linewidth}} 

\begin{tabular}{|p{0.3cm}|p{0.3cm}|p{0.3cm}|p{0.3cm}|p{0.3cm}|p{0.3cm}|p{0.3cm}|p{0.3cm}|}
\hline
0 & n & n & n & n & n & n & n\\
\hline
\end{tabular}
& $<$D2$>$ & Serial Number high byte low 7 bits.\\
\end{tabular}

Byte 8:

\begin{tabular}{p{0.4\linewidth} p{0.15\linewidth} p{0.38\linewidth}} 

\begin{tabular}{|p{0.3cm}|p{0.3cm}|p{0.3cm}|p{0.3cm}|p{0.3cm}|p{0.3cm}|p{0.3cm}|p{0.3cm}|}
\hline
0 & n & n & n & n & n & n & n\\
\hline
\end{tabular}
& $<$D3$>$ & It contains a value but the meaning is unknown.\\
\end{tabular}

Byte 9:

\begin{tabular}{p{0.4\linewidth} p{0.15\linewidth} p{0.38\linewidth}} 

\begin{tabular}{|p{0.3cm}|p{0.3cm}|p{0.3cm}|p{0.3cm}|p{0.3cm}|p{0.3cm}|p{0.3cm}|p{0.3cm}|}
\hline
0 & n & n & n & n & n & n & n\\
\hline
\end{tabular}
& $<$D4$>$ & Unknown - set to zero for PR4 and DCS240.\\
\end{tabular}

Byte 10:

\begin{tabular}{p{0.4\linewidth} p{0.15\linewidth} p{0.38\linewidth}} 

\begin{tabular}{|p{0.3cm}|p{0.3cm}|p{0.3cm}|p{0.3cm}|p{0.3cm}|p{0.3cm}|p{0.3cm}|p{0.3cm}|}
\hline
0 & n & n & n & n & n & n & n\\
\hline
\end{tabular}
& $<$PXCT2$>$ & Unknown - set to zero for PR4 and DCS240.\\
\end{tabular}

Byte 11:

\begin{tabular}{p{0.4\linewidth} p{0.15\linewidth} p{0.38\linewidth}} 

\begin{tabular}{|p{0.3cm}|p{0.3cm}|p{0.3cm}|p{0.3cm}|p{0.3cm}|p{0.3cm}|p{0.3cm}|p{0.3cm}|}
\hline
0 & n & n & n & n & n & n & n\\
\hline
\end{tabular}
& $<$D5$>$ & Maybe hardware version.\\
\end{tabular}

Byte 12:

\begin{tabular}{p{0.4\linewidth} p{0.15\linewidth} p{0.38\linewidth}} 

\begin{tabular}{|p{0.3cm}|p{0.3cm}|p{0.3cm}|p{0.3cm}|p{0.3cm}|p{0.3cm}|p{0.3cm}|p{0.3cm}|}
\hline
0 & n & n & n & n & n & n & n\\
\hline
\end{tabular}
& $<$D6$>$ & Software version.\\
\end{tabular}

Byte 13:

\begin{tabular}{p{0.4\linewidth} p{0.15\linewidth} p{0.38\linewidth}} 

\begin{tabular}{|p{0.3cm}|p{0.3cm}|p{0.3cm}|p{0.3cm}|p{0.3cm}|p{0.3cm}|p{0.3cm}|p{0.3cm}|}
\hline
0 & n & n & n & n & n & n & n\\
\hline
\end{tabular}
& $<$D7$>$ & Maybe hardware version.\\
\end{tabular}

Byte 14:

\begin{tabular}{p{0.4\linewidth} p{0.15\linewidth} p{0.38\linewidth}} 

\begin{tabular}{|p{0.3cm}|p{0.3cm}|p{0.3cm}|p{0.3cm}|p{0.3cm}|p{0.3cm}|p{0.3cm}|p{0.3cm}|}
\hline
0 & n & n & n & n & n & n & n\\
\hline
\end{tabular}
& $<$D8$>$ & Product code.\\
\end{tabular}

Byte 15:

\begin{tabular}{p{0.4\linewidth} p{0.15\linewidth} p{0.38\linewidth}} 

\begin{tabular}{|p{0.3cm}|p{0.3cm}|p{0.3cm}|p{0.3cm}|p{0.3cm}|p{0.3cm}|p{0.3cm}|p{0.3cm}|}
\hline
0 & n & n & n & n & n & n & n\\
\hline
\end{tabular}
& $<$CHK$>$ & Checksum.\\
\end{tabular}

\underline{Response:} 

None

\underline{Signature:}

Byte 0:

\begin{tabular}{p{0.4\linewidth} p{0.38\linewidth}} 

\begin{tabular}{|p{0.3cm}|p{0.3cm}|p{0.3cm}|p{0.3cm}|p{0.3cm}|p{0.3cm}|p{0.3cm}|p{0.3cm}|}
\hline
1 & 1 & 1 & 0 & 0 & 1 & 0 & 1\\
\hline
\end{tabular}
& 0xE5 \\
\end{tabular}

Byte 1:

\begin{tabular}{p{0.4\linewidth} p{0.38\linewidth}} 

\begin{tabular}{|p{0.3cm}|p{0.3cm}|p{0.3cm}|p{0.3cm}|p{0.3cm}|p{0.3cm}|p{0.3cm}|p{0.3cm}|}
\hline
0 & 0 & 0 & 1 & 0 & 0 & 0 & 0\\
\hline
\end{tabular}
& 0x10 \\
\end{tabular}

Byte 2:

\begin{tabular}{p{0.4\linewidth} p{0.38\linewidth}} 

\begin{tabular}{|p{0.3cm}|p{0.3cm}|p{0.3cm}|p{0.3cm}|p{0.3cm}|p{0.3cm}|p{0.3cm}|p{0.3cm}|}
\hline
0 & 0 & 1 & 0 & 0 & 0 & 1 & 0\\
\hline
\end{tabular}
& 0x22 \\
\end{tabular}

Byte 3:

\begin{tabular}{p{0.4\linewidth} p{0.38\linewidth}} 

\begin{tabular}{|p{0.3cm}|p{0.3cm}|p{0.3cm}|p{0.3cm}|p{0.3cm}|p{0.3cm}|p{0.3cm}|p{0.3cm}|}
\hline
0 & 0 & 1 & 0 & 0 & 0 & 1 & 0\\
\hline
\end{tabular}
& 0x22 \\
\end{tabular}

Byte 4:

\begin{tabular}{p{0.4\linewidth} p{0.38\linewidth}} 

\begin{tabular}{|p{0.3cm}|p{0.3cm}|p{0.3cm}|p{0.3cm}|p{0.3cm}|p{0.3cm}|p{0.3cm}|p{0.3cm}|}
\hline
0 & 0 & 0 & 0 & 0 & 0 & 0 & 1\\
\hline
\end{tabular}
& 0x01 \\
\end{tabular}

Byte 5:

\begin{tabular}{p{0.4\linewidth} p{0.38\linewidth}} 

\begin{tabular}{|p{0.3cm}|p{0.3cm}|p{0.3cm}|p{0.3cm}|p{0.3cm}|p{0.3cm}|p{0.3cm}|p{0.3cm}|}
\hline
0 & 0 & 0 & 0 & 0 & 0 & 0 & 0\\
\hline
\end{tabular}
& 0x00\\
\end{tabular}

\underline{Notes:} 

\begin{verbatim}

PR4 #1

<D0> 0xe5 OPCODE
<D1> 0x10 LENGTH
<D2> 0x22 SRC
<D3> 0x22 DSTL
<D4> 0x01 DSTH
<D5> 0x00 PXCT1 <- I would have expected b4 = 1
<D6> 0x08 Serial Number Low Byte
<D7> 0x07 Serial Number High Byte - Actual serial number 0x0788
<D8> 0x16 
<D9> 0x00 
<D10> 0x00 PXCT2
<D11> 0x00 
<D12> 0x00 
<D13> 0x00 
<D14> 0x24 Product Code for PR4
<D15> 0x36  CHSUM

PR4 #2

<D0> 0xe5 OPCODE OPC_PEER_XFER
<D1> 0x10 LENGTH
<D2> 0x22 SRC
<D3> 0x22 DSTL
<D4> 0x01 DSTH
<D5> 0x00 PXCT1 
<D6> 0x57 Serial Number Low Byte
<D7> 0x13 Serial Number High Byte - Actual serial number 0x1357
<D8> 0x16 
<D9> 0x00 
<D10> 0x00 PXCT2
<D11> 0x00 
<D12> 0x00 
<D13> 0x00 
<D14> 0x24 Product Code for PR4
<D15> 0x7d CHKSUM

DCS240

<D0> 0xe5 OPCODE
<D1> 0x10 Length
<D2> 0x22 SRC
<D3> 0x22 DSTL
<D4> 0x01 DSTH
<D5> 0x00 PXCT1 <- I would have expected b4 to be 1
<D6> 0x2b Serial Number Low Byte
<D7> 0x0a  Serial Number High Byte - Actual serial number 0x0aab
<D8> 0x14 
<D9> 0x00 
<D10> 0x00 PXCT2
<D11> 0x01 Hardware Version?
<D12> 0x03 Software Version
<D13> 0x01 Hardware Version?
<D14> 0x1c Product Code for DCS240
<D15> 0x21

\end{verbatim}

\rule{15.1cm}{0.4pt}

\newpage
\subsection{LinkSlotsP1}

\underline{Description:}

This command links slot SL1 to slot SL2. The command station sets SL\_CONUP/DN flags appropriately. If the command was successful then a \textbf{LocoSlotDataP1} response will be returned. An invalid link will return a \textbf{Ack} with a response code of 0x00.

\underline{Protocol:}

1

\underline{Group:}

4-Byte Message

\underline{Opcode:}

OPC\_LINK\_SLOTS

\underline{Type:}

Command

\underline{Encoding:} 

Byte 0:

\begin{tabular}{p{0.4\linewidth} p{0.15\linewidth} p{0.38\linewidth}} 

\begin{tabular}{|p{0.3cm}|p{0.3cm}|p{0.3cm}|p{0.3cm}|p{0.3cm}|p{0.3cm}|p{0.3cm}|p{0.3cm}|}
\hline
1 & 0 & 1 & 1 & 1 & 0 & 0 & 1\\
\hline
\end{tabular}
& 0xB9 & Opcode.\\
\end{tabular}

Byte 1:

\begin{tabular}{p{0.4\linewidth} p{0.15\linewidth} p{0.38\linewidth}} 

\begin{tabular}{|p{0.3cm}|p{0.3cm}|p{0.3cm}|p{0.3cm}|p{0.3cm}|p{0.3cm}|p{0.3cm}|p{0.3cm}|}
\hline
0 & n & n & n & n & n & n & n\\
\hline
\end{tabular}
& $<$SL1$>$ & Slot number in the range 0x01 to 0x77.\\
\end{tabular}

Byte 2:

\begin{tabular}{p{0.4\linewidth} p{0.15\linewidth} p{0.38\linewidth}} 

\begin{tabular}{|p{0.3cm}|p{0.3cm}|p{0.3cm}|p{0.3cm}|p{0.3cm}|p{0.3cm}|p{0.3cm}|p{0.3cm}|}
\hline
0 & n & n & n & n & n & n & n\\
\hline
\end{tabular}
& $<$SL2$>$ & Slot number in the range 0x01 to 0x77.\\
\end{tabular}

Byte 3:

\begin{tabular}{p{0.4\linewidth} p{0.15\linewidth} p{0.38\linewidth}} 

\begin{tabular}{|p{0.3cm}|p{0.3cm}|p{0.3cm}|p{0.3cm}|p{0.3cm}|p{0.3cm}|p{0.3cm}|p{0.3cm}|}
\hline
0 & n & n & n & n & n & n & n\\
\hline
\end{tabular}
& $<$CHK$>$ & Checksum.
\end{tabular}

\underline{Response:} 

\textbf{LocoSlotDataP1} 

or 

\textbf{Ack}

\begin{tabular}{p{0.10\linewidth} p{0.10\linewidth} p{0.40\linewidth}}
\underline{$<$LOPC$>$} & \underline{$<$ACK1$>$} & \underline{Meaning}\\
0x39 & 0x00 & Invalid link, link failed.\\
\end{tabular}

\underline{Signature:}

Byte 0:

\begin{tabular}{p{0.4\linewidth} p{0.38\linewidth}} 

\begin{tabular}{|p{0.3cm}|p{0.3cm}|p{0.3cm}|p{0.3cm}|p{0.3cm}|p{0.3cm}|p{0.3cm}|p{0.3cm}|}
\hline
1 & 0 & 1 & 1 & 1 & 0 & 0 & 1\\
\hline
\end{tabular}
& 0xB9 \\
\end{tabular}

Byte 1:

\begin{tabular}{p{0.4\linewidth} p{0.38\linewidth}} 

\begin{tabular}{|p{0.3cm}|p{0.3cm}|p{0.3cm}|p{0.3cm}|p{0.3cm}|p{0.3cm}|p{0.3cm}|p{0.3cm}|}
\hline
0 & n & n & n & n & n & n & n\\
\hline
\end{tabular}
& in the range 0x01 to 0x77.\\
\end{tabular}

Byte 2:

\begin{tabular}{p{0.4\linewidth} p{0.38\linewidth}} 

\begin{tabular}{|p{0.3cm}|p{0.3cm}|p{0.3cm}|p{0.3cm}|p{0.3cm}|p{0.3cm}|p{0.3cm}|p{0.3cm}|}
\hline
0 & n & n & n & n & n & n & n\\
\hline
\end{tabular}
& in the range 0x01 to 0x77.\\
\end{tabular}

\underline{Notes:} 

None.

\rule{15.1cm}{0.4pt}

\newpage
\subsection{LocoBinStateP2}

\underline{Description:}

This command sets the locomotive's binary states with addresses in the range 1 to 32767. The address of 0 is a broadcast command and will set or reset all binary states.

\underline{Protocol:}

2?

\underline{Group:}

6-Byte Message

\underline{Opcode:}

OPC\_D4\_GROUP (Unofficial mnemonic)

\underline{Type:}

Command

\underline{Encoding:} 

Byte 0:

\begin{tabular}{p{0.4\linewidth} p{0.15\linewidth} p{0.38\linewidth}} 

\begin{tabular}{|p{0.3cm}|p{0.3cm}|p{0.3cm}|p{0.3cm}|p{0.3cm}|p{0.3cm}|p{0.3cm}|p{0.3cm}|}
\hline
1 & 1 & 0 & 1 & 0 & 1 & 0 & 0\\
\hline
\end{tabular}
& 0xD4 & Opcode.\\
\end{tabular}

Byte 1:

\begin{tabular}{p{0.4\linewidth} p{0.15\linewidth} p{0.38\linewidth}} 

\begin{tabular}{|p{0.3cm}|p{0.3cm}|p{0.3cm}|p{0.3cm}|p{0.3cm}|p{0.3cm}|p{0.3cm}|p{0.3cm}|}
\hline
0 & 0 & 0 & d4 & d3 & d2 & d1 & d0\\
\hline
\end{tabular}
& $<$SLOTP$>$ & Bits d2 to d0 contain the slot page number in the range 0x0 to 0x7. The bit d4 contains the function state where 1 means on and 0 means off. The bit d3 contains the high bit of the binary state address (bit 14).\\
\end{tabular}

Byte 2:

\begin{tabular}{p{0.4\linewidth} p{0.15\linewidth} p{0.38\linewidth}} 

\begin{tabular}{|p{0.3cm}|p{0.3cm}|p{0.3cm}|p{0.3cm}|p{0.3cm}|p{0.3cm}|p{0.3cm}|p{0.3cm}|}
\hline
0 & n & n & n & n & n & n & n\\
\hline
\end{tabular}
& $<$SLOT\#$>$ & Slot number.\\
\end{tabular}

Byte 3:

\begin{tabular}{p{0.4\linewidth} p{0.15\linewidth} p{0.38\linewidth}} 

\begin{tabular}{|p{0.3cm}|p{0.3cm}|p{0.3cm}|p{0.3cm}|p{0.3cm}|p{0.3cm}|p{0.3cm}|p{0.3cm}|}
\hline
0 & n & n & n & n & n & n & n\\
\hline
\end{tabular}
& $<$BSA0$>$ & Binary state address bits 0 to 6.\\
\end{tabular}

Byte 4:

\begin{tabular}{p{0.4\linewidth} p{0.15\linewidth} p{0.38\linewidth}} 

\begin{tabular}{|p{0.3cm}|p{0.3cm}|p{0.3cm}|p{0.3cm}|p{0.3cm}|p{0.3cm}|p{0.3cm}|p{0.3cm}|}
\hline
0 & n & n & n & n & n & n & n\\
\hline
\end{tabular}
& $<$BSA1$>$ & Binary state address bits 7 to 13.\\
\end{tabular}

Byte 5:

\begin{tabular}{p{0.4\linewidth} p{0.15\linewidth} p{0.38\linewidth}} 

\begin{tabular}{|p{0.3cm}|p{0.3cm}|p{0.3cm}|p{0.3cm}|p{0.3cm}|p{0.3cm}|p{0.3cm}|p{0.3cm}|}
\hline
0 & n & n & n & n & n & n & n\\
\hline
\end{tabular}
& $<$CHK$>$ & Checksum.
\end{tabular}

\underline{Response:} 

None.

\underline{Signature:}

Byte 0:

\begin{tabular}{p{0.4\linewidth} p{0.38\linewidth}} 

\begin{tabular}{|p{0.3cm}|p{0.3cm}|p{0.3cm}|p{0.3cm}|p{0.3cm}|p{0.3cm}|p{0.3cm}|p{0.3cm}|}
\hline
1 & 1 & 0 & 1 & 0 & 1 & 0 & 0\\
\hline
\end{tabular}
& 0xD4 \\
\end{tabular}

Byte 1:

\begin{tabular}{p{0.4\linewidth} p{0.38\linewidth}} 

\begin{tabular}{|p{0.3cm}|p{0.3cm}|p{0.3cm}|p{0.3cm}|p{0.3cm}|p{0.3cm}|p{0.3cm}|p{0.3cm}|}
\hline
0 & 0 & 0 & $\times$ & $\times$ & $\times$ & $\times$ & $\times$\\
\hline
\end{tabular}
& \\
\end{tabular}

\underline{Notes:} 

*** THIS HAS NOT BEEN TESTED ***

\rule{15.1cm}{0.4pt}

\newpage
\subsection{LocoDirF0F4P1}\index{LocoDirF0F4P1}\index{Direction}\index{Functions}

\underline{Description:}

This \gls{Command} requests the command station to set the locomotive's direction and function F0 to F4 states.

\underline{Protocol:}

1

\underline{Group:}

4-Byte Message

\underline{Opcode:}

OPC\_LOCO\_DIRF

\underline{Type:}

\gls{Command}

\underline{Encoding:} 

Byte 0:

\begin{tabular}{p{0.4\linewidth} p{0.15\linewidth} p{0.38\linewidth}} 

\begin{tabular}{|p{0.3cm}|p{0.3cm}|p{0.3cm}|p{0.3cm}|p{0.3cm}|p{0.3cm}|p{0.3cm}|p{0.3cm}|}
\hline
1 & 0 & 1 & 0 & 0 & 0 & 0 & 1\\
\hline
\end{tabular}
& 0xA1 & Opcode.\\
\end{tabular}

Byte 1:

\begin{tabular}{p{0.4\linewidth} p{0.15\linewidth} p{0.38\linewidth}} 

\begin{tabular}{|p{0.3cm}|p{0.3cm}|p{0.3cm}|p{0.3cm}|p{0.3cm}|p{0.3cm}|p{0.3cm}|p{0.3cm}|}
\hline
0 & n & n & n & n & n & n & n\\
\hline
\end{tabular}
& $<$SLOT\#$>$ & Slot number in the range 0x00 to 0x77.\\
\end{tabular}

Byte 2:

\begin{tabular}{p{0.4\linewidth} p{0.15\linewidth} p{0.38\linewidth}} 

\begin{tabular}{|p{0.3cm}|p{0.3cm}|p{0.3cm}|p{0.3cm}|p{0.3cm}|p{0.3cm}|p{0.3cm}|p{0.3cm}|}
\hline
0 & 0 & d5 & d4 & d3 & d2 & d1 & d0\\
\hline
\end{tabular}
& $<$DIRF$>$ & Locomotive's direction and state of functions F0 to F4.\\
& \\
\end{tabular}
\begin{tabular}{p{0.05\linewidth} p{0.95\linewidth}} 
d5 & Direction: 1 means forward and 0 means backwards.\\
d4 & F0 state: 1 means on and 0 means off.\\
d3 & F4 state: 1 means on and 0 means off.\\
d2 & F3 state: 1 means on and 0 means off.\\
d1 & F2 state: 1 means on and 0 means off.\\
d0 & F1 state: 1 means on and 0 means off.\\
\end{tabular}

Byte 3:

\begin{tabular}{p{0.4\linewidth} p{0.15\linewidth} p{0.38\linewidth}} 

\begin{tabular}{|p{0.3cm}|p{0.3cm}|p{0.3cm}|p{0.3cm}|p{0.3cm}|p{0.3cm}|p{0.3cm}|p{0.3cm}|}
\hline
0 & n & n & n & n & n & n & n\\
\hline
\end{tabular}
& $<$CHK$>$ & Checksum.\\
\end{tabular}

\underline{Response:} 

None.

\underline{Signature:}

Byte 0:

\begin{tabular}{p{0.4\linewidth} p{0.38\linewidth}} 

\begin{tabular}{|p{0.3cm}|p{0.3cm}|p{0.3cm}|p{0.3cm}|p{0.3cm}|p{0.3cm}|p{0.3cm}|p{0.3cm}|}
\hline
1 & 0 & 1 & 0 & 0 & 0 & 0 & 1\\
\hline
\end{tabular}
& 0xA1 \\
\end{tabular}

Byte 1:

\begin{tabular}{p{0.4\linewidth} p{0.38\linewidth}} 

\begin{tabular}{|p{0.3cm}|p{0.3cm}|p{0.3cm}|p{0.3cm}|p{0.3cm}|p{0.3cm}|p{0.3cm}|p{0.3cm}|}
\hline
0 & n & n & n & n & n & n & n\\
\hline
\end{tabular}
& less than 0x78\\
\end{tabular}

Byte 2:

\begin{tabular}{p{0.4\linewidth} p{0.38\linewidth}} 

\begin{tabular}{|p{0.3cm}|p{0.3cm}|p{0.3cm}|p{0.3cm}|p{0.3cm}|p{0.3cm}|p{0.3cm}|p{0.3cm}|}
\hline
0 & 0 & $\times$ & $\times$ & $\times$ & $\times$ & $\times$ & $\times$\\
\hline
\end{tabular}
&  \\
& \\
\end{tabular}

\underline{Notes:} 

None.

\rule{15.1cm}{0.4pt}

\input{Chapter1-LocoDirF0F4P2}
\input{Chapter1-LocoF5F8P1}
\input{Chapter1-LocoF5F11P2}
\input{Chapter1-LocoF12F20F28P2}
\input{Chapter1-LocoF13F19P2}
\newpage
\section{LocoF21F27P2}\index{LocoF21F27P2}\index{Functions}

\underline{Description:}

This \gls{Command} requests the command station to set the locomotive's function F21 to F27 states.

\underline{Protocol:}

2

\underline{Group:}

6-Byte Message

\underline{Opcode:}

OPC\_D4\_GROUP (unofficial mnemonic)

\underline{Type:}

\gls{Command}

\underline{Encoding:} 

Byte 0:

\begin{tabular}{p{0.4\linewidth} p{0.15\linewidth} p{0.38\linewidth}} 

\begin{tabular}{|p{0.3cm}|p{0.3cm}|p{0.3cm}|p{0.3cm}|p{0.3cm}|p{0.3cm}|p{0.3cm}|p{0.3cm}|}
\hline
1 & 1 & 0 & 1 & 0 & 1 & 0 & 0\\
\hline
\end{tabular}
& 0xD4 & Opcode.\\
\end{tabular}

Byte 1:

\begin{tabular}{p{0.4\linewidth} p{0.15\linewidth} p{0.38\linewidth}} 

\begin{tabular}{|p{0.3cm}|p{0.3cm}|p{0.3cm}|p{0.3cm}|p{0.3cm}|p{0.3cm}|p{0.3cm}|p{0.3cm}|}
\hline
0 & 0 & 1 & 0 & 0 & d2 & d1 & d0\\
\hline
\end{tabular}
& $<$SLOTP$>$ & Bits d2 to d0 contain the slot page number in the range 0x0 to 0x7.\\
\end{tabular}

Byte 2:

\begin{tabular}{p{0.4\linewidth} p{0.15\linewidth} p{0.38\linewidth}} 

\begin{tabular}{|p{0.3cm}|p{0.3cm}|p{0.3cm}|p{0.3cm}|p{0.3cm}|p{0.3cm}|p{0.3cm}|p{0.3cm}|}
\hline
0 & n & n & n & n & n & n & n\\
\hline
\end{tabular}
& $<$SLOT\#$>$ & Slot number.\\
\end{tabular}

Byte 3:

\begin{tabular}{p{0.4\linewidth} p{0.15\linewidth} p{0.38\linewidth}} 

\begin{tabular}{|p{0.3cm}|p{0.3cm}|p{0.3cm}|p{0.3cm}|p{0.3cm}|p{0.3cm}|p{0.3cm}|p{0.3cm}|}
\hline
0 & 0 & 0 & 0 & 0 & 1 & 0 & 1\\
\hline
\end{tabular}
& 0x09 & Subcode.\\
\end{tabular}

Byte 4:

\begin{tabular}{p{0.4\linewidth} p{0.15\linewidth} p{0.38\linewidth}} 

\begin{tabular}{|p{0.3cm}|p{0.3cm}|p{0.3cm}|p{0.3cm}|p{0.3cm}|p{0.3cm}|p{0.3cm}|p{0.3cm}|}
\hline
0 & d6 & d5 & d4 & d3 & d2 & d1 & d0\\
\hline
\end{tabular}
&  & Function states.\\
\end{tabular}

\begin{tabular}{p{0.05\linewidth} p{0.95\linewidth}} 
d6 & F27 state: 1 means on and 0 means off.\\
d5 & F26 state: 1 means on and 0 means off.\\
d4 & F25 state: 1 means on and 0 means off.\\
d3 & F24 state: 1 means on and 0 means off.\\
d2 & F23 state: 1 means on and 0 means off.\\
d1 & F22 state: 1 means on and 0 means off.\\
d0 & F21 state: 1 means on and 0 means off.\\
\end{tabular}

Byte 5:

\begin{tabular}{p{0.4\linewidth} p{0.15\linewidth} p{0.38\linewidth}} 

\begin{tabular}{|p{0.3cm}|p{0.3cm}|p{0.3cm}|p{0.3cm}|p{0.3cm}|p{0.3cm}|p{0.3cm}|p{0.3cm}|}
\hline
0 & n & n & n & n & n & n & n\\
\hline
\end{tabular}
& $<$CHK$>$ & Checksum.\\
\end{tabular}

\underline{Response:} 

None.

\underline{Signature:}

Byte 0:

\begin{tabular}{p{0.4\linewidth} p{0.38\linewidth}} 

\begin{tabular}{|p{0.3cm}|p{0.3cm}|p{0.3cm}|p{0.3cm}|p{0.3cm}|p{0.3cm}|p{0.3cm}|p{0.3cm}|}
\hline
1 & 1 & 0 & 1 & 0 & 1 & 0 & 0\\
\hline
\end{tabular}
& 0xD4\\
\end{tabular}

Byte 1:

\begin{tabular}{p{0.4\linewidth} p{0.15\linewidth} p{0.38\linewidth}} 

\begin{tabular}{|p{0.3cm}|p{0.3cm}|p{0.3cm}|p{0.3cm}|p{0.3cm}|p{0.3cm}|p{0.3cm}|p{0.3cm}|}
\hline
0 & 0 & 1 & 0 & 0 & $\times$ & $\times$ & $\times$\\
\hline
\end{tabular}
& \\
\end{tabular}

Byte 3:

\begin{tabular}{p{0.4\linewidth} p{0.15\linewidth} p{0.38\linewidth}} 

\begin{tabular}{|p{0.3cm}|p{0.3cm}|p{0.3cm}|p{0.3cm}|p{0.3cm}|p{0.3cm}|p{0.3cm}|p{0.3cm}|}
\hline
0 & 0 & 0 & 0 & 1 & 0 & 0 & 1\\
\hline
\end{tabular}
& 0x09\\
\end{tabular}

\underline{Notes:} 

This command was identified in the output from the iTrain commercial model railway control application. It has not been made to work outside of the original context. It is included in this manual as the ``missing link" information that enables it to function may be found in the future. Until that time use the D5 Group commands for protocol 2 control.

\rule{15.1cm}{0.4pt}

\newpage
\subsection{LocoSlotDataP1}

\rule{15.1cm}{0.4pt}

\underline{Description:}

This response provides the data for a specific locomotive slot.

\underline{Protocol:}

1

\underline{Group:}

Variable-Byte Message

\underline{Opcode:}

OPC\_SL\_RD\_DATA

\underline{Type:}

Response

\underline{Encoding:} 

Byte 0:

\begin{tabular}{p{0.4\linewidth} p{0.15\linewidth} p{0.38\linewidth}} 

\begin{tabular}{|p{0.3cm}|p{0.3cm}|p{0.3cm}|p{0.3cm}|p{0.3cm}|p{0.3cm}|p{0.3cm}|p{0.3cm}|}
\hline
1 & 1 & 1 & 0 & 0 & 1 & 1 & 1\\
\hline
\end{tabular}
& 0xE7 & Opcode.\\
\end{tabular}

Byte 1:

\begin{tabular}{p{0.4\linewidth} p{0.15\linewidth} p{0.38\linewidth}} 

\begin{tabular}{|p{0.3cm}|p{0.3cm}|p{0.3cm}|p{0.3cm}|p{0.3cm}|p{0.3cm}|p{0.3cm}|p{0.3cm}|}
\hline
0 & 1 & 1 & 1 & 1 & 1 & 1 & 0\\
\hline
\end{tabular}
& 0x0E & Message length (14 bytes).\\
\end{tabular}

Byte 2:

\begin{tabular}{p{0.4\linewidth} p{0.15\linewidth} p{0.375\linewidth}} 

\begin{tabular}{|p{0.3cm}|p{0.3cm}|p{0.3cm}|p{0.3cm}|p{0.3cm}|p{0.3cm}|p{0.3cm}|p{0.3cm}|}
\hline
0 & n & n & n & n & n & n & n\\
\hline
\end{tabular}
& $<$SLOT\#$>$ & Slot number in the range 0x00 to 0x77. Slot 0x00 is the dispatch special slot.\\
\end{tabular}

Byte 3:

\begin{tabular}{p{0.4\linewidth} p{0.15\linewidth} p{0.38\linewidth}} 

\begin{tabular}{|p{0.3cm}|p{0.3cm}|p{0.3cm}|p{0.3cm}|p{0.3cm}|p{0.3cm}|p{0.3cm}|p{0.3cm}|}
\hline
d7 & d6 & d5 & d4 & d3 & d2 & d1 & d0\\
\hline
\end{tabular}
& $<$STAT1$>$ & Slot status 1.\\
\end{tabular}

\begin{tabular}{p{0.05\linewidth} p{0.05\linewidth} p{0.05\linewidth} p{0.70\linewidth}} 
& \underline{d7} & \underline{d6} & \\
& 0 & 0 & Free, no consist linking.\\
& 0 & 1 & Consist sub-member.\\
& 1 & 0 & Consist top-member.\\
& 1 & 1 & Consist Mid-Consist member. \\
\end{tabular}

Note: d7 is set to 0 in the message by the command station and so may not correctly reflect the actual setting in the slot table.\\

\begin{tabular}{p{0.05\linewidth} p{0.05\linewidth} p{0.05\linewidth} p{0.70\linewidth}} 
& \underline{d5} & \underline{d4} & \\
& 0 & 0 & Free slot, no valid data. Not refreshed.\\
& 0 & 1 & Common. Locomotive address in this slot. Refreshed.\\
& 1 & 0 & Idle. Locomotive address in this slot. Not refreshed.\\
& 1 & 1 & In Use. Locomotive address in this slot. Refreshed. \\
\end{tabular}

 \begin{tabular}{p{0.05\linewidth} p{0.05\linewidth} p{0.05\linewidth} p{0.70\linewidth}} 
& & \underline{d3} & \\
& & 0 & No slot consist linked into this slot.\\
& & 1 & Slot consist linked into this slot.\\
\end{tabular}

\begin{tabular}{p{0.05\linewidth} p{0.05\linewidth} p{0.05\linewidth} p{0.75\linewidth}} 
\underline{d2} & \underline{d1} & \underline{d0} & \\
0 & 0 & 0 & 28 step decoder. 3-byte packet regular mode\\
0 & 0 & 1 & 28 step decoder. Generate Motorola trinary packets for this mobile address\\
0 & 1 & 0 & 14 step decoder. \\
0 & 1 & 1 & 128 step decoder. \\
1 & 0 & 0 & 28 step decoder with \gls{Advanced Consisting} enabled\\
1 & 0 & 1 & reserved\\
1 & 1 & 0 & reserved\\
1 & 1 & 1 & 128 step decoder with \gls{Advanced Consisting} enabled\\
\end{tabular}

Byte 4:

\begin{tabular}{p{0.4\linewidth} p{0.15\linewidth} p{0.38\linewidth}} 

\begin{tabular}{|p{0.3cm}|p{0.3cm}|p{0.3cm}|p{0.3cm}|p{0.3cm}|p{0.3cm}|p{0.3cm}|p{0.3cm}|}
\hline
0 & n & n & n & n & n & n & n\\
\hline
\end{tabular}
& $<$ADR$>$ & If $<$ADR2$>$ is 0 then this contains the NMRA short address. If $<$ADR2$>$ is greater than 0 then this contains the low 7 bits of the NMRA long address.\\
\end{tabular}

Byte 5:

\begin{tabular}{p{0.4\linewidth} p{0.15\linewidth} p{0.36\linewidth}} 

\begin{tabular}{|p{0.3cm}|p{0.3cm}|p{0.3cm}|p{0.3cm}|p{0.3cm}|p{0.3cm}|p{0.3cm}|p{0.3cm}|}
\hline
0 & n & n & n & n & n & n & n\\
\hline
\end{tabular}
& $<$SPD$>$ & Speed in the range 0x00 to 0x7F. 0x00 means inertial stop and 0x01 means emergency stop. Other values mean increasing speed.\\
\end{tabular}

Byte 6:

\begin{tabular}{p{0.4\linewidth} p{0.15\linewidth} p{0.36\linewidth}} 

\begin{tabular}{|p{0.3cm}|p{0.3cm}|p{0.3cm}|p{0.3cm}|p{0.3cm}|p{0.3cm}|p{0.3cm}|p{0.3cm}|}
\hline
0 & 0 & d5 & d4 & d3 & d2 & d1 & d0\\
\hline
\end{tabular}
& $<$DIRF$>$ & Locomotive direction and state of functions F0 to F4.\\
\end{tabular}

\begin{tabular}{p{0.05\linewidth} p{0.95\linewidth}} 
d5 & Direction: 1 means forward and 0 means backwards.\\
d4 & F0 state: 1 means on and 0 means off.\\
d3 & F4 state: 1 means on and 0 means off.\\
d2 & F3 state: 1 means on and 0 means off.\\
d1 & F2 state: 1 means on and 0 means off.\\
d0 & F1 state: 1 means on and 0 means off.\\
\end{tabular}

Byte 7:\index{Global System Track Status}

\begin{tabular}{p{0.4\linewidth} p{0.15\linewidth} p{0.38\linewidth}} 

\begin{tabular}{|p{0.3cm}|p{0.3cm}|p{0.3cm}|p{0.3cm}|p{0.3cm}|p{0.3cm}|p{0.3cm}|p{0.3cm}|}
\hline
0 & d6 & 0 & 0 & d3 & d2 & d1 & d0\\
\hline
\end{tabular}
& $<$TRK$>$ & \gls{Global System Track Status}.\\
\end{tabular}

\begin{tabular}{p{0.05\linewidth} p{0.9\linewidth}} 
d6 & 1 means this command station implements protocol 2 commands.\\
d3 & 1 means the programming track is busy.\\
d2 & 1 means this command station implements protocol 1 commands,  0 means the command station is a DT200.\\
d1 & 0 means the track is paused, broadcast an emergency stop.\\
d0 & 1 means the DCC packets are on and global power is up.\\
\end{tabular}

Byte 8:

\begin{tabular}{p{0.4\linewidth} p{0.15\linewidth} p{0.38\linewidth}} 

\begin{tabular}{|p{0.3cm}|p{0.3cm}|p{0.3cm}|p{0.3cm}|p{0.3cm}|p{0.3cm}|p{0.3cm}|p{0.3cm}|}
\hline
0 & 0 & 0 & 0 & d3 & d2 & 0 & d0\\
\hline
\end{tabular}
& $<$SS2$>$ & Slot status 2.\\
\end{tabular}

\begin{tabular}{p{0.05\linewidth} p{0.6\linewidth}} 
d3 & 1 means expansion in ID1/2, 0 means encoded alias.\\
d2 & 1 means expansion ID1/2 is not ID usage.\\
d0 & 1 means this slot has suppressed advanced consist.\\
\end{tabular}

Byte 9:

\begin{tabular}{p{0.4\linewidth} p{0.15\linewidth} p{0.38\linewidth}} 

\begin{tabular}{|p{0.3cm}|p{0.3cm}|p{0.3cm}|p{0.3cm}|p{0.3cm}|p{0.3cm}|p{0.3cm}|p{0.3cm}|}
\hline
0 & n & n & n & n & n & n & n\\
\hline
\end{tabular}
& $<$ADR2$>$ & If $<$ADR2$>$ is greater than 0 then this contains the high 7 bits of the NMRA long address.\\
\end{tabular}

Byte 10:

\begin{tabular}{p{0.4\linewidth} p{0.15\linewidth} p{0.38\linewidth}} 

\begin{tabular}{|p{0.3cm}|p{0.3cm}|p{0.3cm}|p{0.3cm}|p{0.3cm}|p{0.3cm}|p{0.3cm}|p{0.3cm}|}
\hline
0 & 0 & 0 & 0 & d3 & d2 & d1 & d0\\
\hline
\end{tabular}
& $<$SND$>$ & Function F5 to F8 states.\\
\end{tabular}

\begin{tabular}{p{0.05\linewidth} p{0.6\linewidth}} 
d3 & F8 state: 1 means on and 0 means off.\\
d2 & F7 state: 1 means on and 0 means off.\\
d1 & F6 state: 1 means on and 0 means off.\\
d0 & F5 state: 1 means on and 0 means off.\\
\end{tabular}

Byte 11:

\begin{tabular}{p{0.4\linewidth} p{0.15\linewidth} p{0.37\linewidth}} 

\begin{tabular}{|p{0.3cm}|p{0.3cm}|p{0.3cm}|p{0.3cm}|p{0.3cm}|p{0.3cm}|p{0.3cm}|p{0.3cm}|}
\hline
0 & n & n & n & n & n & n & n\\
\hline
\end{tabular}
& $<$ID1$>$ & 7-bit ls ID code written by throttle when STAT2.4 = 1.\\
\end{tabular}

Byte 12:

\begin{tabular}{p{0.4\linewidth} p{0.15\linewidth} p{0.37\linewidth}} 

\begin{tabular}{|p{0.3cm}|p{0.3cm}|p{0.3cm}|p{0.3cm}|p{0.3cm}|p{0.3cm}|p{0.3cm}|p{0.3cm}|}
\hline
0 & n & n & n & n & n & n & n\\
\hline
\end{tabular}
& $<$ID2$>$ & 7-bit ms ID code written by throttle when STAT2.4 = 1.\\
\end{tabular}



Byte 13:

\begin{tabular}{p{0.4\linewidth} p{0.15\linewidth} p{0.37\linewidth}} 

\begin{tabular}{|p{0.3cm}|p{0.3cm}|p{0.3cm}|p{0.3cm}|p{0.3cm}|p{0.3cm}|p{0.3cm}|p{0.3cm}|}
\hline
0 & n & n & n & n & n & n & n\\
\hline
\end{tabular}
& $<$CHK$>$ & Checksum.\\
\end{tabular}

\underline{Response:} 

None.

\underline{Signature:}

Byte 0:

\begin{tabular}{p{0.4\linewidth} p{0.38\linewidth}} 

\begin{tabular}{|p{0.3cm}|p{0.3cm}|p{0.3cm}|p{0.3cm}|p{0.3cm}|p{0.3cm}|p{0.3cm}|p{0.3cm}|}
\hline
1 & 1 & 1 & 0 & 0 & 1 & 1 & 1\\
\hline
\end{tabular}
& 0xE7 \\
\end{tabular}

Byte 1:

\begin{tabular}{p{0.4\linewidth} p{0.38\linewidth}} 

\begin{tabular}{|p{0.3cm}|p{0.3cm}|p{0.3cm}|p{0.3cm}|p{0.3cm}|p{0.3cm}|p{0.3cm}|p{0.3cm}|}
\hline
0 & 1 & 1 & 1 & 1 & 1 & 1 & 0\\
\hline
\end{tabular}
& 0x0E \\
\end{tabular}

Byte 2:

\begin{tabular}{p{0.4\linewidth} p{0.375\linewidth}} 

\begin{tabular}{|p{0.3cm}|p{0.3cm}|p{0.3cm}|p{0.3cm}|p{0.3cm}|p{0.3cm}|p{0.3cm}|p{0.3cm}|}
\hline
0 & n & n & n & n & n & n & n\\
\hline
\end{tabular}
&  less than 0x78.\\
\end{tabular}

Byte 6:

\begin{tabular}{p{0.4\linewidth} p{0.38\linewidth}} 

\begin{tabular}{|p{0.3cm}|p{0.3cm}|p{0.3cm}|p{0.3cm}|p{0.3cm}|p{0.3cm}|p{0.3cm}|p{0.3cm}|}
\hline
0 & 0 & $\times$ & $\times$ & $\times$ & $\times$ & $\times$ & $\times$\\
\hline
\end{tabular}
& \\
\end{tabular}

Byte 7:

\begin{tabular}{p{0.4\linewidth} p{0.38\linewidth}} 

\begin{tabular}{|p{0.3cm}|p{0.3cm}|p{0.3cm}|p{0.3cm}|p{0.3cm}|p{0.3cm}|p{0.3cm}|p{0.3cm}|}
\hline
0 & 0 & 0 & 0 & $\times$ & $\times$ & $\times$ & $\times$\\
\hline
\end{tabular}
& \\
\end{tabular}

Byte 8:

\begin{tabular}{p{0.4\linewidth} p{0.38\linewidth}} 

\begin{tabular}{|p{0.3cm}|p{0.3cm}|p{0.3cm}|p{0.3cm}|p{0.3cm}|p{0.3cm}|p{0.3cm}|p{0.3cm}|}
\hline
0 & 0 & 0 & 0 & $\times$ & $\times$ & 0 & $\times$\\
\hline
\end{tabular}
&  \\
\end{tabular}

Byte 10:

\begin{tabular}{p{0.4\linewidth} p{0.38\linewidth}} 

\begin{tabular}{|p{0.3cm}|p{0.3cm}|p{0.3cm}|p{0.3cm}|p{0.3cm}|p{0.3cm}|p{0.3cm}|p{0.3cm}|}
\hline
0 & 0 & 0 & 0 & $\times$ & $\times$ & $\times$ & $\times$\\
\hline
\end{tabular}
& \\
\end{tabular}

\underline{Notes:} 

None.

\rule{15.1cm}{0.4pt}

\newpage
\subsection{LocoSlotDataP2}

\underline{Description:}

This response provides data for a specific locomotive slot.

\underline{Protocol:}

2

\underline{Group:}

Variable-Byte Message

\underline{Opcode:}

OPC\_SL\_RD\_DATA\_P2

\underline{Type:}

Response

\underline{Encoding:} 

Byte 0:

\begin{tabular}{p{0.4\linewidth} p{0.15\linewidth} p{0.38\linewidth}} 

\begin{tabular}{|p{0.3cm}|p{0.3cm}|p{0.3cm}|p{0.3cm}|p{0.3cm}|p{0.3cm}|p{0.3cm}|p{0.3cm}|}
\hline
1 & 1 & 1 & 0 & 0 & 1 & 1 & 0\\
\hline
\end{tabular}
& 0xE6 & Opcode.\\
\end{tabular}

Byte 1:

\begin{tabular}{p{0.4\linewidth} p{0.15\linewidth} p{0.38\linewidth}} 

\begin{tabular}{|p{0.3cm}|p{0.3cm}|p{0.3cm}|p{0.3cm}|p{0.3cm}|p{0.3cm}|p{0.3cm}|p{0.3cm}|}
\hline
0 & 0 & 0 & 1 & 0 & 1 & 0 & 1\\
\hline
\end{tabular}
& 0x15 & Message length (21 bytes).\\
\end{tabular}

Byte 2:

\begin{tabular}{p{0.4\linewidth} p{0.15\linewidth} p{0.38\linewidth}} 

\begin{tabular}{|p{0.3cm}|p{0.3cm}|p{0.3cm}|p{0.3cm}|p{0.3cm}|p{0.3cm}|p{0.3cm}|p{0.3cm}|}
\hline
0 & 0 & 0 & 0 & 0 & d2 & d1 & d0\\
\hline
\end{tabular}
& $<$SLOTP\#$>$ & Slot page number in the range 0x0 to 0x7.\\
\end{tabular}

Byte 3:

\begin{tabular}{p{0.4\linewidth} p{0.15\linewidth} p{0.38\linewidth}} 

\begin{tabular}{|p{0.3cm}|p{0.3cm}|p{0.3cm}|p{0.3cm}|p{0.3cm}|p{0.3cm}|p{0.3cm}|p{0.3cm}|}
\hline
0 & n & n & n & n & n & n & n\\
\hline
\end{tabular}
& $<$SLOT\#$>$ & Slot number in the range 0x00 to 0x77.\\
\end{tabular}

Byte 4:

\begin{tabular}{p{0.4\linewidth} p{0.15\linewidth} p{0.38\linewidth}} 

\begin{tabular}{|p{0.3cm}|p{0.3cm}|p{0.3cm}|p{0.3cm}|p{0.3cm}|p{0.3cm}|p{0.3cm}|p{0.3cm}|}
\hline
0 & d6 & d5 & d4 & d3 & d2 & d1 & d0\\
\hline
\end{tabular}
&  $<$STAT1$>$ & Slot status 1.\\
\end{tabular}

\begin{tabular}{p{0.05\linewidth} p{0.05\linewidth} p{0.05\linewidth} p{0.70\linewidth}} 
& \underline{d7} & \underline{d6} & \\
& 0 & 0 & Free, no consist linking.\\
& 0 & 1 & Consist sub-member.\\
& 1 & 0 & Consist top-member.\\
& 1 & 1 & Consist Mid-Consist member. \\
\end{tabular}

Note: d7 is set to 0 in the message by the command station and so may not correctly reflect the actual setting in the slot table.\\

\begin{tabular}{p{0.05\linewidth} p{0.05\linewidth} p{0.05\linewidth} p{0.70\linewidth}} 
& \underline{d5} & \underline{d4} & \\
& 0 & 0 & Free slot, no valid data. Not refreshed.\\
& 0 & 1 & Common. Locomotive address in this slot. Refreshed.\\
& 1 & 0 & Idle. Locomotive address in this slot. Not refreshed.\\
& 1 & 1 & In Use. Locomotive address in this slot. Refreshed. \\
\end{tabular}

 \begin{tabular}{p{0.05\linewidth} p{0.05\linewidth} p{0.05\linewidth} p{0.70\linewidth}} 
& & \underline{d3} & \\
& & 0 & No slot consist linked into this slot.\\
& & 1 & Slot consist linked into this slot.\\
\end{tabular}

\begin{tabular}{p{0.05\linewidth} p{0.05\linewidth} p{0.05\linewidth} p{0.75\linewidth}} 
\underline{d2} & \underline{d1} & \underline{d0} & \\
0 & 0 & 0 & 28 step decoder. 3-byte packet regular mode\\
0 & 0 & 1 & 28 step decoder. Generate Motorola trinary packets for this mobile address\\
0 & 1 & 0 & 14 step decoder. \\
0 & 1 & 1 & 128 step decoder. \\
1 & 0 & 0 & 28 step decode with \gls{Advanced Consisting} enabled\\
1 & 0 & 1 & reserved\\
1 & 1 & 0 & reserved\\
1 & 1 & 1 & 128 step decoder with \gls{Advanced Consisting} enabled\\
\end{tabular}

Byte 5:

\begin{tabular}{p{0.4\linewidth} p{0.15\linewidth} p{0.38\linewidth}} 

\begin{tabular}{|p{0.3cm}|p{0.3cm}|p{0.3cm}|p{0.3cm}|p{0.3cm}|p{0.3cm}|p{0.3cm}|p{0.3cm}|}
\hline
0 & n & n & n & n & n & n & n\\
\hline
\end{tabular}
& $<$ADR$>$ & Low address.\\
\end{tabular}

Byte 6:

\begin{tabular}{p{0.4\linewidth} p{0.15\linewidth} p{0.38\linewidth}} 

\begin{tabular}{|p{0.3cm}|p{0.3cm}|p{0.3cm}|p{0.3cm}|p{0.3cm}|p{0.3cm}|p{0.3cm}|p{0.3cm}|}
\hline
0 & n & n & n & n & n & n & n\\
\hline
\end{tabular}
& $<$ADR2$>$ & High address.\\
\end{tabular}

Byte 7:

\begin{tabular}{p{0.4\linewidth} p{0.15\linewidth} p{0.38\linewidth}} 

\begin{tabular}{|p{0.3cm}|p{0.3cm}|p{0.3cm}|p{0.3cm}|p{0.3cm}|p{0.3cm}|p{0.3cm}|p{0.3cm}|}
\hline
0 & d6 & d5 & d4 & d3 & d2 & d1 & d0\\
\hline
\end{tabular}
& $<$TRK$>$ & Global system track status.\\
\end{tabular}

\begin{tabular}{p{0.05\linewidth} p{0.92\linewidth}} 
d6 & 1 means this command station implements protocol 2 messages.\\
d5 & Reserved. Set to 0.\\
d4 & Reserved. Set to 0.\\
d3 & 1 means the programming track is busy.\\
d2 & 1 means this command station implements protocol 1 messages.  0 means the command station is a DT200.\\
d1 & 0 means the track is paused, broadcast an emergency stop.\\
d0 & 1 means the DCC packets are on in the command station and track power is on.\\
\end{tabular}

Byte 8:

\begin{tabular}{p{0.4\linewidth} p{0.15\linewidth} p{0.38\linewidth}} 

\begin{tabular}{|p{0.3cm}|p{0.3cm}|p{0.3cm}|p{0.3cm}|p{0.3cm}|p{0.3cm}|p{0.3cm}|p{0.3cm}|}
\hline
0 & d6 & d5 & d4 & d3 & d2 & d1 & d0\\
\hline
\end{tabular}
& $<$SPD$>$ & Speed in the range 0x00 to 0x7F. 0x00 means inertial stop and 0x01 means emergency stop. Other values mean increasing speed.\\
\end{tabular}

Byte 9:

\begin{tabular}{p{0.4\linewidth} p{0.15\linewidth} p{0.38\linewidth}} 

\begin{tabular}{|p{0.3cm}|p{0.3cm}|p{0.3cm}|p{0.3cm}|p{0.3cm}|p{0.3cm}|p{0.3cm}|p{0.3cm}|}
\hline
0 & d6 & d5 & d4 & d3 & d2 & d1 & d0\\
\hline
\end{tabular}
&  & Functions.\\
\end{tabular}

\begin{tabular}{p{0.05\linewidth} p{0.6\linewidth}} 
d6 & F28 state: 1 means on and 0 means off\\
d5 & F20 state: 1 means on and 0 means off\\
d4 & F12 state: 1 means on and 0 means off\\
d3 & \\
d2 & \\
d1 & \\
d0 & \\
\end{tabular}

Byte 10:

\begin{tabular}{p{0.4\linewidth} p{0.15\linewidth} p{0.38\linewidth}} 

\begin{tabular}{|p{0.3cm}|p{0.3cm}|p{0.3cm}|p{0.3cm}|p{0.3cm}|p{0.3cm}|p{0.3cm}|p{0.3cm}|}
\hline
0 & d6 & d5 & d4 & d3 & d2 & d1 & d0\\
\hline
\end{tabular}
&  & Direction and Functions.\\
\end{tabular}

\begin{tabular}{p{0.05\linewidth} p{0.6\linewidth}} 
d6 & \\
d5 & Direction: 1 means forwards and 0 means backwards\\
d4 & F0 state: 1 means on and 0 means off\\
d3 & F4 state: 1 means on and 0 means off\\
d2 & F3 state: 1 means on and 0 means off\\
d1 & F2 state: 1 means on and 0 means off\\
d0 & F1 state: 1 means on and 0 means off\\
\end{tabular}

Byte 11:

\begin{tabular}{p{0.4\linewidth} p{0.15\linewidth} p{0.38\linewidth}} 

\begin{tabular}{|p{0.3cm}|p{0.3cm}|p{0.3cm}|p{0.3cm}|p{0.3cm}|p{0.3cm}|p{0.3cm}|p{0.3cm}|}
\hline
0 & d6 & d5 & d4 & d3 & d2 & d1 & d0\\
\hline
\end{tabular}
&  & Functions.\\
\end{tabular}

\begin{tabular}{p{0.05\linewidth} p{0.6\linewidth}} 
d6 & F11 state: 1 means on and 0 means off\\
d5 & F10 state: 1 means on and 0 means off\\
d4 & F9 state: 1 means on and 0 means off\\
d3 & F8 state: 1 means on and 0 means off\\
d2 & F7 state: 1 means on and 0 means off\\
d1 & F6 state: 1 means on and 0 means off\\
d0 & F5 state: 1 means on and 0 means off\\
\end{tabular}

Byte 12:

\begin{tabular}{p{0.4\linewidth} p{0.15\linewidth} p{0.38\linewidth}} 

\begin{tabular}{|p{0.3cm}|p{0.3cm}|p{0.3cm}|p{0.3cm}|p{0.3cm}|p{0.3cm}|p{0.3cm}|p{0.3cm}|}
\hline
0 & d6 & d5 & d4 & d3 & d2 & d1 & d0\\
\hline
\end{tabular}
&  & Functions.\\
\end{tabular}

\begin{tabular}{p{0.05\linewidth} p{0.6\linewidth}} 
d6 & F19 state: 1 means on and 0 means off\\
d5 & F18 state: 1 means on and 0 means off\\
d4 & F17 state: 1 means on and 0 means off\\
d3 & F16 state: 1 means on and 0 means off\\
d2 & F15 state: 1 means on and 0 means off\\
d1 & F14 state: 1 means on and 0 means off\\
d0 & F13 state: 1 means on and 0 means off\\
\end{tabular}

Byte 13:

\begin{tabular}{p{0.4\linewidth} p{0.15\linewidth} p{0.38\linewidth}} 

\begin{tabular}{|p{0.3cm}|p{0.3cm}|p{0.3cm}|p{0.3cm}|p{0.3cm}|p{0.3cm}|p{0.3cm}|p{0.3cm}|}
\hline
0 & d6 & d5 & d4 & d3 & d2 & d1 & d0\\
\hline
\end{tabular}
&  & Functions.\\
\end{tabular}

\begin{tabular}{p{0.05\linewidth} p{0.6\linewidth}} 
d6 & F27 state: 1 means on and 0 means off\\
d5 & F26 state: 1 means on and 0 means off\\
d4 & F25 state: 1 means on and 0 means off\\
d3 & F24 state: 1 means on and 0 means off\\
d2 & F23 state: 1 means on and 0 means off\\
d1 & F22 state: 1 means on and 0 means off\\
d0 & F21 state: 1 means on and 0 means off\\
\end{tabular}

Byte 14:

\begin{tabular}{p{0.4\linewidth} p{0.15\linewidth} p{0.38\linewidth}} 

\begin{tabular}{|p{0.3cm}|p{0.3cm}|p{0.3cm}|p{0.3cm}|p{0.3cm}|p{0.3cm}|p{0.3cm}|p{0.3cm}|}
\hline
0 & d6 & d5 & d4 & d3 & d2 & d1 & d0\\
\hline
\end{tabular}
&  & Unknown.\\
\end{tabular}

\begin{tabular}{p{0.05\linewidth} p{0.6\linewidth}} 
d6 & \\
d5 & \\
d4 & \\
d3 & \\
d2 & \\
d1 & \\
d0 & \\
\end{tabular}

Byte 15:

\begin{tabular}{p{0.4\linewidth} p{0.15\linewidth} p{0.38\linewidth}} 

\begin{tabular}{|p{0.3cm}|p{0.3cm}|p{0.3cm}|p{0.3cm}|p{0.3cm}|p{0.3cm}|p{0.3cm}|p{0.3cm}|}
\hline
0 & d6 & d5 & d4 & d3 & d2 & d1 & d0\\
\hline
\end{tabular}
&  & Unknown.\\
\end{tabular}

\begin{tabular}{p{0.05\linewidth} p{0.6\linewidth}} 
d6 & \\
d5 & \\
d4 & \\
d3 & \\
d2 & \\
d1 & \\
d0 & \\
\end{tabular}

Byte 16:

\begin{tabular}{p{0.4\linewidth} p{0.15\linewidth} p{0.38\linewidth}} 

\begin{tabular}{|p{0.3cm}|p{0.3cm}|p{0.3cm}|p{0.3cm}|p{0.3cm}|p{0.3cm}|p{0.3cm}|p{0.3cm}|}
\hline
0 & d6 & d5 & d4 & d3 & d2 & d1 & d0\\
\hline
\end{tabular}
&  & Consist slot page number.\\
\end{tabular}

\begin{tabular}{p{0.05\linewidth} p{0.6\linewidth}} 
d6 & \\
d5 & \\
d4 & \\
d3 & \\
d2 & Consist slot page b2\\
d1 & Consist slot page b1\\
d0 & Consist slot page b0\\
\end{tabular}

Byte 17:

\begin{tabular}{p{0.4\linewidth} p{0.15\linewidth} p{0.38\linewidth}} 

\begin{tabular}{|p{0.3cm}|p{0.3cm}|p{0.3cm}|p{0.3cm}|p{0.3cm}|p{0.3cm}|p{0.3cm}|p{0.3cm}|}
\hline
0 & n & n & n & n & n & n & n\\
\hline
\end{tabular}
&  & Consist slot number in the range 0x00 to 0x77.\\
\end{tabular}

Byte 18:

\begin{tabular}{p{0.4\linewidth} p{0.15\linewidth} p{0.38\linewidth}} 

\begin{tabular}{|p{0.3cm}|p{0.3cm}|p{0.3cm}|p{0.3cm}|p{0.3cm}|p{0.3cm}|p{0.3cm}|p{0.3cm}|}
\hline
0 & d6 & d5 & d4 & d3 & d2 & d1 & d0\\
\hline
\end{tabular}
&  & Throttle id low bits.\\
\end{tabular}

Byte 19:

\begin{tabular}{p{0.4\linewidth} p{0.15\linewidth} p{0.38\linewidth}} 

\begin{tabular}{|p{0.3cm}|p{0.3cm}|p{0.3cm}|p{0.3cm}|p{0.3cm}|p{0.3cm}|p{0.3cm}|p{0.3cm}|}
\hline
0 & d6 & d5 & d4 & d3 & d2 & d1 & d0\\
\hline
\end{tabular}
&  & Throttle id high bits.\\
\end{tabular}


Byte 20:

\begin{tabular}{p{0.4\linewidth} p{0.15\linewidth} p{0.38\linewidth}} 

\begin{tabular}{|p{0.3cm}|p{0.3cm}|p{0.3cm}|p{0.3cm}|p{0.3cm}|p{0.3cm}|p{0.3cm}|p{0.3cm}|}
\hline
0 & n & n & n & n & n & n & n\\
\hline
\end{tabular}
& $<$CHK$>$ & Checksum.\\
\end{tabular}

\underline{Response:} 

None.

\underline{Signature:}

Byte 0:

\begin{tabular}{p{0.4\linewidth} p{0.38\linewidth}} 

\begin{tabular}{|p{0.3cm}|p{0.3cm}|p{0.3cm}|p{0.3cm}|p{0.3cm}|p{0.3cm}|p{0.3cm}|p{0.3cm}|}
\hline
1 & 1 & 1 & 0 & 0 & 1 & 1 & 0\\
\hline
\end{tabular}
& 0xE6 \\
\end{tabular}

Byte 1:

\begin{tabular}{p{0.4\linewidth} p{0.38\linewidth}} 

\begin{tabular}{|p{0.3cm}|p{0.3cm}|p{0.3cm}|p{0.3cm}|p{0.3cm}|p{0.3cm}|p{0.3cm}|p{0.3cm}|}
\hline
0 & 0 & 0 & 1 & 0 & 1 & 0 & 1\\
\hline
\end{tabular}
& 0x15\\
\end{tabular}

Byte 2:

\begin{tabular}{p{0.4\linewidth} p{0.15\linewidth} p{0.38\linewidth}} 

\begin{tabular}{|p{0.3cm}|p{0.3cm}|p{0.3cm}|p{0.3cm}|p{0.3cm}|p{0.3cm}|p{0.3cm}|p{0.3cm}|}
\hline
0 & 0 & 0 & 0 & 0 & $\times$ & $\times$ & $\times$\\
\hline
\end{tabular}
& \\
\end{tabular}

Byte 7:

\begin{tabular}{p{0.4\linewidth} p{0.38\linewidth}} 

\begin{tabular}{|p{0.3cm}|p{0.3cm}|p{0.3cm}|p{0.3cm}|p{0.3cm}|p{0.3cm}|p{0.3cm}|p{0.3cm}|}
\hline
0 & $\times$ & 0 & 0 & $\times$ & $\times$ & $\times$ & $\times$\\
\hline
\end{tabular}
& \\
\end{tabular}

\underline{Notes:} 

None.

\rule{15.1cm}{0.4pt}

\newpage
\subsection{LocoSpdP1}\index{LocoSpdP1}\index{Speed}

\underline{Description:}

This command sets the locomotive's speed in the range 0 to 127. 0 means inertial stop and 1 means emergency stop. Other values mean increasing speed.

\underline{Protocol:}

1

\underline{Group:}

4-Byte Message

\underline{Opcode:}

OPC\_LOCO\_SPD

\underline{Type:}

Command

\underline{Encoding:} 

Byte 0:

\begin{tabular}{p{0.4\linewidth} p{0.15\linewidth} p{0.38\linewidth}} 

\begin{tabular}{|p{0.3cm}|p{0.3cm}|p{0.3cm}|p{0.3cm}|p{0.3cm}|p{0.3cm}|p{0.3cm}|p{0.3cm}|}
\hline
1 & 0 & 1 & 0 & 0 & 0 & 0 & 0\\
\hline
\end{tabular}
& 0xA0 & Opcode.\\
\end{tabular}

Byte 1:

\begin{tabular}{p{0.4\linewidth} p{0.15\linewidth} p{0.38\linewidth}} 

\begin{tabular}{|p{0.3cm}|p{0.3cm}|p{0.3cm}|p{0.3cm}|p{0.3cm}|p{0.3cm}|p{0.3cm}|p{0.3cm}|}
\hline
0 & n & n & n & n & n & n & n\\
\hline
\end{tabular}
& $<$SLOT\#$>$ & Slot number in the range 0x00 to 0x77.\\
\end{tabular}

Byte 2:

\begin{tabular}{p{0.4\linewidth} p{0.15\linewidth} p{0.38\linewidth}} 

\begin{tabular}{|p{0.3cm}|p{0.3cm}|p{0.3cm}|p{0.3cm}|p{0.3cm}|p{0.3cm}|p{0.3cm}|p{0.3cm}|}
\hline
0 & n & n & n & n & n & n & n\\
\hline
\end{tabular}
& $<$SPD$>$ & Locomotive speed in the range 0 to 127.\\
\end{tabular}

Byte 3:

\begin{tabular}{p{0.4\linewidth} p{0.15\linewidth} p{0.38\linewidth}} 

\begin{tabular}{|p{0.3cm}|p{0.3cm}|p{0.3cm}|p{0.3cm}|p{0.3cm}|p{0.3cm}|p{0.3cm}|p{0.3cm}|}
\hline
0 & n & n & n & n & n & n & n\\
\hline
\end{tabular}
& $<$CHK$>$ & Checksum.\\
\end{tabular}

\underline{Response:} 

None.

\underline{Signature:}

Byte 0:

\begin{tabular}{p{0.4\linewidth} p{0.38\linewidth}} 

\begin{tabular}{|p{0.3cm}|p{0.3cm}|p{0.3cm}|p{0.3cm}|p{0.3cm}|p{0.3cm}|p{0.3cm}|p{0.3cm}|}
\hline
1 & 0 & 1 & 0 & 0 & 0 & 0 & 0\\
\hline
\end{tabular}
& 0xA0 \\
\end{tabular}

Byte 1:

\begin{tabular}{p{0.4\linewidth} p{0.38\linewidth}} 

\begin{tabular}{|p{0.3cm}|p{0.3cm}|p{0.3cm}|p{0.3cm}|p{0.3cm}|p{0.3cm}|p{0.3cm}|p{0.3cm}|}
\hline
0 & n & n & n & n & n & n & n\\
\hline
\end{tabular}
&less than 0x78\\
\end{tabular}

\underline{Notes:} 

None.

\rule{15.1cm}{0.4pt}

\newpage
\section{LocoSpdP2}\index{LocoSpdP2}\index{Speed}

\underline{Description:}

This \gls{Command} sets the locomotive's speed in the range 0 to 127. 0 means inertial stop and 1 means emergency stop. Other values mean increasing speed.

\underline{Protocol:}

2

\underline{Group:}

6-Byte Message

\underline{Opcode:}

OPC\_D4\_GROUP (unofficial mnemonic)

\underline{Type:}

\gls{Command}

\underline{Encoding:} 

Byte 0:

\begin{tabular}{p{0.4\linewidth} p{0.15\linewidth} p{0.38\linewidth}} 

\begin{tabular}{|p{0.3cm}|p{0.3cm}|p{0.3cm}|p{0.3cm}|p{0.3cm}|p{0.3cm}|p{0.3cm}|p{0.3cm}|}
\hline
1 & 1 & 0 & 1 & 0 & 1 & 0 & 0\\
\hline
\end{tabular}
& 0xD4 & Opcode.\\
\end{tabular}

Byte 1:

\begin{tabular}{p{0.4\linewidth} p{0.15\linewidth} p{0.38\linewidth}} 

\begin{tabular}{|p{0.3cm}|p{0.3cm}|p{0.3cm}|p{0.3cm}|p{0.3cm}|p{0.3cm}|p{0.3cm}|p{0.3cm}|}
\hline
0 & 0 & 1 & 0 & 0 & d2 & d1 & d0\\
\hline
\end{tabular}
& $<$SLOTP$>$ & Bits d2 to d0 contain the slot page number in the range 0x0 to 0x7.\\
\end{tabular}

Byte 2:

\begin{tabular}{p{0.4\linewidth} p{0.15\linewidth} p{0.38\linewidth}} 

\begin{tabular}{|p{0.3cm}|p{0.3cm}|p{0.3cm}|p{0.3cm}|p{0.3cm}|p{0.3cm}|p{0.3cm}|p{0.3cm}|}
\hline
0 & n & n & n & n & n & n & n\\
\hline
\end{tabular}
& $<$SLOT\#$>$ & Slot number in the range 0x00 to 0x77.\\
\end{tabular}

Byte 3:

\begin{tabular}{p{0.4\linewidth} p{0.15\linewidth} p{0.38\linewidth}} 

\begin{tabular}{|p{0.3cm}|p{0.3cm}|p{0.3cm}|p{0.3cm}|p{0.3cm}|p{0.3cm}|p{0.3cm}|p{0.3cm}|}
\hline
0 & 0 & 0 & 0 & 0 & 1 & 0 & 0\\
\hline
\end{tabular}
& 0x04 & Subcode.\\
\end{tabular}

Byte 4:

\begin{tabular}{p{0.4\linewidth} p{0.15\linewidth} p{0.38\linewidth}} 

\begin{tabular}{|p{0.3cm}|p{0.3cm}|p{0.3cm}|p{0.3cm}|p{0.3cm}|p{0.3cm}|p{0.3cm}|p{0.3cm}|}
\hline
0 & n & n & n & n & n & n & n\\
\hline
\end{tabular}
& $<$SPD$>$ & Locomotive speed in the range 0x00 to 0x7F. \\
\end{tabular}

Byte 5:

\begin{tabular}{p{0.4\linewidth} p{0.15\linewidth} p{0.38\linewidth}} 

\begin{tabular}{|p{0.3cm}|p{0.3cm}|p{0.3cm}|p{0.3cm}|p{0.3cm}|p{0.3cm}|p{0.3cm}|p{0.3cm}|}
\hline
0 & n & n & n & n & n & n & n\\
\hline
\end{tabular}
& $<$CHK$>$ & Checksum.\\
\end{tabular}

\underline{Response:} 

None.

\underline{Signature:}

Byte 0:

\begin{tabular}{p{0.4\linewidth} p{0.38\linewidth}} 

\begin{tabular}{|p{0.3cm}|p{0.3cm}|p{0.3cm}|p{0.3cm}|p{0.3cm}|p{0.3cm}|p{0.3cm}|p{0.3cm}|}
\hline
1 & 1 & 0 & 1 & 0 & 1 & 0 & 0\\
\hline
\end{tabular}
& 0xD4 \\
\end{tabular}

Byte 1:

\begin{tabular}{p{0.4\linewidth} p{0.38\linewidth}} 

\begin{tabular}{|p{0.3cm}|p{0.3cm}|p{0.3cm}|p{0.3cm}|p{0.3cm}|p{0.3cm}|p{0.3cm}|p{0.3cm}|}
\hline
0 & 0 & 1 & 0 & 0 & $\times$ & $\times$ & $\times$\\
\hline
\end{tabular}
&  \\
\end{tabular}

Byte 3:

\begin{tabular}{p{0.4\linewidth} p{0.38\linewidth}} 

\begin{tabular}{|p{0.3cm}|p{0.3cm}|p{0.3cm}|p{0.3cm}|p{0.3cm}|p{0.3cm}|p{0.3cm}|p{0.3cm}|}
\hline
0 & 0 & 0 & 0 & 0 & 1 & 0 & 0\\
\hline
\end{tabular}
& 0x04\\
\end{tabular}

\underline{Notes:} 

This \gls{Command} was identified in the output from the iTrain commercial model railway control application. It has not been made to work outside of the original context. It is included in this manual as the ``missing link" information that enables it to function may be found in the future. Until that time use the D5 Group commands for protocol 2 control.

\rule{15.1cm}{0.4pt}

\newpage
\subsection{MoveSlotsP1}

\underline{Description:}

Move slots.

\begin{tabular}{p{0.1\linewidth} p{0.1\linewidth} p{0.70\linewidth}} 
\underline{SRC} & \underline{DEST} & \underline{Action}\\
0 & $\times$ & Dispatch get. Return \textbf{LocoSlotDataP1} of dispatch slot.\\
SRC & 0 & Dispatch put. Mark slot as dispatch.\\
SRC & SRC & Null move. SRC is set to in use.\\
SRC & DEST & Move slot data from SRC to DEST if not in use. Clear SRC.\\
\end{tabular}

\underline{Protocol:}

1

\underline{Group:}

4-Byte Message

\underline{Opcode:}

OPC\_MOVE\_SLOTS

\underline{Type:}

Command

\underline{Encoding:} 

Byte 0:

\begin{tabular}{p{0.4\linewidth} p{0.15\linewidth} p{0.38\linewidth}} 

\begin{tabular}{|p{0.3cm}|p{0.3cm}|p{0.3cm}|p{0.3cm}|p{0.3cm}|p{0.3cm}|p{0.3cm}|p{0.3cm}|}
\hline
1 & 0 & 1 & 1 & 1 & 0 & 1 & 0\\
\hline
\end{tabular}
& 0xBA & Opcode.\\
\end{tabular}

Byte 1:

\begin{tabular}{p{0.4\linewidth} p{0.15\linewidth} p{0.38\linewidth}} 

\begin{tabular}{|p{0.3cm}|p{0.3cm}|p{0.3cm}|p{0.3cm}|p{0.3cm}|p{0.3cm}|p{0.3cm}|p{0.3cm}|}
\hline
0 & n & n & n & n & n & n & n\\
\hline
\end{tabular}
& $<$SRC$>$ & Source slot number in the range 0x00 to 0x77.\\
\end{tabular}

Byte 2:

\begin{tabular}{p{0.4\linewidth} p{0.15\linewidth} p{0.38\linewidth}} 

\begin{tabular}{|p{0.3cm}|p{0.3cm}|p{0.3cm}|p{0.3cm}|p{0.3cm}|p{0.3cm}|p{0.3cm}|p{0.3cm}|}
\hline
0 & n & n & n & n & n & n & n\\
\hline
\end{tabular}
& $<$DEST$>$ & Destination slot number in the range 0x00 to 0x77.\\
\end{tabular}

Byte 3:

\begin{tabular}{p{0.4\linewidth} p{0.15\linewidth} p{0.38\linewidth}} 

\begin{tabular}{|p{0.3cm}|p{0.3cm}|p{0.3cm}|p{0.3cm}|p{0.3cm}|p{0.3cm}|p{0.3cm}|p{0.3cm}|}
\hline
0 & n & n & n & n & n & n & n\\
\hline
\end{tabular}
& $<$CHK$>$ & Checksum.

\end{tabular}

\underline{Response:} 

\textbf{LocoSlotDataP1} 

or 

\textbf{Ack}.

\begin{tabular}{p{0.10\linewidth} p{0.10\linewidth} p{0.40\linewidth}}
\underline{$<$LOPC$>$} & \underline{$<$ACK1$>$} & \underline{Meaning}\\
0x3A & 0x00 & Illegal move.\\
\end{tabular}

\underline{Signature:}

Byte 0:

\begin{tabular}{p{0.4\linewidth}  p{0.38\linewidth}} 

\begin{tabular}{|p{0.3cm}|p{0.3cm}|p{0.3cm}|p{0.3cm}|p{0.3cm}|p{0.3cm}|p{0.3cm}|p{0.3cm}|}
\hline
1 & 0 & 1 & 1 & 1 & 0 & 1 & 0\\
\hline
\end{tabular}
& 0xBA \\
\end{tabular}

Byte 1:

\begin{tabular}{p{0.4\linewidth} p{0.38\linewidth}} 

\begin{tabular}{|p{0.3cm}|p{0.3cm}|p{0.3cm}|p{0.3cm}|p{0.3cm}|p{0.3cm}|p{0.3cm}|p{0.3cm}|}
\hline
0 & n & n & n & n & n & n & n\\
\hline
\end{tabular}
& less than 0x78\\
\end{tabular}

Byte 2:

\begin{tabular}{p{0.4\linewidth} p{0.38\linewidth}} 

\begin{tabular}{|p{0.3cm}|p{0.3cm}|p{0.3cm}|p{0.3cm}|p{0.3cm}|p{0.3cm}|p{0.3cm}|p{0.3cm}|}
\hline
0 & n & n & n & n & n & n & n\\
\hline
\end{tabular}
& less than 0x78\\
\end{tabular}

\underline{Notes:} 

None.

\rule{15.1cm}{0.4pt}

\newpage
\subsection{MoveSlotsP2}\index{MoveSlotsP2}\index{Slot}\index{Null Move}

\underline{Description:}

Move slots.

\begin{tabular}{p{0.1\linewidth} p{0.1\linewidth} p{0.70\linewidth}} 
\underline{SRC} & \underline{DEST} & \underline{Action}\\
0 & $\times$ & Dispatch get. Return \textbf{LocoSlotDataP2} of dispatch slot.\\
SRC & 0 & Dispatch put. Mark slot as dispatch.\\
SRC & SRC & Null move. SRC is set to in use.\\
SRC & DEST & Move slot data from SRC to DEST if not in use. Clear SRC.\\
\end{tabular}

\underline{Protocol:}

2

\underline{Group:}

6-Byte Message

\underline{Opcode:}

OPC\_D4\_GROUP (unofficial mnemonic)

\underline{Type:}

Command

\underline{Encoding:} 

Byte 0:

\begin{tabular}{p{0.4\linewidth} p{0.15\linewidth} p{0.38\linewidth}} 

\begin{tabular}{|p{0.3cm}|p{0.3cm}|p{0.3cm}|p{0.3cm}|p{0.3cm}|p{0.3cm}|p{0.3cm}|p{0.3cm}|}
\hline
1 & 1 & 0 & 1 & 0 & 1 & 0 & 0\\
\hline
\end{tabular}
& 0xD4 & Opcode.\\
\end{tabular}

Byte 1:

\begin{tabular}{p{0.4\linewidth} p{0.15\linewidth} p{0.38\linewidth}} 

\begin{tabular}{|p{0.3cm}|p{0.3cm}|p{0.3cm}|p{0.3cm}|p{0.3cm}|p{0.3cm}|p{0.3cm}|p{0.3cm}|}
\hline
0 & 0 & 1 & 1 & 1 & d2 & d1 & d0\\
\hline
\end{tabular}
& $<$SRCP$>$ & Bits d2 to d0 contain the source slot page number in the range 0x0 to 0x7. \\
\end{tabular}

Byte 2:

\begin{tabular}{p{0.4\linewidth} p{0.15\linewidth} p{0.38\linewidth}} 

\begin{tabular}{|p{0.3cm}|p{0.3cm}|p{0.3cm}|p{0.3cm}|p{0.3cm}|p{0.3cm}|p{0.3cm}|p{0.3cm}|}
\hline
0 & n & n & n & n & n & n & n\\
\hline
\end{tabular}
& $<$SRC$>$ & Source slot number.\\
\end{tabular}

Byte 3:

\begin{tabular}{p{0.4\linewidth} p{0.15\linewidth} p{0.38\linewidth}} 

\begin{tabular}{|p{0.3cm}|p{0.3cm}|p{0.3cm}|p{0.3cm}|p{0.3cm}|p{0.3cm}|p{0.3cm}|p{0.3cm}|}
\hline
0 & 0 & 0 & 0 & 0 & d2 & d1 & d0\\
\hline
\end{tabular}
& $<$DESTP$>$ & Bits d2 to d0 contain the destination slot page number in the range 0x0 to 0x7. \\
\end{tabular}

Byte 4:

\begin{tabular}{p{0.4\linewidth} p{0.15\linewidth} p{0.38\linewidth}} 

\begin{tabular}{|p{0.3cm}|p{0.3cm}|p{0.3cm}|p{0.3cm}|p{0.3cm}|p{0.3cm}|p{0.3cm}|p{0.3cm}|}
\hline
0 & n & n & n & n & n & n & n\\
\hline
\end{tabular}
& $<$DEST$>$ & Destination slot number.\\
\end{tabular}

Byte 5:

\begin{tabular}{p{0.4\linewidth} p{0.15\linewidth} p{0.38\linewidth}} 

\begin{tabular}{|p{0.3cm}|p{0.3cm}|p{0.3cm}|p{0.3cm}|p{0.3cm}|p{0.3cm}|p{0.3cm}|p{0.3cm}|}
\hline
0 & n & n & n & n & n & n & n\\
\hline
\end{tabular}
& $<$CHK$>$ & Checksum.
\end{tabular}

\underline{Response:} 

\textbf{LocoSlotDataP2} if successful, otherwise \textbf{IllegalMoveP2}.

\underline{Signature:}

Byte 0:

\begin{tabular}{p{0.4\linewidth} p{0.38\linewidth}} 

\begin{tabular}{|p{0.3cm}|p{0.3cm}|p{0.3cm}|p{0.3cm}|p{0.3cm}|p{0.3cm}|p{0.3cm}|p{0.3cm}|}
\hline
1 & 1 & 0 & 1 & 0 & 1 & 0 & 0\\
\hline
\end{tabular}
& 0xD4\\
\end{tabular}

Byte 1:

\begin{tabular}{p{0.4\linewidth} p{0.38\linewidth}} 

\begin{tabular}{|p{0.3cm}|p{0.3cm}|p{0.3cm}|p{0.3cm}|p{0.3cm}|p{0.3cm}|p{0.3cm}|p{0.3cm}|}
\hline
0 & 0 & 1 & 1 & 1 & $\times$ & $\times$ & $\times$\\
\hline
\end{tabular}
& \\
\end{tabular}

Byte 3:

\begin{tabular}{p{0.4\linewidth} p{0.38\linewidth}} 

\begin{tabular}{|p{0.3cm}|p{0.3cm}|p{0.3cm}|p{0.3cm}|p{0.3cm}|p{0.3cm}|p{0.3cm}|p{0.3cm}|}
\hline
0 & 0 & 0 & 0 & 0 & $\times$ & $\times$ & $\times$\\
\hline
\end{tabular}
&  \\
\end{tabular}

\underline{Notes:} 

None.

\rule{15.1cm}{0.4pt}

\newpage
\section{PeerXfer16}\index{PeerXfer16}

\underline{Description:}

This \gls{Command} sends the 8 bytes of data from one device to another peer to peer. This \gls{Command} takes many forms and so what is presented here is a generic description. The specific forms are included elsewhere as detailed messages in their own right.

\begin{tabular}{p{0.2\linewidth} p{0.1\linewidth} p{0.1\linewidth} p{0.50\linewidth}} 
\underline{SRC} & \underline{DSTL} & \underline{DSTH} & Comments\\
0x00 & & & Source is command station.\\
Don't Care & 0x00 & 0x00 & Broadcast Message.\\
0x70 to 0x7E & & & Reserved.\\
0x7F & 0x00 & 0x00 & Broadcast throttle message transfer.\\
0x7F & ID1 & ID2 & Throttle message transfer. ID1 and ID2 encode ID.\\
\end{tabular}

\underline{Protocol:}

1

\underline{Group:}

Variable-Byte Message

\underline{Opcode:}

OPC\_PEER\_XFER

\underline{Type:}

\gls{Command}

\underline{Encoding:} 

Byte 0:

\begin{tabular}{p{0.4\linewidth} p{0.15\linewidth} p{0.38\linewidth}} 

\begin{tabular}{|p{0.3cm}|p{0.3cm}|p{0.3cm}|p{0.3cm}|p{0.3cm}|p{0.3cm}|p{0.3cm}|p{0.3cm}|}
\hline
1 & 1 & 1 & 0 & 0 & 1 & 0 & 1\\
\hline
\end{tabular}
& 0xE5 & Opcode.\\
\end{tabular}

Byte 1:

\begin{tabular}{p{0.4\linewidth} p{0.15\linewidth} p{0.38\linewidth}} 

\begin{tabular}{|p{0.3cm}|p{0.3cm}|p{0.3cm}|p{0.3cm}|p{0.3cm}|p{0.3cm}|p{0.3cm}|p{0.3cm}|}
\hline
0 & 0 & 0 & 1 & 0 & 0 & 0 & 0\\
\hline
\end{tabular}
& 0x10 & Message length (16 bytes).\\
\end{tabular}

Byte 2:

\begin{tabular}{p{0.4\linewidth} p{0.15\linewidth} p{0.38\linewidth}} 

\begin{tabular}{|p{0.3cm}|p{0.3cm}|p{0.3cm}|p{0.3cm}|p{0.3cm}|p{0.3cm}|p{0.3cm}|p{0.3cm}|}
\hline
0 & n & n & n & n & n & n & n\\
\hline
\end{tabular}
& $<$SRC$>$ & Source id in the range 0x00 to 0x7F.\\
\end{tabular}

Byte 3:

\begin{tabular}{p{0.4\linewidth} p{0.15\linewidth} p{0.38\linewidth}} 

\begin{tabular}{|p{0.3cm}|p{0.3cm}|p{0.3cm}|p{0.3cm}|p{0.3cm}|p{0.3cm}|p{0.3cm}|p{0.3cm}|}
\hline
0 & n & n & n & n & n & n & n\\
\hline
\end{tabular}
& $<$DSTL$>$ & Destination id low in the range 0x00 to 0x7F.\\
\end{tabular}

Byte 4:

\begin{tabular}{p{0.4\linewidth} p{0.15\linewidth} p{0.38\linewidth}} 

\begin{tabular}{|p{0.3cm}|p{0.3cm}|p{0.3cm}|p{0.3cm}|p{0.3cm}|p{0.3cm}|p{0.3cm}|p{0.3cm}|}
\hline
0 & n & n & n & n & n & n & n\\
\hline
\end{tabular}
& $<$DSTH$>$ & Destination id high in the range 0x00 to 0x7F.\\
\end{tabular}

Byte 5:

\begin{tabular}{p{0.4\linewidth} p{0.15\linewidth} p{0.38\linewidth}} 

\begin{tabular}{|p{0.3cm}|p{0.3cm}|p{0.3cm}|p{0.3cm}|p{0.3cm}|p{0.3cm}|p{0.3cm}|p{0.3cm}|}
\hline
0 & d6 & d5 & d4 & d3 & d2 & d1 & d0\\
\hline
\end{tabular}
& $<$PXCT1$>$ & Address type code and high bits of D1 to D4.\\
\end{tabular}

\begin{tabular}{p{0.05\linewidth} p{0.6\linewidth}} 
d6 & XC2. Address type code.\\
d5 & XC1. Address type code.\\
d4 & XC0. Address type code.\\
d3 & D4.7. High bit\\
d2 & D3.7. High bit\\
d1 & D2.7. High bit\\
d0 & D1.7. High bit\\
\end{tabular}

\begin{tabular}{p{0.1\linewidth} p{0.1\linewidth} p{0.1\linewidth} p{0.4\linewidth}} 
\underline{XC2} & \underline{XC1} & \underline{XC0} & \underline{Meaning}\\
0 & 0 & 0 & 7 bit peer to peer addresses.\\
0 & 0 & 1 & reserved.\\
0 & 1 & 0 & reserved.\\
0 & 1 & 1 & reserved.\\
1 & 0 & 0 & IPL download.\\
1 & 0 & 1 & reserved.\\
1 & 1 & 0 & reserved.\\
1 & 1 & 1 & reserved.\\
\end{tabular}

Byte 6:

\begin{tabular}{p{0.4\linewidth} p{0.15\linewidth} p{0.38\linewidth}} 

\begin{tabular}{|p{0.3cm}|p{0.3cm}|p{0.3cm}|p{0.3cm}|p{0.3cm}|p{0.3cm}|p{0.3cm}|p{0.3cm}|}
\hline
0 & n & n & n & n & n & n & n\\
\hline
\end{tabular}
& $<$D1$>$ & Data item 1. Low 7 bits.\\
\end{tabular}

Byte 7:

\begin{tabular}{p{0.4\linewidth} p{0.15\linewidth} p{0.38\linewidth}} 

\begin{tabular}{|p{0.3cm}|p{0.3cm}|p{0.3cm}|p{0.3cm}|p{0.3cm}|p{0.3cm}|p{0.3cm}|p{0.3cm}|}
\hline
0 & n & n & n & n & n & n & n\\
\hline
\end{tabular}
& $<$D2$>$ & Data item 2. Low 7 bits.\\
\end{tabular}

Byte 8:

\begin{tabular}{p{0.4\linewidth} p{0.15\linewidth} p{0.38\linewidth}} 

\begin{tabular}{|p{0.3cm}|p{0.3cm}|p{0.3cm}|p{0.3cm}|p{0.3cm}|p{0.3cm}|p{0.3cm}|p{0.3cm}|}
\hline
0 & n & n & n & n & n & n & n\\
\hline
\end{tabular}
& $<$D3$>$ & Data item 3. Low 7 bits.\\
\end{tabular}

Byte 9:

\begin{tabular}{p{0.4\linewidth} p{0.15\linewidth} p{0.38\linewidth}} 

\begin{tabular}{|p{0.3cm}|p{0.3cm}|p{0.3cm}|p{0.3cm}|p{0.3cm}|p{0.3cm}|p{0.3cm}|p{0.3cm}|}
\hline
0 & n & n & n & n & n & n & n\\
\hline
\end{tabular}
& $<$D4$>$ & Data item 4. Low 7 bits.\\
\end{tabular}

Byte 10:

\begin{tabular}{p{0.4\linewidth} p{0.15\linewidth} p{0.38\linewidth}} 

\begin{tabular}{|p{0.3cm}|p{0.3cm}|p{0.3cm}|p{0.3cm}|p{0.3cm}|p{0.3cm}|p{0.3cm}|p{0.3cm}|}
\hline
0 & n & n & n & n & n & n & n\\
\hline
\end{tabular}
& $<$PXCT2$>$ & Data type code and high bits for D5 to D8.\\
\end{tabular}

\begin{tabular}{p{0.05\linewidth} p{0.91\linewidth}} 
d6 & XC5. Data type code.\\
d5 & XC4. Data type code.\\
d4 & XC3. Data type code.\\
d3 & D8.7. High bit\\
d2 & D7.7. High bit\\
d1 & D6.7. High bit\\
d0 & D5.7. High bit\\
\end{tabular}

\begin{tabular}{p{0.1\linewidth} p{0.1\linewidth} p{0.1\linewidth} p{0.6\linewidth}} 
\underline{XC5} & \underline{XC4} & \underline{XC3} & \underline{Meaning}\\
0 & 0 & 0 & ANSI text string. IPL download setup subcode.\\
0 & 0 & 1 & IPL download address subcode.\\
0 & 1 & 0 & IPL download send data subcode.\\
0 & 1 & 1 & IPL download verify data subcode.\\
1 & 0 & 0 & IPL download end of operation subcode.\\
1 & 0 & 1 & reserved.\\
1 & 1 & 0 & reserved.\\
1 & 1 & 1 & reserved.\\
\end{tabular}

Options flags
\begin{verbatim}
private static final int DO_NOT_CHECK_SOFTWARE_VERSION = 0x00;
    private static final int CHECK_SOFTWARE_VERSION_LESS = 0x04;

    private static final int DO_NOT_CHECK_HARDWARE_VERSION = 0x00;
    private static final int REQUIRE_HARDWARE_VERSION_EXACT_MATCH = 0x01;
    private static final int ACCEPT_LATER_HARDWARE_VERSIONS = 0x03;
\end{verbatim}

Byte 11:

\begin{tabular}{p{0.4\linewidth} p{0.15\linewidth} p{0.38\linewidth}} 

\begin{tabular}{|p{0.3cm}|p{0.3cm}|p{0.3cm}|p{0.3cm}|p{0.3cm}|p{0.3cm}|p{0.3cm}|p{0.3cm}|}
\hline
0 & n & n & n & n & n & n & n\\
\hline
\end{tabular}
& $<$D5$>$ & Data item 5. Low 7 bits.\\
\end{tabular}

Byte 12:

\begin{tabular}{p{0.4\linewidth} p{0.15\linewidth} p{0.38\linewidth}} 

\begin{tabular}{|p{0.3cm}|p{0.3cm}|p{0.3cm}|p{0.3cm}|p{0.3cm}|p{0.3cm}|p{0.3cm}|p{0.3cm}|}
\hline
0 & n & n & n & n & n & n & n\\
\hline
\end{tabular}
& $<$D6$>$ & Data item 6. Low 7 bits.\\
\end{tabular}

Byte 13:

\begin{tabular}{p{0.4\linewidth} p{0.15\linewidth} p{0.38\linewidth}} 

\begin{tabular}{|p{0.3cm}|p{0.3cm}|p{0.3cm}|p{0.3cm}|p{0.3cm}|p{0.3cm}|p{0.3cm}|p{0.3cm}|}
\hline
0 & n & n & n & n & n & n & n\\
\hline
\end{tabular}
& $<$D7$>$ & Data item 7. Low 7 bits.\\
\end{tabular}

Byte 14:

\begin{tabular}{p{0.4\linewidth} p{0.15\linewidth} p{0.38\linewidth}} 

\begin{tabular}{|p{0.3cm}|p{0.3cm}|p{0.3cm}|p{0.3cm}|p{0.3cm}|p{0.3cm}|p{0.3cm}|p{0.3cm}|}
\hline
0 & n & n & n & n & n & n & n\\
\hline
\end{tabular}
& $<$D8$>$ & Data item 8. Low 7 bits.\\
\end{tabular}

Byte 15:

\begin{tabular}{p{0.4\linewidth} p{0.1\linewidth} p{0.5\linewidth}} 

\begin{tabular}{|p{0.3cm}|p{0.3cm}|p{0.3cm}|p{0.3cm}|p{0.3cm}|p{0.3cm}|p{0.3cm}|p{0.3cm}|}
\hline
0 & n & n & n & n & n & n & n\\
\hline
\end{tabular}
& $<$CHK$>$ & Checksum.\\
\end{tabular}

\underline{Response:} 

None

\underline{Signature:}

Byte 0:

\begin{tabular}{p{0.4\linewidth} p{0.38\linewidth}} 

\begin{tabular}{|p{0.3cm}|p{0.3cm}|p{0.3cm}|p{0.3cm}|p{0.3cm}|p{0.3cm}|p{0.3cm}|p{0.3cm}|}
\hline
1 & 1 & 1 & 0 & 0 & 1 & 0 & 1\\
\hline
\end{tabular}
& 0xE5 \\
\end{tabular}

Byte 1:

\begin{tabular}{p{0.4\linewidth} p{0.38\linewidth}} 

\begin{tabular}{|p{0.3cm}|p{0.3cm}|p{0.3cm}|p{0.3cm}|p{0.3cm}|p{0.3cm}|p{0.3cm}|p{0.3cm}|}
\hline
0 & 0 & 0 & 1 & 0 & 0 & 0 & 0\\
\hline
\end{tabular}
& 0x10 \\
\end{tabular}

\underline{Notes:} 

None.

\rule{15.1cm}{0.4pt}

\subsubsection{OPC\_PEER\_XFER\_20}
\underline{Operation:} Move bytes peer to peer.

\underline{Group:} \hspace{0.5cm} Variable-Byte Message

\underline{Direction:} \hspace{0.05cm} device $\rightarrow$ device  

\underline{Encoding:} 

Byte 0:

\begin{tabular}{p{0.4\linewidth} p{0.15\linewidth} p{0.38\linewidth}} 

\begin{tabular}{|p{0.3cm}|p{0.3cm}|p{0.3cm}|p{0.3cm}|p{0.3cm}|p{0.3cm}|p{0.3cm}|p{0.3cm}|}
\hline
1 & 1 & 1 & 0 & 0 & 1 & 0 & 1\\
\hline
\end{tabular}
& 0xE5 & Opcode.\\
\end{tabular}

Byte 1:

\begin{tabular}{p{0.4\linewidth} p{0.15\linewidth} p{0.38\linewidth}} 

\begin{tabular}{|p{0.3cm}|p{0.3cm}|p{0.3cm}|p{0.3cm}|p{0.3cm}|p{0.3cm}|p{0.3cm}|p{0.3cm}|}
\hline
0 & 0 & 0 & 1 & 0 & 1 & 0 & 0\\
\hline
\end{tabular}
& 0x14 & Message length (20 bytes).\\
\end{tabular}

Byte 2:

\begin{tabular}{p{0.4\linewidth} p{0.15\linewidth} p{0.38\linewidth}} 

\begin{tabular}{|p{0.3cm}|p{0.3cm}|p{0.3cm}|p{0.3cm}|p{0.3cm}|p{0.3cm}|p{0.3cm}|p{0.3cm}|}
\hline
0 & n & n & n & n & n & n & n\\
\hline
\end{tabular}
& $<$SRC$>$ & Source id in the range 0x00 to 0x7F.\\
\end{tabular}

Byte 3:

\begin{tabular}{p{0.4\linewidth} p{0.15\linewidth} p{0.38\linewidth}} 

\begin{tabular}{|p{0.3cm}|p{0.3cm}|p{0.3cm}|p{0.3cm}|p{0.3cm}|p{0.3cm}|p{0.3cm}|p{0.3cm}|}
\hline
0 & n & n & n & n & n & n & n\\
\hline
\end{tabular}
& $<$DSTL$>$ & Destination id low in the range 0x00 to 0x7F.\\
\end{tabular}

Byte 4:

\begin{tabular}{p{0.4\linewidth} p{0.15\linewidth} p{0.38\linewidth}} 

\begin{tabular}{|p{0.3cm}|p{0.3cm}|p{0.3cm}|p{0.3cm}|p{0.3cm}|p{0.3cm}|p{0.3cm}|p{0.3cm}|}
\hline
0 & n & n & n & n & n & n & n\\
\hline
\end{tabular}
& $<$DSTH$>$ & Destination id high in the range 0x00 to 0x7F.\\
\end{tabular}

Byte 5:

\begin{tabular}{p{0.4\linewidth} p{0.15\linewidth} p{0.38\linewidth}} 

\begin{tabular}{|p{0.3cm}|p{0.3cm}|p{0.3cm}|p{0.3cm}|p{0.3cm}|p{0.3cm}|p{0.3cm}|p{0.3cm}|}
\hline
0 & n & n & n & n & n & n & n\\
\hline
\end{tabular}
& $<$HOST$>$ & Device host identifier.\\
\end{tabular}

This should be 0x00 for discover devices broadcast.

\begin{tabular}{l l}
\underline{Host Id} & \underline{Device}\\
0x01 & LNRP\\
0x04 & UT4\\
0x0C & WTL12\\
0x14 & DB210 Opto\\
0x15 & DB210\\
0x16 & DB220\\
0x1A & DCS210+\\
0x1B & DCS210\\
0x1C & DCS240\\
0x23 & PR3\\
0x24 & PR4\\
0x2A & DT402\\
0x32 & DT500\\
0x33 & DCS51\\
0x34 & DCS52\\
0x3E & DT602\\
0x51 & BXPA1\\
0x58 & BXP88\\
0x5C & UR92\\
0x63 & LNWI\\
\end{tabular}

Byte 6:

\begin{tabular}{p{0.4\linewidth} p{0.15\linewidth} p{0.38\linewidth}} 

\begin{tabular}{|p{0.3cm}|p{0.3cm}|p{0.3cm}|p{0.3cm}|p{0.3cm}|p{0.3cm}|p{0.3cm}|p{0.3cm}|}
\hline
0 & n & n & n & n & n & n & n\\
\hline
\end{tabular}
&  & Hardware version.\\
\end{tabular}

\begin{tabular}{l l}
\underline{Host Id} & \underline{Device}\\
0x00 & Slave all\\
0x18 & Slave RF24\\
\end{tabular}

Byte 7:

\begin{tabular}{p{0.4\linewidth} p{0.15\linewidth} p{0.38\linewidth}} 

\begin{tabular}{|p{0.3cm}|p{0.3cm}|p{0.3cm}|p{0.3cm}|p{0.3cm}|p{0.3cm}|p{0.3cm}|p{0.3cm}|}
\hline
0 & n & n & n & n & n & n & n\\
\hline
\end{tabular}
&  & Reserved.\\
\end{tabular}

Byte 8:

\begin{tabular}{p{0.4\linewidth} p{0.15\linewidth} p{0.38\linewidth}} 

\begin{tabular}{|p{0.3cm}|p{0.3cm}|p{0.3cm}|p{0.3cm}|p{0.3cm}|p{0.3cm}|p{0.3cm}|p{0.3cm}|}
\hline
0 & d6 & d5 & d4 & d3 & d2 & d1 & d0\\
\hline
\end{tabular}
&  & Software Version Number.\\
\end{tabular}

\begin{tabular}{p{0.05\linewidth} p{0.6\linewidth}} 
d6 & version number bit 3\\
d5 & version number bit 2.\\
d4 & version number bit 1\\
d3 & version number bit 0\\
d2 & subversion number bit 2\\
d1 & subversion number bit 1\\
d0 & subversion number bit 0\\
\end{tabular}

e.g. 0x09 decodes as version 1.1.

This is set to 0x00 for discover devices broadcast message.

Byte 9:

\begin{tabular}{p{0.4\linewidth} p{0.15\linewidth} p{0.38\linewidth}} 

\begin{tabular}{|p{0.3cm}|p{0.3cm}|p{0.3cm}|p{0.3cm}|p{0.3cm}|p{0.3cm}|p{0.3cm}|p{0.3cm}|}
\hline
0 & d6 & d5 & d4 & d3 & d2 & d1 & d0\\
\hline
\end{tabular}
& $<$PXCT1$>$ & Address type code and high bits of D1 to D4.\\
\end{tabular}

\begin{tabular}{p{0.05\linewidth} p{0.6\linewidth}} 
d6 & XC2. Address type code.\\
d5 & XC1. Address type code.\\
d4 & XC0. Address type code.\\
d3 & D4.7. High bit\\
d2 & D3.7. High bit\\
d1 & D2.7. High bit\\
d0 & D1.7. High bit\\
\end{tabular}

\begin{tabular}{p{0.1\linewidth} p{0.1\linewidth} p{0.1\linewidth} p{0.4\linewidth}} 
\underline{XC2} & \underline{XC1} & \underline{XC0} & \underline{Meaning}\\
0 & 0 & 0 & 7 bit peer to peer addresses.\\
0 & 0 & 1 & reserved.\\
0 & 1 & 0 & reserved.\\
0 & 1 & 1 & reserved.\\
1 & 0 & 0 & reserved.\\
1 & 0 & 1 & reserved.\\
1 & 1 & 0 & reserved.\\
1 & 1 & 1 & reserved.\\
\end{tabular}

Byte 10:

\begin{tabular}{p{0.4\linewidth} p{0.15\linewidth} p{0.38\linewidth}} 

\begin{tabular}{|p{0.3cm}|p{0.3cm}|p{0.3cm}|p{0.3cm}|p{0.3cm}|p{0.3cm}|p{0.3cm}|p{0.3cm}|}
\hline
0 & n & n & n & n & n & n & n\\
\hline
\end{tabular}
& $<$D1$>$ & Data item 1. Low 7 bits.\\
\end{tabular}

Byte 11:

\begin{tabular}{p{0.4\linewidth} p{0.15\linewidth} p{0.38\linewidth}} 

\begin{tabular}{|p{0.3cm}|p{0.3cm}|p{0.3cm}|p{0.3cm}|p{0.3cm}|p{0.3cm}|p{0.3cm}|p{0.3cm}|}
\hline
0 & n & n & n & n & n & n & n\\
\hline
\end{tabular}
& $<$D2$>$ & Data item 2. Low 7 bits.\\
\end{tabular}

This should be 0x01 for a discover devices broadcast message.

Byte 12:

\begin{tabular}{p{0.4\linewidth} p{0.15\linewidth} p{0.38\linewidth}} 

\begin{tabular}{|p{0.3cm}|p{0.3cm}|p{0.3cm}|p{0.3cm}|p{0.3cm}|p{0.3cm}|p{0.3cm}|p{0.3cm}|}
\hline
0 & n & n & n & n & n & n & n\\
\hline
\end{tabular}
& $<$D3$>$ & Data item 3. Low 7 bits.\\
\end{tabular}

Byte 13:

\begin{tabular}{p{0.4\linewidth} p{0.15\linewidth} p{0.38\linewidth}} 

\begin{tabular}{|p{0.3cm}|p{0.3cm}|p{0.3cm}|p{0.3cm}|p{0.3cm}|p{0.3cm}|p{0.3cm}|p{0.3cm}|}
\hline
0 & n & n & n & n & n & n & n\\
\hline
\end{tabular}
& $<$D4$>$ & Data item 4. Low 7 bits.\\
\end{tabular}

Byte 14:

\begin{tabular}{p{0.4\linewidth} p{0.15\linewidth} p{0.38\linewidth}} 

\begin{tabular}{|p{0.3cm}|p{0.3cm}|p{0.3cm}|p{0.3cm}|p{0.3cm}|p{0.3cm}|p{0.3cm}|p{0.3cm}|}
\hline
0 & n & n & n & n & n & n & n\\
\hline
\end{tabular}
& $<$PXCT2$>$ & Data type code and high bits for D5 to D8.\\
\end{tabular}

\begin{tabular}{p{0.05\linewidth} p{0.6\linewidth}} 
d6 & XC5. Data type code.\\
d5 & XC4. Data type code.\\
d4 & XC3. Data type code.\\
d3 & D8.7. High bit\\
d2 & D7.7. High bit\\
d1 & D6.7. High bit\\
d0 & D5.7. High bit\\
\end{tabular}

\begin{tabular}{p{0.1\linewidth} p{0.1\linewidth} p{0.1\linewidth} p{0.4\linewidth}} 
\underline{XC5} & \underline{XC4} & \underline{XC3} & \underline{Meaning}\\
0 & 0 & 0 & ANSI text string.\\
0 & 0 & 1 & reserved.\\
0 & 1 & 0 & reserved.\\
0 & 1 & 1 & reserved.\\
1 & 0 & 0 & reserved.\\
1 & 0 & 1 & reserved.\\
1 & 1 & 0 & reserved.\\
1 & 1 & 1 & reserved.\\
\end{tabular}

Byte 15:

\begin{tabular}{p{0.4\linewidth} p{0.15\linewidth} p{0.38\linewidth}} 

\begin{tabular}{|p{0.3cm}|p{0.3cm}|p{0.3cm}|p{0.3cm}|p{0.3cm}|p{0.3cm}|p{0.3cm}|p{0.3cm}|}
\hline
0 & n & n & n & n & n & n & n\\
\hline
\end{tabular}
& $<$D5$>$ & Data item 5. Low 7 bits.\\
\end{tabular}

Byte 16:

\begin{tabular}{p{0.4\linewidth} p{0.15\linewidth} p{0.38\linewidth}} 

\begin{tabular}{|p{0.3cm}|p{0.3cm}|p{0.3cm}|p{0.3cm}|p{0.3cm}|p{0.3cm}|p{0.3cm}|p{0.3cm}|}
\hline
0 & n & n & n & n & n & n & n\\
\hline
\end{tabular}
& $<$D6$>$ & Data item 6. Low 7 bits.\\
\end{tabular}

Byte 17:

\begin{tabular}{p{0.4\linewidth} p{0.15\linewidth} p{0.38\linewidth}} 

\begin{tabular}{|p{0.3cm}|p{0.3cm}|p{0.3cm}|p{0.3cm}|p{0.3cm}|p{0.3cm}|p{0.3cm}|p{0.3cm}|}
\hline
0 & n & n & n & n & n & n & n\\
\hline
\end{tabular}
& $<$D7$>$ & Data item 7. Low 7 bits.\\
\end{tabular}

Byte 18:

\begin{tabular}{p{0.4\linewidth} p{0.15\linewidth} p{0.38\linewidth}} 

\begin{tabular}{|p{0.3cm}|p{0.3cm}|p{0.3cm}|p{0.3cm}|p{0.3cm}|p{0.3cm}|p{0.3cm}|p{0.3cm}|}
\hline
0 & n & n & n & n & n & n & n\\
\hline
\end{tabular}
& $<$D8$>$ & Data item 8. Low 7 bits.\\
\end{tabular}

Byte 19:

\begin{tabular}{p{0.4\linewidth} p{0.1\linewidth} p{0.5\linewidth}} 

\begin{tabular}{|p{0.3cm}|p{0.3cm}|p{0.3cm}|p{0.3cm}|p{0.3cm}|p{0.3cm}|p{0.3cm}|p{0.3cm}|}
\hline
0 & n & n & n & n & n & n & n\\
\hline
\end{tabular}
& $<$CHK$>$ & Checksum.\\
\end{tabular}

\underline{Description:}

This command sends the data from one device to another peer to peer.

\begin{tabular}{p{0.2\linewidth} p{0.1\linewidth} p{0.1\linewidth} p{0.50\linewidth}} 
\underline{SRC} & \underline{DSTL} & \underline{DSTH} & Comments\\
0x0F & 0x08 & 0x00 & Discover devices broadcast message.\\
0xoF & 0x10 & 0x00 & Discover device response.\\
\end{tabular}

\underline{Response:} 

OPC\_PEER\_XFER\_20 for discover devices.

\underline{Notes:} 

The discover response decoded peer transfer message encodes as follows:

\begin{tabular}{p{0.05\linewidth} p{0.6\linewidth}} 
D1 & IPL Version Number\\
D2 & Serial Number - low byte\\
D3 & Serial Number - high byte\\
D4 & \\
D5 & Serial Number 2 - low byte\\
D6 & Serial Number 2 - high byte\\
D7 & \\
D8 & \\
\end{tabular}

The IPL version number is encoded as follows:

\begin{tabular}{p{0.05\linewidth} p{0.6\linewidth}} 
d6 & version number bit 3\\
d5 & version number bit 2.\\
d4 & version number bit 1\\
d3 & version number bit 0\\
d2 & subversion number bit 2\\
d1 & subversion number bit 1\\
d0 & subversion number bit 0\\
\end{tabular}

e.g. 0x09 decodes as version 1.1.

These came from DigiPLII:

message Length = 20
e5 14 0f 10 00 24 00 00 00 02 00 08 07 00 00 00 00 00 00 38 

message Length = 20
e5 14 0f 08 00 00 00 00 00 00 00 01 00 00 00 00 00 00 00 08 

message Length = 20
e5 14 0f 10 00 24 00 00 00 00 00 57 13 00 00 00 00 00 00 71 

message Length = 20
e5 14 0f 10 00 1b 00 00 03 02 00 54 10 00 00 00 00 00 00 4f 

It reports PR4 with serial number 0x0788 ver 0
PR4 with serial 0x1357 ver 0
DCS240 with SN 0x0AAB ver 0.3
DCS210 with SN 0x10D4 ver 0.3


\rule{15.1cm}{0.4pt}

\input{Chapter1-PwrOff}
\input{Chapter1-PwrOn}
\newpage
\subsection{Reset}
\underline{Description:}

This broadcast message is sent by a command station when its ``Loco Reset" button has been pressed. Software should reload any locally cached slot data from the command station.

\underline{Group:}

2-Byte Message

\underline{Opcode:}

OPC\_LOCO\_RESET

\underline{Type:}

Broadcast

\underline{Encoding:} 

Byte 0:

\begin{tabular}{p{0.4\linewidth} p{0.15\linewidth} p{0.38\linewidth}} 

\begin{tabular}{|p{0.3cm}|p{0.3cm}|p{0.3cm}|p{0.3cm}|p{0.3cm}|p{0.3cm}|p{0.3cm}|p{0.3cm}|}
\hline
1 & 0 & 0 & 0 & 1 & 0 & 1 & 0\\
\hline
\end{tabular}
& 0x8A & Opcode.\\
\end{tabular}

Byte 1:

\begin{tabular}{p{0.4\linewidth} p{0.15\linewidth} p{0.38\linewidth}} 

\begin{tabular}{|p{0.3cm}|p{0.3cm}|p{0.3cm}|p{0.3cm}|p{0.3cm}|p{0.3cm}|p{0.3cm}|p{0.3cm}|}
\hline
0 & 1 & 1 & 1 & 0 & 1 & 0 & 1\\
\hline
\end{tabular}
& 0x75 & Checksum.
\end{tabular}

\underline{Response:} 

None.

\underline{Signature:}

Byte 0:

\begin{tabular}{p{0.4\linewidth} p{0.38\linewidth}} 

\begin{tabular}{|p{0.3cm}|p{0.3cm}|p{0.3cm}|p{0.3cm}|p{0.3cm}|p{0.3cm}|p{0.3cm}|p{0.3cm}|}
\hline
1 & 0 & 0 & 0 & 1 & 0 & 1 & 0\\
\hline
\end{tabular}
& 0x8A \\
\end{tabular}

\underline{Notes:} 

None.

\rule{15.1cm}{0.4pt}

\input{Chapter1-SensRepGIn}
\newpage
\subsection{OPC\_SW\_REP}
\underline{Operation:} Turnout sensor report.

\underline{Group:} \hspace{0.5cm} 4-Byte Message

\underline{Direction:} \hspace{0.05cm} Turnout sensor $\rightarrow$ 

\underline{Encoding:} 

Byte 0:

\begin{tabular}{p{0.4\linewidth} p{0.15\linewidth} p{0.38\linewidth}} 

\begin{tabular}{|p{0.3cm}|p{0.3cm}|p{0.3cm}|p{0.3cm}|p{0.3cm}|p{0.3cm}|p{0.3cm}|p{0.3cm}|}
\hline
1 & 0 & 1 & 1 & 0 & 0 & 0 & 1\\
\hline
\end{tabular}
& 0xB1 & Opcode.\\
\end{tabular}

Byte 1:

\begin{tabular}{p{0.4\linewidth} p{0.15\linewidth} p{0.38\linewidth}} 

\begin{tabular}{|p{0.3cm}|p{0.3cm}|p{0.3cm}|p{0.3cm}|p{0.3cm}|p{0.3cm}|p{0.3cm}|p{0.3cm}|}
\hline
0 & d6 & d5 & d4 & d3 & d2 & d1 & d0\\
\hline
\end{tabular}
& $<$SN1$>$ & Sensor address.\\
\end{tabular}

\begin{tabular}{p{0.5\linewidth} p{0.5\linewidth}}
\underline{SN2.d6 = 1} & \underline{SN2.d6 = 0}\\
\begin{tabular}{p{0.05\linewidth} p{0.6\linewidth}} 
d6 & A7.\\
d5 & A6.\\
d4 & A5.\\
d3 & A4.\\
d2 & A3.\\
d1 & A2.\\
d0 & A1.\\
\end{tabular} &
\begin{tabular}{p{0.05\linewidth} p{0.6\linewidth}} 
d6 & A6.\\
d5 & A5.\\
d4 & A4.\\
d3 & A3.\\
d2 & A2.\\
d1 & A1.\\
d0 & A0.\\
\end{tabular} \\

\end{tabular}

Byte 2:

\begin{tabular}{p{0.4\linewidth} p{0.15\linewidth} p{0.38\linewidth}} 

\begin{tabular}{|p{0.3cm}|p{0.3cm}|p{0.3cm}|p{0.3cm}|p{0.3cm}|p{0.3cm}|p{0.3cm}|p{0.3cm}|}
\hline
0 & d6 & d5 & d4 & d3 & d2 & d1 & d0\\
\hline
\end{tabular}
& $<$SN2$>$ & Sensor address and sensor state.\\
\end{tabular}

\begin{tabular}{p{0.555\linewidth} p{0.555\linewidth}}

\underline{SN2.d6 = 1} & \underline{SN2.d6 = 0}\\
\begin{tabular}{p{0.05\linewidth} p{0.6\linewidth}} 
d6 & Report type. 1 means the report is an input report, and 0 means the report is an output report.\\
d5 & A0.\\
d4 & Input sensor state, 1 means sensor $>=$ 6V, 0 means sensor = 0V.\\
d3 & A11.\\
d2 & A10.\\
d1 & A9.\\
d0 & A8.\\
\end{tabular} &
\begin{tabular}{p{0.05\linewidth} p{0.6\linewidth}} 
d6 & Report type. 1 means the report is an input report, and 0 means the report is an output report.\\
d5 & 0 means closed output line is off, 1 means the closed output line is on.\\
d4 & 0 means thrown output line is off, 1 means the thrown output line is on.\\
d3 & A10.\\
d2 & A9.\\
d1 & A8.\\
d0 & A7.\\
\end{tabular} \\

\end{tabular}

Byte 3:

\begin{tabular}{p{0.4\linewidth} p{0.15\linewidth} p{0.38\linewidth}} 

\begin{tabular}{|p{0.3cm}|p{0.3cm}|p{0.3cm}|p{0.3cm}|p{0.3cm}|p{0.3cm}|p{0.3cm}|p{0.3cm}|}
\hline
0 & n & n & n & n & n & n & n\\
\hline
\end{tabular}
& $<$CHK$>$ & Checksum.

\end{tabular}

\underline{Description:}

Turnout sensor report.

\underline{Response:} 

None.

\underline{Notes:} 

None.

\rule{15.1cm}{0.4pt}

\input{Chapter1-SensRepTOut}
\newpage
\subsection{OPC\_BRD\_OPSW}

\underline{Operation:} Read and write board option switches.

\underline{Group:} \hspace{0.5cm} 6-Byte Message

\underline{Direction:} \hspace{0.05cm} $\rightarrow$ Command Station

\underline{Encoding:} 

Byte 0:

\begin{tabular}{p{0.4\linewidth} p{0.15\linewidth} p{0.38\linewidth}} 

\begin{tabular}{|p{0.3cm}|p{0.3cm}|p{0.3cm}|p{0.3cm}|p{0.3cm}|p{0.3cm}|p{0.3cm}|p{0.3cm}|}
\hline
1 & 1 & 0 & 1 & 0 & 0 & 0 & 0\\
\hline
\end{tabular}
& 0xD0 & Opcode.\\
\end{tabular}

Byte 1:

\begin{tabular}{p{0.4\linewidth} p{0.15\linewidth} p{0.38\linewidth}} 

\begin{tabular}{|p{0.3cm}|p{0.3cm}|p{0.3cm}|p{0.3cm}|p{0.3cm}|p{0.3cm}|p{0.3cm}|p{0.3cm}|}
\hline
0 & 1 & 1 & d4 & 0 & 0 & 1 & d0\\
\hline
\end{tabular}
&  & The bit d0 is the most significant bit of the board id. Bit d4 indicates read/write direction. 1 means write and 0 means read.\\
\end{tabular}

Byte 2:

\begin{tabular}{p{0.4\linewidth} p{0.15\linewidth} p{0.38\linewidth}} 

\begin{tabular}{|p{0.3cm}|p{0.3cm}|p{0.3cm}|p{0.3cm}|p{0.3cm}|p{0.3cm}|p{0.3cm}|p{0.3cm}|}
\hline
0 & n & n & n & n & n & n & n\\
\hline
\end{tabular}
& $<$BIDL$>$ & Least significant 7 bits of the board id.\\
\end{tabular}

Byte 3:

\begin{tabular}{p{0.4\linewidth} p{0.15\linewidth} p{0.38\linewidth}} 

\begin{tabular}{|p{0.3cm}|p{0.3cm}|p{0.3cm}|p{0.3cm}|p{0.3cm}|p{0.3cm}|p{0.3cm}|p{0.3cm}|}
\hline
0 & n & n & n & n & n & n & n\\
\hline
\end{tabular}
& $<$BTYPE$>$ & Board type code.\\
\end{tabular}

\begin{tabular}{p{0.2\linewidth} p{0.5\linewidth}} 
\underline{Board} & \underline{Type Code}\\
PM4 & 0x70.\\
BDL16 & 0x71.\\
SE8C & 0x72.\\
DS64 & 0x73.\\
\end{tabular}

Byte 4:

\begin{tabular}{p{0.4\linewidth} p{0.15\linewidth} p{0.38\linewidth}} 

\begin{tabular}{|p{0.3cm}|p{0.3cm}|p{0.3cm}|p{0.3cm}|p{0.3cm}|p{0.3cm}|p{0.3cm}|p{0.3cm}|}
\hline
0 & d6 & d5 & d4 & d3 & d2 & d1 & d0\\
\hline
\end{tabular}
&  & Byte and bit number. The high nibble encodes the byte number, and the low nibble the bit number.\\
\end{tabular}

The byte number is calculated as (OpSw\# - 1) $>>$ 3 and the bit number is (OpSw\# - 1) - byte number $\times$ 8.

Byte 5:

\begin{tabular}{p{0.4\linewidth} p{0.15\linewidth} p{0.38\linewidth}} 

\begin{tabular}{|p{0.3cm}|p{0.3cm}|p{0.3cm}|p{0.3cm}|p{0.3cm}|p{0.3cm}|p{0.3cm}|p{0.3cm}|}
\hline
0 & n & n & n & n & n & n & n\\
\hline
\end{tabular}
& $<$CHK$>$ & Checksum.
\end{tabular}

\underline{Description:}

\underline{Response:} 

OPC\_LONG\_ACK.

\underline{Notes:} 

None.


\newpage
\section{SetIdleState}\index{SetIdleState}

\rule{15.1cm}{0.4pt}

\underline{Description:}

This \gls{Command} sets the network to \gls{Idle} state and the command station broadcasts an emergency stop.

\underline{Group:}

2-Byte Message

\underline{Opcode:}

OPC\_IDLE

\underline{Type:}

Command

\underline{Encoding:} 

Byte 0:

\begin{tabular}{p{0.4\linewidth} p{0.15\linewidth} p{0.38\linewidth}} 

\begin{tabular}{|p{0.3cm}|p{0.3cm}|p{0.3cm}|p{0.3cm}|p{0.3cm}|p{0.3cm}|p{0.3cm}|p{0.3cm}|}
\hline
1 & 0 & 0 & 0 & 0 & 1 & 0 & 1\\
\hline
\end{tabular}
& 0x85 & Opcode.\\
\end{tabular}

Byte 1:

\begin{tabular}{p{0.4\linewidth} p{0.15\linewidth} p{0.38\linewidth}} 

\begin{tabular}{|p{0.3cm}|p{0.3cm}|p{0.3cm}|p{0.3cm}|p{0.3cm}|p{0.3cm}|p{0.3cm}|p{0.3cm}|}
\hline
0 & 1 & 1 & 1 & 1 & 0 & 1 & 0\\
\hline
\end{tabular}
& 0x7A & Checksum.\\
\end{tabular}

\underline{Response:} 

None.

\underline{Signature:}

Byte 0:

\begin{tabular}{p{0.4\linewidth} p{0.15\linewidth} p{0.38\linewidth}} 

\begin{tabular}{|p{0.3cm}|p{0.3cm}|p{0.3cm}|p{0.3cm}|p{0.3cm}|p{0.3cm}|p{0.3cm}|p{0.3cm}|}
\hline
1 & 0 & 0 & 0 & 0 & 1 & 0 & 1\\
\hline
\end{tabular}
& 0x85 & \\
\end{tabular}

\underline{Notes:} 

This doesn't seem to work.

\rule{15.1cm}{0.4pt}

\newpage
\subsection{SetLocoSlotDataP1}

\rule{15.1cm}{0.4pt}

\underline{Description:}

This command sets the locomotive slot data for the specified slot.

\underline{Protocol:}

1

\underline{Group:}

Variable-Byte Message

\underline{Opcode:}

OPC\_WR\_SL\_DATA

\underline{Type:}

Command

\underline{Encoding:} 

Byte 0:

\begin{tabular}{p{0.4\linewidth} p{0.15\linewidth} p{0.38\linewidth}} 

\begin{tabular}{|p{0.3cm}|p{0.3cm}|p{0.3cm}|p{0.3cm}|p{0.3cm}|p{0.3cm}|p{0.3cm}|p{0.3cm}|}
\hline
1 & 1 & 1 & 0 & 1 & 1 & 1 & 1\\
\hline
\end{tabular}
& 0xEF & Opcode.\\
\end{tabular}

Byte 1:

\begin{tabular}{p{0.4\linewidth} p{0.15\linewidth} p{0.38\linewidth}} 

\begin{tabular}{|p{0.3cm}|p{0.3cm}|p{0.3cm}|p{0.3cm}|p{0.3cm}|p{0.3cm}|p{0.3cm}|p{0.3cm}|}
\hline
0 & 0 & 0 & 0 & 1 & 1 & 1 & 0\\
\hline
\end{tabular}
& 0x0E & Message length (14 bytes).\\
\end{tabular}

Byte 2:

\begin{tabular}{p{0.4\linewidth} p{0.15\linewidth} p{0.375\linewidth}} 

\begin{tabular}{|p{0.3cm}|p{0.3cm}|p{0.3cm}|p{0.3cm}|p{0.3cm}|p{0.3cm}|p{0.3cm}|p{0.3cm}|}
\hline
0 & n & n & n & n & n & n & n\\
\hline
\end{tabular}
& $<$SLOT\#$>$ & Slot number in the range 0x00 to 0x77. Slot 0x00 is the dispatch special slot.\\
\end{tabular}

Byte 3:

\begin{tabular}{p{0.4\linewidth} p{0.15\linewidth} p{0.38\linewidth}} 

\begin{tabular}{|p{0.3cm}|p{0.3cm}|p{0.3cm}|p{0.3cm}|p{0.3cm}|p{0.3cm}|p{0.3cm}|p{0.3cm}|}
\hline
d7 & d6 & d5 & d4 & d3 & d2 & d1 & d0\\
\hline
\end{tabular}
& $<$STAT1$>$ & Slot status 1.\\
\end{tabular}

\begin{tabular}{p{0.05\linewidth} p{0.05\linewidth} p{0.05\linewidth} p{0.70\linewidth}} 
& \underline{d7} & \underline{d6} & \\
& 0 & 0 & Free, no consist linking.\\
& 0 & 1 & Consist sub-member.\\
& 1 & 0 & Consist top-member.\\
& 1 & 1 & Consist Mid-Consist member. \\
\end{tabular}

Note: d7 is set to 0 in the message by the command station and so may not correctly reflect the actual setting in the slot table.\\

\begin{tabular}{p{0.05\linewidth} p{0.05\linewidth} p{0.05\linewidth} p{0.70\linewidth}} 
& \underline{d5} & \underline{d4} & \\
& 0 & 0 & Free slot, no valid data. Not refreshed.\\
& 0 & 1 & Common. Locomotive address in this slot. Refreshed.\\
& 1 & 0 & Idle. Locomotive address in this slot. Not refreshed.\\
& 1 & 1 & In Use. Locomotive address in this slot. Refreshed. \\
\end{tabular}

 \begin{tabular}{p{0.05\linewidth} p{0.05\linewidth} p{0.05\linewidth} p{0.70\linewidth}} 
& & \underline{d3} & \\
& & 0 & No slot consist linked into this slot.\\
& & 1 & Slot consist linked into this slot.\\
\end{tabular}

\begin{tabular}{p{0.05\linewidth} p{0.05\linewidth} p{0.05\linewidth} p{0.75\linewidth}} 
\underline{d2} & \underline{d1} & \underline{d0} & \\
0 & 0 & 0 & 28 step decoder. 3-byte packet regular mode\\
0 & 0 & 1 & 28 step decoder. Generate Motorola trinary packets for this mobile address\\
0 & 1 & 0 & 14 step decoder. \\
0 & 1 & 1 & 128 step decoder. \\
1 & 0 & 0 & 28 step decoder with \gls{Advanced Consisting} enabled\\
1 & 0 & 1 & reserved\\
1 & 1 & 0 & reserved\\
1 & 1 & 1 & 128 step decoder with \gls{Advanced Consisting} enabled\\
\end{tabular}

Byte 4:

\begin{tabular}{p{0.4\linewidth} p{0.15\linewidth} p{0.38\linewidth}} 

\begin{tabular}{|p{0.3cm}|p{0.3cm}|p{0.3cm}|p{0.3cm}|p{0.3cm}|p{0.3cm}|p{0.3cm}|p{0.3cm}|}
\hline
0 & n & n & n & n & n & n & n\\
\hline
\end{tabular}
& $<$ADR$>$ & If $<$ADR2$>$ is 0 then this contains the NMRA short address. If $<$ADR2$>$ is greater than 0 then this contains the low 7 bits of the NMRA long address.\\
\end{tabular}

Byte 5:

\begin{tabular}{p{0.4\linewidth} p{0.15\linewidth} p{0.36\linewidth}} 

\begin{tabular}{|p{0.3cm}|p{0.3cm}|p{0.3cm}|p{0.3cm}|p{0.3cm}|p{0.3cm}|p{0.3cm}|p{0.3cm}|}
\hline
0 & n & n & n & n & n & n & n\\
\hline
\end{tabular}
& $<$SPD$>$ & Speed in the range 0x00 to 0x7F. 0x00 means inertial stop and 0x01 means emergency stop. Other values mean increasing speed.\\
\end{tabular}

Byte 6:

\begin{tabular}{p{0.4\linewidth} p{0.15\linewidth} p{0.36\linewidth}} 

\begin{tabular}{|p{0.3cm}|p{0.3cm}|p{0.3cm}|p{0.3cm}|p{0.3cm}|p{0.3cm}|p{0.3cm}|p{0.3cm}|}
\hline
0 & 0 & d5 & d4 & d3 & d2 & d1 & d0\\
\hline
\end{tabular}
& $<$DIRF$>$ & Locomotive direction and state of functions F0 to F4.\\
\end{tabular}

\begin{tabular}{p{0.05\linewidth} p{0.95\linewidth}} 
d5 & Direction: 1 means forward and 0 means backwards.\\
d4 & F0 state: 1 means on and 0 means off.\\
d3 & F4 state: 1 means on and 0 means off.\\
d2 & F3 state: 1 means on and 0 means off.\\
d1 & F2 state: 1 means on and 0 means off.\\
d0 & F1 state: 1 means on and 0 means off.\\
\end{tabular}

Byte 7:\index{Global System Track Status}

\begin{tabular}{p{0.4\linewidth} p{0.15\linewidth} p{0.38\linewidth}} 

\begin{tabular}{|p{0.3cm}|p{0.3cm}|p{0.3cm}|p{0.3cm}|p{0.3cm}|p{0.3cm}|p{0.3cm}|p{0.3cm}|}
\hline
0 & d6 & 0 & 0 & d3 & d2 & d1 & d0\\
\hline
\end{tabular}
& $<$TRK$>$ & \gls{Global System Track Status}.\\
\end{tabular}

\begin{tabular}{p{0.05\linewidth} p{0.9\linewidth}} 
d6 & 1 means this command station implements protocol 2 commands.\\
d3 & 1 means the programming track is busy.\\
d2 & 1 means this command station implements protocol 1 commands,  0 means the command station is a DT200.\\
d1 & 0 means the track is paused, broadcast an emergency stop.\\
d0 & 1 means the DCC packets are on and global power is up.\\
\end{tabular}

Byte 8:

\begin{tabular}{p{0.4\linewidth} p{0.15\linewidth} p{0.38\linewidth}} 

\begin{tabular}{|p{0.3cm}|p{0.3cm}|p{0.3cm}|p{0.3cm}|p{0.3cm}|p{0.3cm}|p{0.3cm}|p{0.3cm}|}
\hline
0 & 0 & 0 & 0 & d3 & d2 & 0 & d0\\
\hline
\end{tabular}
& $<$SS2$>$ & Slot status 2.\\
\end{tabular}

\begin{tabular}{p{0.05\linewidth} p{0.6\linewidth}} 
d3 & 1 means expansion in ID1/2, 0 means encoded alias.\\
d2 & 1 means expansion ID1/2 is not ID usage.\\
d0 & 1 means this slot has suppressed advanced consist.\\
\end{tabular}

Byte 9:

\begin{tabular}{p{0.4\linewidth} p{0.15\linewidth} p{0.38\linewidth}} 

\begin{tabular}{|p{0.3cm}|p{0.3cm}|p{0.3cm}|p{0.3cm}|p{0.3cm}|p{0.3cm}|p{0.3cm}|p{0.3cm}|}
\hline
0 & n & n & n & n & n & n & n\\
\hline
\end{tabular}
& $<$ADR2$>$ & If $<$ADR2$>$ is greater than 0 then this contains the high 7 bits of the NMRA long address.\\
\end{tabular}

Byte 10:

\begin{tabular}{p{0.4\linewidth} p{0.15\linewidth} p{0.38\linewidth}} 

\begin{tabular}{|p{0.3cm}|p{0.3cm}|p{0.3cm}|p{0.3cm}|p{0.3cm}|p{0.3cm}|p{0.3cm}|p{0.3cm}|}
\hline
0 & 0 & 0 & 0 & d3 & d2 & d1 & d0\\
\hline
\end{tabular}
& $<$SND$>$ & Function F5 to F8 states.\\
\end{tabular}

\begin{tabular}{p{0.05\linewidth} p{0.6\linewidth}} 
d3 & F8 state: 1 means on and 0 means off.\\
d2 & F7 state: 1 means on and 0 means off.\\
d1 & F6 state: 1 means on and 0 means off.\\
d0 & F5 state: 1 means on and 0 means off.\\
\end{tabular}

Byte 11:

\begin{tabular}{p{0.4\linewidth} p{0.15\linewidth} p{0.37\linewidth}} 

\begin{tabular}{|p{0.3cm}|p{0.3cm}|p{0.3cm}|p{0.3cm}|p{0.3cm}|p{0.3cm}|p{0.3cm}|p{0.3cm}|}
\hline
0 & n & n & n & n & n & n & n\\
\hline
\end{tabular}
& $<$ID1$>$ & 7-bit ls ID code written by throttle when STAT2.4 = 1.\\
\end{tabular}

Byte 12:

\begin{tabular}{p{0.4\linewidth} p{0.15\linewidth} p{0.37\linewidth}} 

\begin{tabular}{|p{0.3cm}|p{0.3cm}|p{0.3cm}|p{0.3cm}|p{0.3cm}|p{0.3cm}|p{0.3cm}|p{0.3cm}|}
\hline
0 & n & n & n & n & n & n & n\\
\hline
\end{tabular}
& $<$ID2$>$ & 7-bit ms ID code written by throttle when STAT2.4 = 1.\\
\end{tabular}



Byte 13:

\begin{tabular}{p{0.4\linewidth} p{0.1\linewidth} p{0.5\linewidth}} 

\begin{tabular}{|p{0.3cm}|p{0.3cm}|p{0.3cm}|p{0.3cm}|p{0.3cm}|p{0.3cm}|p{0.3cm}|p{0.3cm}|}
\hline
0 & n & n & n & n & n & n & n\\
\hline
\end{tabular}
& $<$CHK$>$ & Checksum.\\
\end{tabular}

\underline{Response:} 

Returns OPC\_LONG\_ACK.

\underline{Signature:}

Byte 0:

\begin{tabular}{p{0.4\linewidth} p{0.15\linewidth} p{0.38\linewidth}} 

\begin{tabular}{|p{0.3cm}|p{0.3cm}|p{0.3cm}|p{0.3cm}|p{0.3cm}|p{0.3cm}|p{0.3cm}|p{0.3cm}|}
\hline
1 & 1 & 1 & 0 & 1 & 1 & 1 & 1\\
\hline
\end{tabular}
& 0xEF\\
\end{tabular}

Byte 1:

\begin{tabular}{p{0.4\linewidth} p{0.15\linewidth} p{0.38\linewidth}} 

\begin{tabular}{|p{0.3cm}|p{0.3cm}|p{0.3cm}|p{0.3cm}|p{0.3cm}|p{0.3cm}|p{0.3cm}|p{0.3cm}|}
\hline
0 & 0 & 0 & 0 & 1 & 1 & 1 & 0\\
\hline
\end{tabular}
& 0x0E \\
\end{tabular}

Byte 2:

\begin{tabular}{p{0.4\linewidth} p{0.375\linewidth}} 

\begin{tabular}{|p{0.3cm}|p{0.3cm}|p{0.3cm}|p{0.3cm}|p{0.3cm}|p{0.3cm}|p{0.3cm}|p{0.3cm}|}
\hline
0 & n & n & n & n & n & n & n\\
\hline
\end{tabular}
& less than 0x78
\end{tabular}

Byte 6:

\begin{tabular}{p{0.4\linewidth} p{0.15\linewidth} p{0.36\linewidth}} 

\begin{tabular}{|p{0.3cm}|p{0.3cm}|p{0.3cm}|p{0.3cm}|p{0.3cm}|p{0.3cm}|p{0.3cm}|p{0.3cm}|}
\hline
0 & 0 & $\times$ & $\times$ & $\times$ & $\times$ & $\times$ & $\times$\\
\hline
\end{tabular}
& \\ 
\end{tabular}

Byte 7:

\begin{tabular}{p{0.4\linewidth} p{0.15\linewidth} p{0.38\linewidth}} 

\begin{tabular}{|p{0.3cm}|p{0.3cm}|p{0.3cm}|p{0.3cm}|p{0.3cm}|p{0.3cm}|p{0.3cm}|p{0.3cm}|}
\hline
0 & $\times$ & 0 & 0 & $\times$ & $\times$ & $\times$ & $\times$\\
\hline
\end{tabular}
& \\
\end{tabular}

Byte 8:

\begin{tabular}{p{0.4\linewidth} p{0.15\linewidth} p{0.38\linewidth}} 

\begin{tabular}{|p{0.3cm}|p{0.3cm}|p{0.3cm}|p{0.3cm}|p{0.3cm}|p{0.3cm}|p{0.3cm}|p{0.3cm}|}
\hline
0 & 0 & 0 & 0 & $\times$ & $\times$ & 0 & $\times$\\
\hline
\end{tabular}
& \\ 
\end{tabular}

Byte 10:

\begin{tabular}{p{0.4\linewidth} p{0.15\linewidth} p{0.38\linewidth}} 

\begin{tabular}{|p{0.3cm}|p{0.3cm}|p{0.3cm}|p{0.3cm}|p{0.3cm}|p{0.3cm}|p{0.3cm}|p{0.3cm}|}
\hline
0 & 0 & 0 & 0 & $\times$ & $\times$ & $\times$ & $\times$\\
\hline
\end{tabular}
& \\
\end{tabular}

\underline{Notes:} 

None.

\rule{15.1cm}{0.4pt}


\input{Chapter1-SetLocoSlotDataP2}
\newpage
\subsection{OPC\_SLOT\_STAT1}
\underline{Operation:} Set slot status 1.

\underline{Group:} \hspace{0.5cm} 4-Byte Message

\underline{Direction:} \hspace{0.05cm} $\rightarrow$ Command Station

\underline{Encoding:} 

Byte 0:

\begin{tabular}{p{0.4\linewidth} p{0.15\linewidth} p{0.38\linewidth}} 

\begin{tabular}{|p{0.3cm}|p{0.3cm}|p{0.3cm}|p{0.3cm}|p{0.3cm}|p{0.3cm}|p{0.3cm}|p{0.3cm}|}
\hline
1 & 0 & 1 & 1 & 0 & 1 & 0 & 1\\
\hline
\end{tabular}
& 0xB5 & Opcode.\\
\end{tabular}

Byte 1:

\begin{tabular}{p{0.4\linewidth} p{0.15\linewidth} p{0.38\linewidth}} 

\begin{tabular}{|p{0.3cm}|p{0.3cm}|p{0.3cm}|p{0.3cm}|p{0.3cm}|p{0.3cm}|p{0.3cm}|p{0.3cm}|}
\hline
0 & n & n & n & n & n & n & n\\
\hline
\end{tabular}
& $<$SLOT\#$>$ & Slot number in the range 0x00 to 0x7F.\\
\end{tabular}

Byte 2:

\begin{tabular}{p{0.4\linewidth} p{0.15\linewidth} p{0.38\linewidth}} 

\begin{tabular}{|p{0.3cm}|p{0.3cm}|p{0.3cm}|p{0.3cm}|p{0.3cm}|p{0.3cm}|p{0.3cm}|p{0.3cm}|}
\hline
0 & d6 & d5 & d4 & d3 & d2 & d1 & d0\\
\hline
\end{tabular}
& $<$STAT1$>$ & Slot status 1.\\
\end{tabular}

Byte 3:

\begin{tabular}{p{0.4\linewidth} p{0.15\linewidth} p{0.38\linewidth}} 

\begin{tabular}{|p{0.3cm}|p{0.3cm}|p{0.3cm}|p{0.3cm}|p{0.3cm}|p{0.3cm}|p{0.3cm}|p{0.3cm}|}
\hline
0 & n & n & n & n & n & n & n\\
\hline
\end{tabular}
& $<$CHK$>$ & Checksum.

\end{tabular}

\underline{Description:}

This function sets the slot's status 1 values.

\underline{Response:} 

None.

\underline{Notes:} 

None.

\rule{15.1cm}{0.4pt}

\input{Chapter1-SVProg}
\input{Chapter1-SWAck}
\newpage
\subsection{SwReq}

\underline{Description:}

Command a turnout controller to a specified state. *** CHECK THIS ***

\underline{Group:}

4-Byte Message

\underline{Opcode:}

OPC\_SW\_REQ

\underline{Type:}

Command

\underline{Encoding:} 

Byte 0:

\begin{tabular}{p{0.4\linewidth} p{0.15\linewidth} p{0.38\linewidth}} 

\begin{tabular}{|p{0.3cm}|p{0.3cm}|p{0.3cm}|p{0.3cm}|p{0.3cm}|p{0.3cm}|p{0.3cm}|p{0.3cm}|}
\hline
1 & 0 & 1 & 1 & 0 & 0 & 0 & 0\\
\hline
\end{tabular}
& 0xB0 & Opcode.\\
\end{tabular}

Byte 1:

\begin{tabular}{p{0.4\linewidth} p{0.15\linewidth} p{0.38\linewidth}} 

\begin{tabular}{|p{0.3cm}|p{0.3cm}|p{0.3cm}|p{0.3cm}|p{0.3cm}|p{0.3cm}|p{0.3cm}|p{0.3cm}|}
\hline
0 & d6 & d5 & d4 & d3 & d2 & d1 & d0\\
\hline
\end{tabular}
& $<$SW1$>$ & Switch address A6 to A0.\\
\end{tabular}

\begin{tabular}{p{0.05\linewidth} p{0.6\linewidth}} 
d6 & A6.\\
d5 & A5.\\
d4 & A4.\\
d3 & A3.\\
d2 & A2.\\
d1 & A1.\\
d0 & A0.\\
\end{tabular}

Byte 2:

\begin{tabular}{p{0.4\linewidth} p{0.15\linewidth} p{0.38\linewidth}} 

\begin{tabular}{|p{0.3cm}|p{0.3cm}|p{0.3cm}|p{0.3cm}|p{0.3cm}|p{0.3cm}|p{0.3cm}|p{0.3cm}|}
\hline
0 & 0 & d5 & d4 & d3 & d2 & d1 & d0\\
\hline
\end{tabular}
& $<$SW2$>$ & Switch address A10 to A7 and switch control bits.\\
\end{tabular}

\begin{tabular}{p{0.05\linewidth} p{0.6\linewidth}} 
d5 & Direction. 1 means closed/green, and 0 means thrown/red.\\
d4 & Output. 1 means on, and 0 means off.\\
d3 & A10.\\
d2 & A9.\\
d1 & A8.\\
d0 & A7.\\
\end{tabular}

Byte 3:

\begin{tabular}{p{0.4\linewidth} p{0.15\linewidth} p{0.38\linewidth}} 

\begin{tabular}{|p{0.3cm}|p{0.3cm}|p{0.3cm}|p{0.3cm}|p{0.3cm}|p{0.3cm}|p{0.3cm}|p{0.3cm}|}
\hline
0 & n & n & n & n & n & n & n\\
\hline
\end{tabular}
& $<$CHK$>$ & Checksum.
\end{tabular}

\underline{Response:} 

\textbf{Ack} if command failed, otherwise no response.

\begin{tabular}{p{0.10\linewidth} p{0.10\linewidth} p{0.40\linewidth}}
\underline{$<$LOPC$>$} & \underline{$<$ACK1$>$} & \underline{Meaning}\\
0x30 & 0x00 & Command failed.\\
\end{tabular}

\underline{Signature:}

Byte 0:

\begin{tabular}{p{0.4\linewidth} p{0.15\linewidth} p{0.38\linewidth}} 

\begin{tabular}{|p{0.3cm}|p{0.3cm}|p{0.3cm}|p{0.3cm}|p{0.3cm}|p{0.3cm}|p{0.3cm}|p{0.3cm}|}
\hline
1 & 0 & 1 & 1 & 0 & 0 & 0 & 0\\
\hline
\end{tabular}
& 0xB0 & \\
\end{tabular}

Byte 2:

\begin{tabular}{p{0.4\linewidth} p{0.15\linewidth} p{0.38\linewidth}} 

\begin{tabular}{|p{0.3cm}|p{0.3cm}|p{0.3cm}|p{0.3cm}|p{0.3cm}|p{0.3cm}|p{0.3cm}|p{0.3cm}|}
\hline
0 & 0 & $\times$ & $\times$ & $\times$ & $\times$ & $\times$ & $\times$\\
\hline
\end{tabular}
& \\
\end{tabular}

\underline{Notes:} 

The on power on the command station sends a sequence of OPC\_SW\_REQ messages with the following values of SW1 and SW2:

\begin{tabular}{l l l}
\underline{SW1} & \underline{SW2} & \underline{Purpose}\\
0x78 & 0x27\\
0x79 & 0x27\\
0x7A & 0x27\\
0x7B & 0x27\\
0x78 & 0x07 & Interrogate all PM4 inputs?\\
0x79 & 0x07 & Interrogate all BDL16 input reports?\\
0x7A & 0x07 & Interrogate all SE8 input reports?\\
0x7B & 0x07 & Interrogate all DS64 input reports.\\
\end{tabular}

\rule{15.1cm}{0.4pt}

\newpage
\section{SwState}\index{SwState}

\underline{Description:}

This \gls{Response} is returned in response to a \textbf{GetSwState} \gls{Command}. 

\underline{Group:}

4-Byte Message

\underline{Opcode:}

OPC\_LONG\_ACK

\underline{Type:}

\gls{Response}

\underline{Encoding:} 

Byte 0:

\begin{tabular}{p{0.4\linewidth} p{0.15\linewidth} p{0.38\linewidth}} 

\begin{tabular}{|p{0.3cm}|p{0.3cm}|p{0.3cm}|p{0.3cm}|p{0.3cm}|p{0.3cm}|p{0.3cm}|p{0.3cm}|}
\hline
1 & 0 & 1 & 1 & 0 & 1 & 0 & 0\\
\hline
\end{tabular}
& 0xB4 & Opcode.\\
\end{tabular}

Byte 1:

\begin{tabular}{p{0.4\linewidth} p{0.15\linewidth} p{0.38\linewidth}} 

\begin{tabular}{|p{0.3cm}|p{0.3cm}|p{0.3cm}|p{0.3cm}|p{0.3cm}|p{0.3cm}|p{0.3cm}|p{0.3cm}|}
\hline
0 & 0 & 1 & 1 & 1 & 1 & 0 & 0\\
\hline
\end{tabular}
& 0x3C & \\
\end{tabular}

Byte 2:

\begin{tabular}{p{0.4\linewidth} p{0.15\linewidth} p{0.38\linewidth}} 

\begin{tabular}{|p{0.3cm}|p{0.3cm}|p{0.3cm}|p{0.3cm}|p{0.3cm}|p{0.3cm}|p{0.3cm}|p{0.3cm}|}
\hline
0 & 0 & d5 & d4 & 0 & 0 & 0 & 0\\
\hline
\end{tabular}
\end{tabular}

\begin{tabular}{l l}
\underline{Bit} & \underline{Meaning}\\
d5 & Switch state: 1 means closed/green and 0 means thrown/red\\
d4 & Output state: 1 means on and 0 means off.
\end{tabular}

Byte 3:

\begin{tabular}{p{0.4\linewidth} p{0.15\linewidth} p{0.38\linewidth}} 

\begin{tabular}{|p{0.3cm}|p{0.3cm}|p{0.3cm}|p{0.3cm}|p{0.3cm}|p{0.3cm}|p{0.3cm}|p{0.3cm}|}
\hline
0 & n & n & n & n & n & n & n\\
\hline
\end{tabular}
& $<$CHK$>$ & Checksum.
\end{tabular}

\underline{Response:} 

None.

\underline{Signature:}

Byte 0:

\begin{tabular}{p{0.4\linewidth} p{0.38\linewidth}} 

\begin{tabular}{|p{0.3cm}|p{0.3cm}|p{0.3cm}|p{0.3cm}|p{0.3cm}|p{0.3cm}|p{0.3cm}|p{0.3cm}|}
\hline
1 & 0 & 1 & 1 & 0 & 1 & 0 & 0\\
\hline
\end{tabular}
& 0xB4\\
\end{tabular}

Byte 1:

\begin{tabular}{p{0.4\linewidth} p{0.15\linewidth} p{0.38\linewidth}} 

\begin{tabular}{|p{0.3cm}|p{0.3cm}|p{0.3cm}|p{0.3cm}|p{0.3cm}|p{0.3cm}|p{0.3cm}|p{0.3cm}|}
\hline
0 & 0 & 1 & 1 & 1 & 1 & 0 & 0\\
\hline
\end{tabular}
& 0x3C & \\
\end{tabular}

\underline{Notes:} 

None.

\rule{15.1cm}{0.4pt}

\newpage
\subsection{TransRep}\index{TransRep}

\underline{Description:}

Transponder input report.

\underline{Group:}

6-Byte Message

\underline{Opcode:}

OPC\_TRANS\_REP

\underline{Type:}

Broadcast

\underline{Encoding:} 

Byte 0:

\begin{tabular}{p{0.4\linewidth} p{0.15\linewidth} p{0.38\linewidth}} 

\begin{tabular}{|p{0.3cm}|p{0.3cm}|p{0.3cm}|p{0.3cm}|p{0.3cm}|p{0.3cm}|p{0.3cm}|p{0.3cm}|}
\hline
1 & 1 & 0 & 1 & 0 & 0 & 0 & 0\\
\hline
\end{tabular}
& 0xD0 & Opcode.\\
\end{tabular}

Byte 1:

\begin{tabular}{p{0.4\linewidth} p{0.15\linewidth} p{0.38\linewidth}} 

\begin{tabular}{|p{0.3cm}|p{0.3cm}|p{0.3cm}|p{0.3cm}|p{0.3cm}|p{0.3cm}|p{0.3cm}|p{0.3cm}|}
\hline
0 & n & n & n & n & n & n & n\\
\hline
\end{tabular}
&  & A value of 0x20 means the positive detection of a transponder, 0x00 means no longer detected.\\
\end{tabular}

Byte 2:

\begin{tabular}{p{0.4\linewidth} p{0.15\linewidth} p{0.38\linewidth}} 

\begin{tabular}{|p{0.3cm}|p{0.3cm}|p{0.3cm}|p{0.3cm}|p{0.3cm}|p{0.3cm}|p{0.3cm}|p{0.3cm}|}
\hline
0 & 0 & 0 & 0 & n & n & n & n\\
\hline
\end{tabular}
& $<$ZONE\#$>$ & Zone indicator (0x0 = A, 0x2 = B, 0x4 = C, 0x6 = D).\\
\end{tabular}

Byte 3:

\begin{tabular}{p{0.4\linewidth} p{0.15\linewidth} p{0.38\linewidth}} 

\begin{tabular}{|p{0.3cm}|p{0.3cm}|p{0.3cm}|p{0.3cm}|p{0.3cm}|p{0.3cm}|p{0.3cm}|p{0.3cm}|}
\hline
0 & n & n & n & n & n & n & n\\
\hline
\end{tabular}
& $<$ADR$>$ & Locomotive address low bits.\\
\end{tabular}

Byte 4:

\begin{tabular}{p{0.4\linewidth} p{0.15\linewidth} p{0.38\linewidth}} 

\begin{tabular}{|p{0.3cm}|p{0.3cm}|p{0.3cm}|p{0.3cm}|p{0.3cm}|p{0.3cm}|p{0.3cm}|p{0.3cm}|}
\hline
0 & n & n & n & n & n & n & n\\
\hline
\end{tabular}
& $<$ADR2$>$ & Locomotive address high bits.\\
\end{tabular}

Byte 5:

\begin{tabular}{p{0.4\linewidth} p{0.15\linewidth} p{0.38\linewidth}} 

\begin{tabular}{|p{0.3cm}|p{0.3cm}|p{0.3cm}|p{0.3cm}|p{0.3cm}|p{0.3cm}|p{0.3cm}|p{0.3cm}|}
\hline
0 & n & n & n & n & n & n & n\\
\hline
\end{tabular}
& $<$CHK$>$ & Checksum.\\
\end{tabular}

\underline{Response:} 

None.

\underline{Signature:}

Byte 0:

\begin{tabular}{p{0.4\linewidth} p{0.15\linewidth} p{0.38\linewidth}} 

\begin{tabular}{|p{0.3cm}|p{0.3cm}|p{0.3cm}|p{0.3cm}|p{0.3cm}|p{0.3cm}|p{0.3cm}|p{0.3cm}|}
\hline
1 & 1 & 0 & 1 & 0 & 0 & 0 & 0\\
\hline
\end{tabular}
& 0xD0.\\
\end{tabular}

*** THERE SHOULD BE MORE ***

\underline{Notes:} 

None.

\rule{15.1cm}{0.4pt}

\newpage
\subsection{OPC\_UNLINK\_SLOTS}
\underline{Operation:} Unlink slots.

\underline{Group:} \hspace{0.5cm} Variable-Byte Message

\underline{Direction:} \hspace{0.05cm} $\rightarrow$ Command Station  

\underline{Encoding:} 

Byte 0:

\begin{tabular}{p{0.4\linewidth} p{0.15\linewidth} p{0.38\linewidth}} 

\begin{tabular}{|p{0.3cm}|p{0.3cm}|p{0.3cm}|p{0.3cm}|p{0.3cm}|p{0.3cm}|p{0.3cm}|p{0.3cm}|}
\hline
1 & 0 & 1 & 1 & 1 & 0 & 0 & 0\\
\hline
\end{tabular}
& 0xB8 & Opcode.\\
\end{tabular}

Byte 1:

\begin{tabular}{p{0.4\linewidth} p{0.15\linewidth} p{0.38\linewidth}} 

\begin{tabular}{|p{0.3cm}|p{0.3cm}|p{0.3cm}|p{0.3cm}|p{0.3cm}|p{0.3cm}|p{0.3cm}|p{0.3cm}|}
\hline
0 & n & n & n & n & n & n & n\\
\hline
\end{tabular}
& $<$SL1$>$ & Slot number in the range 0x00 to 0x7F.\\
\end{tabular}

Byte 2:

\begin{tabular}{p{0.4\linewidth} p{0.15\linewidth} p{0.38\linewidth}} 

\begin{tabular}{|p{0.3cm}|p{0.3cm}|p{0.3cm}|p{0.3cm}|p{0.3cm}|p{0.3cm}|p{0.3cm}|p{0.3cm}|}
\hline
0 & n & n & n & n & n & n & n\\
\hline
\end{tabular}
& $<$SL2$>$ & Slot number in the range 0x00 to 0x7F.\\
\end{tabular}

Byte 3:

\begin{tabular}{p{0.4\linewidth} p{0.15\linewidth} p{0.38\linewidth}} 

\begin{tabular}{|p{0.3cm}|p{0.3cm}|p{0.3cm}|p{0.3cm}|p{0.3cm}|p{0.3cm}|p{0.3cm}|p{0.3cm}|}
\hline
0 & n & n & n & n & n & n & n\\
\hline
\end{tabular}
& $<$CHK$>$ & Checksum.\\
\end{tabular}

\underline{Description:}

This command unlinks slot SL1 from slot SL2.

\underline{Response:} 

Returns OPC\_SL\_RD\_DATA or OPC\_LONG\_ACK.

\underline{Notes:} 

None.

\rule{15.1cm}{0.4pt}



--------------------

PR4 Interface Status Message

\begin{verbatim}

PR4 #1

<D0> 0xe5 OPCODE
<D1> 0x10 LENGTH
<D2> 0x22 SRC
<D3> 0x22 DSTL
<D4> 0x01 DSTH
<D5> 0x00 PXCT1 <- I would have expected b4 = 1
<D6> 0x08 Serial Number Low Byte
<D7> 0x07 Serial Number High Byte - Actual serial number 0x0788
<D8> 0x16 
<D9> 0x00 
<D10> 0x00 PXCT2
<D11> 0x00 
<D12> 0x00 
<D13> 0x00 
<D14> 0x24 Product Code for PR4
<D15> 0x36  CHSUM

PR4 #2

<D0> 0xe5 OPCODE OPC_PEER_XFER
<D1> 0x10 LENGTH
<D2> 0x22 SRC
<D3> 0x22 DSTL
<D4> 0x01 DSTH
<D5> 0x00 PXCT1 
<D6> 0x57 Serial Number Low Byte
<D7> 0x13 Serial Number High Byte - Actual serial number 0x1357
<D8> 0x16 
<D9> 0x00 
<D10> 0x00 PXCT2
<D11> 0x00 
<D12> 0x00 
<D13> 0x00 
<D14> 0x24 Product Code for PR4
<D15> 0x7d CHKSUM

DCS240

<D0> 0xe5 OPCODE
<D1> 0x10 Length
<D2> 0x22 SRC
<D3> 0x22 DSTL
<D4> 0x01 DSTH
<D5> 0x00 PXCT1 <- I would have expected b4 to be 1
<D6> 0x2b Serial Number Low Byte
<D7> 0x0a  Serial Number High Byte - Actual serial number 0x0aab
<D8> 0x14 
<D9> 0x00 
<D10> 0x00 PXCT2
<D11> 0x01 Hardware Version?
<D12> 0x03 Software Version
<D13> 0x01 Hardware Version?
<D14> 0x1c Product Code for DCS240
<D15> 0x21

\end{verbatim}
