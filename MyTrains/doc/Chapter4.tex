% Activate the following line by filling in the right side. If for example the name of the root file is Main.tex, write
% "...root = Main.tex" if the chapter file is in the same directory, and "...root = ../Main.tex" if the chapter is in a subdirectory.
 
%!TEX root =  

\chapter[Configuration Variables]{Programming Configuration Variables (CVs)}

\section{Introduction}

The decoders installed in your locomotives provide you with the ability to create a more realistic operating experience through the configuration variables (CVs for short). The network protocol supports configuration of up to
1024 CVs.

It is a good idea to run your decoders with the default CV values that come pre-programmed in your decoders until you get used to the performance characteristic and how they work on your layout. Once you are comfortable with running the trains, then you can begin customizing locomotive characteristics.

Each CV (configuration variable) controls a specific characteristic of the decoder, which in turn controls how the locomotive performs. See your decoder manual for a list of the most commonly used CVs and their meanings.
Each decoder comes pre-programmed from the factory with the default settings outlined in your decoder manual. You can change your decoder’s performance characteristics by changing the CV values entered in the CVs you want to change. Each of these CVs can be set up when your command station is in the programming mode. The CVs are remembered in the decoder until it is reprogrammed to with a different CV value. Please refer to your mobile decoder manual for a complete listing of the CVs supported by each decoder.

Programming decoder CVs is usually done on an isolated programming track.

There are four programming modes:

\begin{itemize}
\item Paged mode
\item Physical register mode
\item Direct mode
\item Operations mode
\end{itemize}

\subsection{Paged Mode Programming}

\subsection{Physical Register Programming}
Physical Register Mode can only read CV01-CV08. You should not rely on values in the display for CVs above 08 when reading back in physical register mode.

\subsection{Direct Mode Programming}
This is the preferred programming mode.

\subsection{Operations Mode Programming}
Operations mode programming lets you program CVs in locomotives equipped with Extended Packet Format decoders while they are on the mainline. A typical use for Ops mode programming would be to change the acceleration rate (CV03) or the deceleration rate (CV04) of your locomotives to simulate the weight and braking capability of the train to compensate for changing the number of cars or power units on a train.

Operations Mode read back can only be used with decoders that are capable of operations mode read back when there is a device attached to the network that supports operations mode read back. Digitrax transponding decoders and the DCS210 or DCS240 command stations would allow operations mode read back.

\section{Programming Mobile Decoder Addresses}

Be sure that only the loco you want to program is on the programming track. If you are using operations mode programming, the loco you want to program can be anywhere on the layout but it must have a decoder that is capable of operations mode programming installed.

There are two addressing methods - short and long. The short addresses can take a value between 0 and 127, and long addresses a value between 128 and 9983. The bit 5 of mobile decoder's configuration register (CV29) determines what addressing method is used. If bit 5 is set to 1 then long addresses are used, and when bit 5 is 0 then short addresses are used. Short addresses are stored in CV1, and long addresses in CV17 and CV18. The address values stored in CV17 and CV18 are not the high and low bytes of the address value. The CV17 and CV18 values must be calculated from the address value as follows:

TEMP = address + 49152

CV18 = TEMP \& 0xFF

CV17 = TEMP $>>$ 8

Example:

address = 4007

TEMP = 49152 + 4007 = 53159 = 0xCFA7

CV18 = 0xA7 = 167

CV17 = 0xCF = 207

read cv

\begin{verbatim}

Read CV
unknown
   65830.9ms 
<D0> 0xef 0b11101111 <- OPC_PROG
<D1> 0x0e 0b00001110 <- Message Length
<D2> 0x7c 0b01111100 <- Special programming slot number
<D3> 0x2b 0b00101011 <- PCMD

d7 0
d6 0 - read
d5 1 - byte mode
d4 0 - TV1
d3 1 - TV0 
d2 0 - service mode on programming track
d1 1 - unknown
d0 1  - unknown

Direct mode byte read on service track

<D4> 0x00 0b00000000 - 0x00
<D5> 0x00 0b00000000 - HOPSA - Ops mode programming - 7 high address bits of Loco to program, 0x00 if service mode
<D6> 0x0e 0b00001110 - LOPSA - Ops Mode programming - 7 low address bits of loco to program, 0x00 if service mode (however 0e is the loco address)
<D7> 0x00 0b00000000 - TRK - normal track status for command station - this doesn't look right set to 0x00 for send
<D8> 0x00 0b00000000 - CVH
<D9> 0x00 0b00000000 - CVL
<D10> 0x0f 0b00001111 - DATA
<D11> 0x6d 0b01101101 - Throttle serial number
<D12> 0x52 0b01010010 - Throttle serial number
<D13> 0x77 0b01110111 

response
    1722.5ms 
<D0> 0xe7 0b11100111 <- Opcode
<D1> 0x0e 0b00001110 <- length
<D2> 0x7c 0b01111100 <- Programming slot
<D3> 0x2b 0b00101011 <- PCMD
<D4> 0x00 0b00000000 <- PSTAT - success
<D5> 0x00 0b00000000 <- HOPSA
<D6> 0x02 0b00000010 <- LOPSA should be 0
<D7> 0x47 0b01000111 <- TRK
<D8> 0x02 0b00000010 <- CVH : 0, 0, CV9, CV8, 0, 0, D7, CV7
<D9> 0x04 0b00000100 <- CVL - CV5
<D10> 0x16 0b00010110  <- low 7 bits of value
<D11> 0x6d 0b01101101 <- SN
<D12> 0x52 0b01010010  <- SN
<D13> 0x2b 0b00101011 <- CHK
\end{verbatim}

value displayed is 150 10010110

PCMD

\begin{tabular}{l l}
d7 & 0\\
d6 & 1 = write, 0 = read\\
d5 & 1 = byte operation, 0 = bit operation (if possible)\\
d4 & TV1\\
d3 & TV0\\
d2 & 1 = Ops mode on mainlines, 0 = service mode on programming track\\
d1 & 0 - reserved\\
d0 & 0 - reserved\\
\end{tabular}

\begin{tabular}{l l l l l}
\underline{Byte Mode} & \underline{Ops Mode} & \underline{TV1} & \underline{TV0} & \underline{Meaning} \\
1 & 0 & 0 & 0 & Paged mode byte read/write on service track\\
1 & 0 & 0 & 1 & Direct mode byte read/write on service track\\
0 & 0 & 0 & 1 & Direct mode bit read/write on service track\\
$\times$ & 0 & 1 & 0 & Physical register byte read/write on service track\\
$\times$ & 0 & 1 & 1 & Service track reserved function\\
1 & 1 & 0 & 0 & Ops mode byte program no feedback\\
1 & 1 & 0 & 1 & Ops mode byte program with feedback\\
0 & 1 & 0 & 0 & Ops mode bit program no feedback\\
0 & 1 & 0 & 1 & Ops mode bit program with feedback\\
\end{tabular}

\begin{verbatim}
ack
<D0> 0xb4 0b10110100 
<D1> 0x6f 0b01101111 
<D2> 0x01 0b00000001 
<D3> 0x25 0b00100101 

unknown
    1731.6ms <D0> 0xe7 0b11100111 
<D1> 0x0e 0b00001110 
<D2> 0x7c 0b01111100 
<D3> 0x2b 0b00101011 
<D4> 0x00 0b00000000 
<D5> 0x00 0b00000000 
<D6> 0x02 0b00000010 
<D7> 0x47 0b01000111 
<D8> 0x00 0b00000000 
<D9> 0x00 0b00000000 
<D10> 0x0f 0b00001111 
<D11> 0x6d 0b01101101 
<D12> 0x52 0b01010010 
<D13> 0x34 0b00110100 

ack
      10.6ms <D0> 0xb4 0b10110100 
<D1> 0x3b 0b00111011 
<D2> 0x00 0b00000000 
<D3> 0x70 0b01110000 

Read CV 2
unknown
    6772.5ms <D0> 0xef 0b11101111 
<D1> 0x0e 0b00001110 
<D2> 0x7c 0b01111100 
<D3> 0x2b 0b00101011 
<D4> 0x00 0b00000000 
<D5> 0x00 0b00000000 
<D6> 0x0e 0b00001110 
<D7> 0x00 0b00000000 
<D8> 0x00 0b00000000 
<D9> 0x01 0b00000001 
<D10> 0x0f 0b00001111 
<D11> 0x6d 0b01101101 
<D12> 0x52 0b01010010 
<D13> 0x76 0b01110110 

ack
      15.5ms <D0> 0xb4 0b10110100 
<D1> 0x6f 0b01101111 
<D2> 0x01 0b00000001 
<D3> 0x25 0b00100101 

unknown
    1720.8ms <D0> 0xe7 0b11100111 
<D1> 0x0e 0b00001110 
<D2> 0x7c 0b01111100 
<D3> 0x2b 0b00101011 
<D4> 0x00 0b00000000 
<D5> 0x00 0b00000000 
<D6> 0x02 0b00000010 
<D7> 0x47 0b01000111 
<D8> 0x00 0b00000000 
<D9> 0x01 0b00000001 
<D10> 0x07 0b00000111 
<D11> 0x6d 0b01101101 
<D12> 0x52 0b01010010 
<D13> 0x3d 0b00111101 

------------------- CV2
unknown
   11836.0ms <D0> 0xef 0b11101111 
<D1> 0x0e 0b00001110 
<D2> 0x7c 0b01111100 
<D3> 0x2b 0b00101011 
<D4> 0x00 0b00000000 
<D5> 0x00 0b00000000 
<D6> 0x0e 0b00001110 
<D7> 0x00 0b00000000 
<D8> 0x00 0b00000000 
<D9> 0x01 0b00000001 
<D10> 0x07 0b00000111 
<D11> 0x6d 0b01101101 
<D12> 0x52 0b01010010 
<D13> 0x7e 0b01111110 

ack
       6.1ms <D0> 0xb4 0b10110100 
<D1> 0x6f 0b01101111 
<D2> 0x01 0b00000001 
<D3> 0x25 0b00100101 

unknown
    1730.2ms <D0> 0xe7 0b11100111 
<D1> 0x0e 0b00001110 
<D2> 0x7c 0b01111100 
<D3> 0x2b 0b00101011 
<D4> 0x00 0b00000000 
<D5> 0x00 0b00000000 
<D6> 0x02 0b00000010 
<D7> 0x47 0b01000111 
<D8> 0x00 0b00000000 
<D9> 0x01 0b00000001 
<D10> 0x07 0b00000111 
<D11> 0x6d 0b01101101 
<D12> 0x52 0b01010010 
<D13> 0x3d 0b00111101 

<- failure nothing on prog track

<D0> 0xe7 0b11100111 <- opcode
<D1> 0x0e 0b00001110 <- length
<D2> 0x7c 0b01111100  <- prog slot
<D3> 0x2b 0b00101011 <- PCMD
<D4> 0x01 0b00000001 <- PSTAT
<D5> 0x00 0b00000000 
<D6> 0x01 0b00000001 
<D7> 0x47 0b01000111 
<D8> 0x02 0b00000010 
<D9> 0x04 0b00000100 
<D10> 0x16 0b00010110 
<D11> 0x6d 0b01101101 
<D12> 0x52 0b01010010 
<D13> 0x29 0b00101001 

PSTAT

d7 0 - reserved
d6 0 - reserved
d5 0 - reserved
d4 0 - reserved
d3 1 = user aborted command
d2 1 = failed to detect read compare ack from decoder
d1 1 = no write ack from decoder
d0 - 1 = service mode programming track empty - no decoder detected


-----> write 150 to CV5

unknown
    7846.9ms 
    
<D0> 0xef 0b11101111 
<D1> 0x0e 0b00001110 
<D2> 0x7c 0b01111100 
<D3> 0x6b 0b01101011 
<D4> 0x00 0b00000000 
<D5> 0x00 0b00000000 
<D6> 0x0e 0b00001110 
<D7> 0x00 0b00000000 
<D8> 0x02 0b00000010 
<D9> 0x04 0b00000100 
<D10> 0x16 0b00010110 
<D11> 0x6d 0b01101101 
<D12> 0x52 0b01010010 
<D13> 0x28 0b00101000 

ack
       4.6ms 
<D0> 0xb4 0b10110100 
<D1> 0x6f 0b01101111 
<D2> 0x01 0b00000001 
<D3> 0x25 0b00100101 

unknown
     894.9ms 
<D0> 0xe7 0b11100111 
<D1> 0x0e 0b00001110 
<D2> 0x7c 0b01111100 
<D3> 0x6b 0b01101011 
<D4> 0x00 0b00000000 
<D5> 0x00 0b00000000 
<D6> 0x02 0b00000010 
<D7> 0x47 0b01000111 
<D8> 0x02 0b00000010 
<D9> 0x04 0b00000100 
<D10> 0x16 0b00010110 
<D11> 0x6d 0b01101101 
<D12> 0x52 0b01010010 
<D13> 0x6b 0b01101011 

---> write 150 to CV5 nothing on prog track

unknown
   11349.0ms <D0> 0xef 0b11101111 
<D1> 0x0e 0b00001110 
<D2> 0x7c 0b01111100 
<D3> 0x6b 0b01101011 
<D4> 0x00 0b00000000 
<D5> 0x00 0b00000000 
<D6> 0x0e 0b00001110 
<D7> 0x00 0b00000000 
<D8> 0x02 0b00000010 
<D9> 0x04 0b00000100 
<D10> 0x16 0b00010110 
<D11> 0x6d 0b01101101 
<D12> 0x52 0b01010010 
<D13> 0x28 0b00101000 

ack
       6.0ms <D0> 0xb4 0b10110100 
<D1> 0x6f 0b01101111 
<D2> 0x01 0b00000001 
<D3> 0x25 0b00100101 

unknown
     723.9ms <D0> 0xe7 0b11100111 
<D1> 0x0e 0b00001110 
<D2> 0x7c 0b01111100 
<D3> 0x6b 0b01101011 
<D4> 0x01 0b00000001 
<D5> 0x00 0b00000000 
<D6> 0x02 0b00000010 
<D7> 0x47 0b01000111 
<D8> 0x02 0b00000010 
<D9> 0x04 0b00000100 
<D10> 0x16 0b00010110 
<D11> 0x6d 0b01101101 
<D12> 0x52 0b01010010 
<D13> 0x6a 0b01101010 

\end{verbatim}



-----------------------

\section{List of all supported CVs}

\begin{tabular}{p{0.1\linewidth} p{0.15\linewidth} p{0.5\linewidth} p{0.1\linewidth} p{0.1\linewidth}} 
\underline{CV} & \underline{Name} &  \underline{Description} & \underline{Range} & \underline{Value}\\
1 & Loco address & Address of engine (For Multiprotocol decoders: Range 1-255 for Motorola) & 1 - 127  & 3\\
2 & Start voltage & Sets the minimum speed of the engine & 1 - 255  & 3\\
3 & Acceleration & This value multiplied by 0.25 is the time from stop to maximum speed. For LokSound 5 DCC: The unit is 0.896 seconds & 0 - 255 & 28\\
4 & Deceleration & This value multiplied by 0.25 is the time from maximum speed to stop. For LokSound 5 DCC: The unit is 0.896 seconds & 0 - 255 & 21\\
5 & Maximum speed & Maximum speed of the engine & 0 - 255 & 255\\
6 & Medium speed & Medium speed of the engine. Use only if 3-point speed table is enabled. For LokSound 5 DCC only.\\
7 & Version number & Internal software version of decoder & - & -\\
8 & Manufacturer‘s ID & Manufacturers‘s ID ESU - Writing value 8 in this CV triggers a reset to factory default values & 151 & -\\
9 & Motor PWM Frequenz & Motor PWM frequency as a multiple of 1000 Hz. & 10 - 50 & 40\\
13 & Analog Modus F1-F8 & Status of functions F1 to F8 in analogue mode (see chapter 12.7) & 0-255 & 1\\
14 & Analog Modus FL, F9-F15 & Status of function F0, F9 to F12 in analogue mode (see chapter 12.7) & 0-63 & 1\\
15 \& 16 & Decoder Lock & Decoder-Lock Function according to NMRA. For details please see: http://www.nmra.org/standards/DCC/WGpublic/0305051/0305051.html & 0 - 255 & 0\\
17 \& 18 & Long address of the loco & Long address of engine (see chapter 9.2) & 128 - 9999 & 192\\
19 & Consist Address & Additional address for consist operation. Value 0 or 128 means: consist address is disabled. 1 – 127 consist address active, normal direction. 129 – 255 consist address active reverse direction. & 0-255 & 0\\
21 & Consist Mode F1-F8 & Status of functions F1 to F8 in Consist mode. Meaning of the bits as in CV 13 & 0-255 & 0\\
22 & Consist Mode FL, F9-F12 & Status of functions FL, F9 to F12 in Consist mode. Meaning of the bits as in CV 14. & 0-63 & 0\\
23 & Adjust Acceleration & Factor for adjusting Acceleration CV 3. Values from 0 to 127 are added to CV 3. If the values are to be subtracted, additionally set bit 7 (value 128). The unit is 0.896 seconds. & 0 - 127 & 0\\
\end{tabular}
\begin{tabular}{p{0.1\linewidth} p{0.15\linewidth} p{0.5\linewidth} p{0.1\linewidth} p{0.1\linewidth}} 
\underline{CV} & \underline{Name} &  \underline{Description} & \underline{Range} & \underline{Value}\\
24 & Adjust Deceleration & Factor for adjusting the deceleration CV 4. Values from 0 to 127 are added to CV 3. If the values are to be subtracted, additionally set bit 7 (value 128). The unit is 0.896 seconds. & 0 - 127 & 0\\
27 & Brake mode & Allowed (enabled) Brake modes 28 & \\
& \begin{tabular}{ l l l }
Bit & Function & Value\\
0 & ABC braking, voltage higher on the right hand side & 1\\
1 & ABC braking, voltage higher on the left hand side  & 2\\
2 & ZIMO® HLU brakes active & 4\\
3 & Brake on DC, if polarity against driving direction & 8\\
4 & Brake on DC, if polarity like driving direction & 16\\
5 & Selectrix brake diode, brakes if polarity is against driving direction & 32\\
6 & Selectrix brake diode, rakes if polarity is like driving direction & 64\\
7 & Loco brakes with constant brake distance if Speed=0 & 128\\
\end{tabular}\\
& - & 28\\
28 & RailCom® Configuration & Settings for RailCom® & 131\\ 
& \begin{tabular}{l l l}
Bit & Function & Value\\
0 & Channel 1 Address broadcast enabled & 1\\
1 & Data transmission allowed on Channel & 2\\
7 & RailCom® Plus automatic loco recognition active & 128\\
\end{tabular} \\
29 & Configuration register & This register contains important information, some of which are only relevant for DCC operation. \\

& \begin{tabular}{l l l}
Bit & Function & Value\\
0 & Normal direction of travel & 0\\
& Reversed direction of travel & 1\\
1 & 14 speed steps DCC & 0\\
& 28 or 128 speed steps DCC & 2\\
2 & Disable analog operation & 0\\
& Enable analog operation & 4\\
3 & Disable RailCom® & 0\\
& Enable RailCom® & 8\\
4 & Speed curve through CV 2, 5, 6 (LokSound 5 DCC ONLY). & 0\\
& Speed curve through CV 67 - 94 (Multiprotocol) & 16\\
5 & Short addresses (CV 1) in DCC mode & 0\\
& Long addresses (CV 17 + 18) in DCC mode & 32\\
\end{tabular} & 12\\

\end{tabular}
\begin{tabular}{p{0.1\linewidth} p{0.15\linewidth} p{0.5\linewidth} p{0.1\linewidth} p{0.1\linewidth}} 
\underline{CV} & \underline{Name} &  \underline{Description} & \underline{Range} & \underline{Value}\\

31 & Index Register H & Selection page for CV257-512. For LokSound 5 usually set to 16 & 16 & 16\\
32 & Index Register L & Selection page for CV257-512 & 0 - 16 & 0\\
47 & Protocol selection & Which protocols are active. Please see chapter 9.5. & 0 - 255 & 13\\
& & \begin{tabular}{l l l}
Bit & Function & Value\\
0 & DCC protocol active & 1\\
1 & M4 protocol active (Not for LokSound 5 DCC) & 2\\
2 & Motorola® protocol active (Not for LokSound 5 DCC) & 4\\
3 & Selectrix® protocol active (Not for LokSound 5 DCC) & 8\\
\end{tabular}\\

49 & Extended Configuration \#1 & & 0-255 & 19\\

&& \begin{tabular}{l l l}
Bit & Function & Value\\
0 & Disable Load control (Back-EMF) & 0\\
& Enable Load control (Back-EMF) & 1\\
1 & Reserved & 2\\
2 & Reserved & 4\\
3 & Märklin® Consecutive addresses, ``low"-Bit & 0, 8\\
4 & Automatic DCC speed step detection \\
& Disable DCC speed step detection & 0\\
& Enable DCC speed step detection & 16\\
5 & LGB® function button mode\\
& Disable LGB® function button mode & 0\\
& Enable LGB® function button mode & 32\\
6 & Reserved & 64\\
7 & Märklin® Consecutive addresses, ``High"-Bit. Please consider chapter 9.3.1. for explanation of Bit 3, 7 & 0, 128\\
\end{tabular}\\

50 & Analogue mode Selection of allowed analogue modes & 0 - 3 & 3\\
& \begin{tabular}{l l l}
Bit & Function & Value\\
0 & AC Analogue Mode\\
& Disable AC Analog Mode & 0\\
& Enable AC Analog Mode & 1\\
1 & DC Analogue mode\\
& Disable DC Analogue mode & 0\\
& Enable DC Analogue Mode & 2\\
2 & QSI Quantum Engineer DC Support \\
& Disable QSI Quantum Engineer Support & 0\\
& Enable QSI Quantum Engineer Support & 4\\
\end{tabular}\\

\end{tabular}
\begin{tabular}{p{0.1\linewidth} p{0.15\linewidth} p{0.5\linewidth} p{0.1\linewidth} p{0.1\linewidth}} 
\underline{CV} & \underline{Name} &  \underline{Description} & \underline{Range} & \underline{Value}\\

51 & K Slow Cutoff & Inernal Speedstep, until K Slow is active & 0 - 255 & 10\\
52 & BEMF Param. K Slow ``K" -& Portion of the PI-Controller valid for lower speed steps & 0 - 255 & 10\\
53 & Control Reference voltage & Defines the Back EMF voltage, which the motor should generate at maximum speed. The higher the efficiency of the motor, the higher this value may be set. If the engine does not reach maximum speed, reduce this parameter & 0 - 255 & 130\\
54 & Load control Parameter K & K–component of the internal PI-controller. Defines the effect of load control. The higher the value, the stronger the effect of Back EMF control. & 0 - 255 & 50\\
55 & Load control Parameter I  & I–component of the internal PI-controller. Defines the momentum (inertia) of the motor. The higher the momentum of the motor (large flywheel or bigger motor), the lower this value has to be set. & 0 - 255 & 100\\
56 & BEMF Influence at VMin & 0-100\%. Defines the ``Strengh" of the BEMF at minimum speed step & 1 - 255 & 255\\
57 & Steam chuff synchronisation \#1 & Defines the steam chuff synchronisation. See chapter 13.3. & 1 - 255 & 30\\
58 & Steam chuff synchronisation \#2 & Defines the steam chuff synchronisation. See chapter 13.3. & 1 - 255 & 20\\
63 & Sound volume ``Master" & Master volume for all sounds. & 0 - 192 & 128\\
64 & Brake sound threshold ``Brake On" & If the actual loco speed step is smaller than or equals the value indicated here, the brake sound is triggered. Compare chapter 13.4. & 0 - 255 & 60\\
\end{tabular}
\begin{tabular}{p{0.1\linewidth} p{0.15\linewidth} p{0.5\linewidth} p{0.1\linewidth} p{0.1\linewidth}} 
\underline{CV} & \underline{Name} &  \underline{Description} & \underline{Range} & \underline{Value}\\


65 & Brake sound threshold ``Brake Off" & If the actual loco speed step is smaller than the one indicated here (up to 255), the brake sound will be switched off again. Compare chapter 13.4. & 0 - 255 & 7\\
66 & Forward Trimm & Divided by 128 is the factor used to multiply the motor voltage when driving forward. The value 0 deactivates the trim. & 0 - 255 & 128\\
67-94 & Speed table & Defines motor voltage for speed steps. The values ``in between" will be interpolated. & 0 - 255 & -\\
95 & Reverse Trimm & Divided by 128 is the factor used to multiply the motor voltage when driving backwards. Value 0 deactivates the trim. & 0 - 255 & 128\\

101 & Shunting Mode Trimm & Divided by 128, this gives the factor by which the motor voltage is multiplied when the shunting gear is active. See section 10.1.2. & 0 - 128 & 64\\
102 & Brake Mode Exit Delay & Time as a multiple of 16 milliseconds that must pass before a detected braking distance is left again. See section 10.4.6. & 0 - 255 & 12\\
103 & Load adjustment ``Optional Load" & Divided by 128, this gives the factor that changes CV3, CV4 and the sound when ``Optional Load" is active. See section 10.7. & 0 - 255 & 0\\
104 & Load adjustment ``Primary Load" &  Divided by 128, this gives the factor that changes CV3, CV4 and the sound when ``Primary Load" is active. See section 10.7. & 0 - 255 & 255\\
105 & User CV \#1 & Free CV. Here you are able to save what ever you want. & 0 - 255 & 0\\
106 & User CV \#2 & Free CV. Here you are able to save what ever you want. & 0 - 255 & 0\\

111 & Gearbox backlash & Time as a multiple of 16 mS, for which the motor runs at minimum speed after reversing the direction to prevent gear box jerking. & 0 - 255 & 0\\
112 & Frequency for Flashing light effects & Flashing frequency for Strobe lighting effects. Multiple of 0.065536 seconds. See section 12.5.4. & 0 - 255 & 20\\
113 & Power Fail Bypass & The time that the decoder bridges via the PowerPack after an interruption of voltage. Unit: A multiple of 0.032768 sec. See section 6.12.2. & 0 - 255 & 32\\
116 & Slow speed BEMF Sampling period & Frequency of BEMF measurement in 0.1 milliseconds at speed step 1 & 50 - 200 & 50\\
117 & Full speed BEMF Sampling period & Frequency of BEMF measurement in 0.1 milliseconds at speed step 255 & 50 - 200 & 150\\
\end{tabular}
\begin{tabular}{p{0.1\linewidth} p{0.15\linewidth} p{0.5\linewidth} p{0.1\linewidth} p{0.1\linewidth}} 
\underline{CV} & \underline{Name} &  \underline{Description} & \underline{Range} & \underline{Value}\\



118 & Slow speed BEMF & Measurement gap length VMin Length of the BEMF measuring gap in 0.1 milliseconds at speed step 1 & 10 - 20 & 150\\
119 & Full speed BEMF & Measurement gap length Vmax Length of the BEMF measuring gap in 0.1 milliseconds at speed step 255 & 10 - 20 & 15\\
123 & ABC Mode ``Slow drive" & Speed which is valid in the slow driving section during ABC braking. & 0 & -\\ 

124 & Extended Configuration \#2 & Additional important settings for decoders & & 21\\
& \begin{tabular}{l l l}
Bit & Description & Value\\
0 & Bi-directional bit: Keep driving direction when changing direction. & 1 \\
& Do not keep driving direction. & 0\\
1 & Disable decoder lock with CV 15 / 16 & 0\\
& Enable decoder lock with CV 15 / 16 & 2\\
2 & Disable prime mover startup delay & 0\\
& Enable prime mover startup delay & 4\\
3 & Disable SUSI protocol & 0 \\
& Enable SUSI protocol & 8\\
4 & Enable Output AUX9 (LokSound 5 H0 only) & 0\\
& Enable Wheel Sensor input (LokSound 5 H0 only) & 16\\
5 & Motor Overload Protection\\
& Motor is not switched off when blocked. & 0\\
& Motor is switched off for a few seconds when blocked to avoid burnout & 32\\
6 & Disable Automatic parking Brake & 0\\
& Enable Automatic parking Brake (EMK Braking) & 64 \\
7 & Reserved & 128\\
\end{tabular}\\

125 & Start voltage & Analog DC See section 10.8. & 0 - 255 & 90\\
126 & Maximum speed & Analog DC See section 10.8. & 0 - 255 & 130\\
127 & Start voltage & Analog AC See section 10.8. & 0 - 255 & 90\\
128 & Maximum speed & Analog AC See section 10.8. & 0 - 255 & 130\\
129 & Analog Functions & ``Hysterese" Offset voltage for functions in analogue mode. Chapter 10.8. & 0 - 255 & 15\\
130 & Analog Motor & ``Hysterese" Offset voltage for motor functions in analogue mode. Chapter 10.8. & 0 - 255 & 5\\
132 & Grade Crossing Hold Time & Grade Crossing holding time. See chapter 12.5.3. & 0 - 255 & 80\\
133 & Sound Fader & Volume when sound fader is active. See chapter 13.5. & 0 - 255 & 128\\
134 & ABC-Mode ``Sensibility" & Threshold, from which asymmentry on ABC shall be recognised. & 4 - 32 & 10\\
\end{tabular}
\begin{tabular}{p{0.1\linewidth} p{0.15\linewidth} p{0.5\linewidth} p{0.1\linewidth} p{0.1\linewidth}} 
\underline{CV} & \underline{Name} &  \underline{Description} & \underline{Range} & \underline{Value}\\


138 & Smoke Unit Trim Fan & Divided by 128, this gives the factor by which the fan speed of synchronized smoke units can be adjusted. & 0 - 255 & 128\\
139 & Smoke Unit Trim Temperature & Divided by 128, this gives the factor by which the temperature of synchronized smoke units can be adjusted. & 0 - 255 & 128\\
140 & Smoke TimeOut & Time until automatic shutdown of the smoke unit. & 0 - 255 & 255\\

141 & Smoke Chuff Min & Minimum duration of a steam chuff of an external smoke unit in 0.041 seconds resolution. & 0 - 255 & 10\\
142 & Smoke Chuff max & Maximum duration of a steam chuff of an external smoke unit in 0.041 seconds resolution. & 0 - 255 & 125\\
143 & Smoke Chuff Length & Divided by 128, this gives the factor by which the duration of the steam chuffs can be adjusted relative to the trigger pulses. & 0 - 255 & 100\\
144 & Smoke Pre Heat Temperature & Preheating temperature in degrees Celsius for secondary smoke generators (cylinder smoke unit) & 0 - 255 & 150\\
149 & ABC Shuttle Train Holdtimet & Time in seconds, which has to be passed for ABC shuttle train operation, before the direction of travel is changed. See section 10.4.4.3. & 0 - 255 & 255\\
150 & HLU Speedlimit 1 & HLU Speed limit 1. Internal speedstep. & 0 - 255 & 42\\
151 & HLU Speedlimit 2 &  (U) HLU Speed limit 2 (U). Internal speedstep. & 0 - 255 & 85\\
152 & HLU Speedlimit 3 & HLU Speed limit 3. Internal speedstep. & 0 - 255 & 127\\
153 & HLU Speedlimit 4 & (L) HLU Speed limit 4 (L). Internal speedstep. & 0 - 255 & 170\\
154 & HLU Speedlimit 5 & HLU Speed limit 5. Internal speedstep. & 0 - 255 & 212\\

155 -170 & Sound CV 1 - Sound CV 16 & 16 CVs for selecting sounds that can be assigned within sound projects. Please note the documentation for the sound project. & 0 - 255 & 0\\
179 & Brake Function 1 & Deceleration Value of which 33\% of CV 4 will be deducted if the Brake Function 1 is active. See section 10.6. & 0 - 255 & 80\\
180 & Brake Function 2 & Deceleration Value of which 33\% of CV 4 will be deducted if the Brake Function 2 is active. See section 10.6. & 0 - 255 & 40\\
181 & Brake Function 3 & Deceleration Value of which 33\% of CV 4 will be deducted if the Brake Function 3 is active. See section 10.6. & 0 - 255 & 40\\
182 & Brake Function 1 max. & Speed Highest speed step that can be reached when Brake function 1 is active. & 0 - 126 & 0\\
\end{tabular}
\begin{tabular}{p{0.1\linewidth} p{0.15\linewidth} p{0.5\linewidth} p{0.1\linewidth} p{0.1\linewidth}} 
\underline{CV} & \underline{Name} &  \underline{Description} & \underline{Range} & \underline{Value}\\

183 & Brake Function 2 max. & Speed Highest speed step that can be reached when Brake function 1 is active. & 0 - 126 & 126\\
184 & Brake Function 3 max. & Speed Highest speed step that can be reached when Brake function 1 is active. & 0 - 126 & 126\\

246 & Automatic decoupling Driving speed & Speed of the loco while decoupling; the higher the value, the faster the loco. Value 0 switches the automatic coupler off. Automatic decoupling is only active if the function output is adjusted to ``pulse" or ``coupler". & 0 - 255 & 0\\
247 & Decoupling - Removing time & This value multiplied with 0.016 defines the time the loco needs for moving away from the train (automatic decoupling). & 0 – 255 & 0\\
248 & Decoupling - Pushing time & This value multiplied with 0.016 defines the time the loco needs for pushing against the train (automatic decoupling). & 0 – 255 & 0\\
249 & Minimum steam chuff distance & Minimum distance of two steam chuffs, independant from sensor data. Compage chapter 13.3. & 0 – 255 & 0\\
250 & Secondary steam chuff trigger & Defines the distance between two consecutive steam chuffs for the secondary steam chuff generator. The value indicates the promilles the steam chuff distances of the secondary steam chuff generator ought to be shorter then those of the primary steam chuff generator. It is needed.for steam locos with two independent boogies, such as ``Big Boy" or ``Mallet". & 0 – 255 & 0\\

253 & Constant brake mode & Determines the constant brake mode. Only active, if CV254 $>$ 0 & 0 – 255 & 0\\
& & Function\\
& & CV 253 = 0: Decoder stops linearly\\
& & CV 253 $>$ 0: Decoder stops constantly linear\\
254 & Constant braking distance forward & A value $>$ 0 determines the way of brake distance it adheres to, independent from speed. & 0 – 255 & 0\\
255 & Constant braking distance backward & Constant braking distances during reverse driving. Only active, if value $>$ 0, otherwise the value of CV 254 is used. Useful for reversible trains. & 0 – 255 & 0\\

\end{tabular}






