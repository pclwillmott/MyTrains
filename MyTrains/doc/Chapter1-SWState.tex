\newpage
\subsection{OPC\_SW\_STATE}
\underline{Operation:} Request state of switch.

\underline{Group:} \hspace{0.5cm} 4-Byte Message

\underline{Direction:} \hspace{0.05cm} $\rightarrow$ Switch

\underline{Encoding:} 

Byte 0:

\begin{tabular}{p{0.4\linewidth} p{0.15\linewidth} p{0.38\linewidth}} 

\begin{tabular}{|p{0.3cm}|p{0.3cm}|p{0.3cm}|p{0.3cm}|p{0.3cm}|p{0.3cm}|p{0.3cm}|p{0.3cm}|}
\hline
1 & 0 & 1 & 1 & 1 & 1 & 0 & 0\\
\hline
\end{tabular}
& 0xBC & Opcode.\\
\end{tabular}

Byte 1:

\begin{tabular}{p{0.4\linewidth} p{0.15\linewidth} p{0.38\linewidth}} 

\begin{tabular}{|p{0.3cm}|p{0.3cm}|p{0.3cm}|p{0.3cm}|p{0.3cm}|p{0.3cm}|p{0.3cm}|p{0.3cm}|}
\hline
0 & n & n & n & n & n & n & n\\
\hline
\end{tabular}
& $<$SW1$>$ & Switch address A6 to A0.\\
\end{tabular}

Byte 2:

\begin{tabular}{p{0.4\linewidth} p{0.15\linewidth} p{0.38\linewidth}} 

\begin{tabular}{|p{0.3cm}|p{0.3cm}|p{0.3cm}|p{0.3cm}|p{0.3cm}|p{0.3cm}|p{0.3cm}|p{0.3cm}|}
\hline
0 & d6 & d5 & d4 & d3 & d2 & d1 & d0\\
\hline
\end{tabular}
& $<$SW2$>$ & Switch address A10 to A7 and switch control bits.\\
\end{tabular}

\begin{tabular}{p{0.05\linewidth} p{0.8\linewidth}} 
d6 & Reserved. Set to 0.\\
d5 & Direction. 1 means closed/green, and 0 means thrown/red.\\
d4 & Output. 1 means on, and 0 means off.\\
d3 & A10.\\
d2 & A9.\\
d1 & A8.\\
d0 & A7.\\
\end{tabular}

Byte 3:

\begin{tabular}{p{0.4\linewidth} p{0.15\linewidth} p{0.38\linewidth}} 

\begin{tabular}{|p{0.3cm}|p{0.3cm}|p{0.3cm}|p{0.3cm}|p{0.3cm}|p{0.3cm}|p{0.3cm}|p{0.3cm}|}
\hline
0 & n & n & n & n & n & n & n\\
\hline
\end{tabular}
& $<$CHK$>$ & Checksum.

\end{tabular}

\underline{Description:}

Request state of switch.

\underline{Response:} 

OPC\_LONG\_ACK.

\underline{Notes:} 

This needs to be tested to see what the real purpose is.

\rule{15.1cm}{0.4pt}
