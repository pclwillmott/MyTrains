\newglossaryentry{address}{
    name=address,
    description={is the numeric identification code by which a decoder recognises commands directed specifically to it}
}

\newglossaryentry{bit}{
    name=Bit,
    description={is a binary digit, either 0 or 1}
}

\newglossaryentry{byte}{
    name=Byte,
    description={is a group of eight bits}
}

\newglossaryentry{Paged Mode}{
    name=Paged Mode,
    description={}
}

\newglossaryentry{Direct Mode}{
    name=Direct Mode,
    description={}
}

\newglossaryentry{Operations Mode}{
    name=Operations Mode,
    description={}
}

\newglossaryentry{Physical Register Mode}{
    name=Physical Register Mode,
    description={}
}

\newglossaryentry{command station}{
    name=command station,
    description={is the electronic device that generates DCC commands based upon inputs it receives and transmits them to decoders}
}

\newglossaryentry{Common}{
    name=Common,
    description={is a locomotive slot state that indicates that the slot is not currently in-use by a throttle but it is still being refreshed by the command station. A slot with a state of Common can be selected by any throttle on the network}
}

\newglossaryentry{configuration variable}{
    name=configuration variable,
    description={is a memory location in the decoder that contains information that controls the decoder's characteristics}
}

\newglossaryentry{consisting}{
    name=Consisting,
    description={is the linking of more than one locomotive together to be controlled by a single throttle and/or address}
}

\newglossaryentry{DCC}{
    name=DCC,
    description={stands for Digital Command Control}
}

\newglossaryentry{decoder}{
    name=Decoder,
    description={is the electronic device that receives the DCC signal from the command station through the track, decodes it and tells the locomotive, turnout or other equipment that it is controlling what to do}
}

\newglossaryentry{Extended Packet Format}{
    name=Extended Packet Format (EPF),
    description={is an extension of the baseline DCC format that allows us to use four digit addressing and other expanded decoder functions}
}

\newglossaryentry{fast clock}{
    name=fast clock,
    description={is a clock set to run faster than real time for operating sessions on a model railway to be run in compressed time. The ratio between fast time and real time is typically 4:1, 6:1 or 8:1}
}

\newglossaryentry{Global System Track Status}{
    name=Global System Track Status,
    description={means the byte 7 of a LocoSlotDataP1 or LocoSlotDataP2 response}
}

\newglossaryentry{Idle}{
    name=Idle,
    description={is a locomotive slot state that indicates that slot is not active and can be selected any throttle. The locomotive's decoder is not refreshed in this state}
}

\newglossaryentry{Free}{
    name=Free,
    description={is a locomotive slot state that indicates that the slot does not have an address loaded in it}
}

\newglossaryentry{In-Use}{
    name=In-Use,
    description={is a locomotive slot state that indicates that the slot has been made active by a throttle and can no longer be selected by another throttle}
}

\newglossaryentry{locomotive slot}{
    name=locomotive slot,
    description={is a memory location in the command station which holds information about a locomotive's decoder and current state}
}

\newglossaryentry{Loconet}{
    name=Loconet,
    description={is the peer-to-peer local area network system architecture used by Digitrax to carry DCC and other commands across Digitrax command control systems}
}

\newglossaryentry{Extended Address}{
    name=Extended Address,
    description={is a decoder address between 128 and 9983. This is stored in encoded form in CV17 and CV18}
}

\newglossaryentry{Consist Address}{
    name=Consist Address,
    description={is a consist address between 1 and 127. This is stored in CV19}
}

\newglossaryentry{Primary Address}{
    name=Primary Address,
    description={is a decoder address between 1 and 127. This is stored in CV1}
}

\newglossaryentry{message}{
    name=message,
    description={means a sequence of two or more bytes sent over the network that conform to the network message format. The first byte of the message is an opcode and the last is a checksum}
}

\newglossaryentry{mobile decoder}{
    name=mobile decoder,
    description={means an electronic device installed in a locomotive that receives a signal from the command station through the track, decodes it and tells the locomotive what to do}
}

\newglossaryentry{New}{
    name=New,
    description={is a locomotive slot state that indicates that the slot contains an address but it has not been made active by a throttle. The locomotive's decoder is not refreshed in this state}
}

\newglossaryentry{NMRA}{
    name=NMRA,
    description={is the National Model Railroad Association, founded in 1935. One of its purposes is to define and manage model railroad standards related to interchange of equipment in North America}
}

\newglossaryentry{occupancy detector}{
    name=occupancy detector,
    description={is a device which senses and provides feedback for the presence of a locomotive or specially equipped rolling stock on a section of track}
}

\newglossaryentry{opcode}{
    name=opcode,
    description={means the first byte of a network message. The opcode indicates the purpose and length of the message}
}

\newglossaryentry{Operating Mode}{
    name=Operating Mode,
    description={means 14 or 28/128 speed steps}
}

\newglossaryentry{Advanced Consist}{
    name=Advanced Consist,
    description={is where the consist information is stored in CV19 in the mobile decoder}
}

\newglossaryentry{Basic Consist}{
    name=Basic Consist,
    description={is where all locomotive mobile decoders in the consist have the same address}
}

\newglossaryentry{Advanced Consist Bit}{
    name=Advanced Consist Bit,
    description={means bit d2 of the Slot Status 1 byte}
}

\newglossaryentry{Universal Consist}{
    name=Universal Consist,
    description={is where the consist information is stored in the command station}
}

\newglossaryentry{Product Code}{
    name=Product Code,
    description={means the Digitrax assigned identifier code of a device's type}
}

\newglossaryentry{Operations Mode Programming}{
    name=Operations Mode Programming (OPS),
    description={means programming configuration variables in DCC locomotives equipped with EPF decoders while they are on the mainline}
}

\newglossaryentry{peer-to-peer}{
    name=peer-to-peer,
    description={is a network communication scheme where messages between devices are not managed or controlled by a central controller or server}
}

\newglossaryentry{Physical Register Programming}{
    name=Physical Register Programming,
    description={is a primitive type of service mode programming. This method is limited to programming 2 digit address, acceleration, deceleration and start voltage}
}

\newglossaryentry{physical throttle}{
    name=physical throttle,
    description={means an electronic input device, often hand-held, that is used to tell the command station what commands to send to the decoders}
}

\newglossaryentry{polled}{
    name=polled,
    description={is the process of interrogating a device to see if it has information or commands to send to the system}
}

\newglossaryentry{polling}{
    name=polling,
    description={is the process by which devices are interrogated sequentially, one after the other, to see if they have information or commands to send to the system}
}

\newglossaryentry{programming}{
    name=programming,
    description={is the action of setting the internal parameters of decoders and other control equipment. During programming, values are set for configuration variables to determine the personality of locomotives, stationary decoders and other programmable DCC devices}
}

\newglossaryentry{Programming Track}{
    name=Programming Track,
    description={is an isolated track section used for programming decoder equipped locomotives or transponder equipped rolling stock}
}

\newglossaryentry{refresh}{
    name=refresh,
    description={is the process by which command stations re-send data to decoders to be sure that the signal is not lost and that you have reliable operation. All in use and common locomotives in the system will continue to be refreshed until they become Idle}
}

\newglossaryentry{short address}{
    name=short Address,
    description={is a decoder address between 1 and 127}
}

\newglossaryentry{signature}{
    name=signature,
    description={is the combination of bits and bytes within a message that uniquely identify the message type}
}

\newglossaryentry{slot state}{
    name=slot state,
    description={means the current state of a locomotive slot. A locomotive slot can be in one the following states: Free, New, In-Use, Common or Idle}
}

\newglossaryentry{Slot Status 1}{
    name=Slot Status 1,
    description={means byte 3 of a LocoSlotDataP1 response or byte 4 of a LocoSlotDataP2 response}
}

\newglossaryentry{software throttle}{
    name=software throttle,
    description={means a software application that is used to tell the command station what commands to send to the decoders}
}

\newglossaryentry{stationary decoder}{
    name=stationary decoder,
    description={means an electronic device for a turnout or other accessory that receives a signal from the command station through the track, decodes it and tells the turnout or accessory what to do}
}

\newglossaryentry{system slot}{
    name=system slot,
    description={is a memory location in the command station which holds system information}
}

\newglossaryentry{throttle}{
    name=throttle,
    description={means a physical throttle or a software throttle}
}

\newglossaryentry{Throttle ID}{
    name=Throttle ID,
    description={means a pair of 7-bit numbers that identify (hopefully uniquely) the throttle to the command station}
}

\newglossaryentry{Command}{
    name=Command,
    description={means a message sent to a device to request it to do something}
}

\newglossaryentry{Response}{
    name=Response,
    description={means a message sent in response to a Command message}
}

\newglossaryentry{Report}{
    name=Report,
    description={means a message sent by a device in response to a change in its internal and/or external state}
}

\newglossaryentry{Broadcast}{
    name=Broadcast,
    description={means a message sent by a device to all devices on the network}
}

\newglossaryentry{standard slots}{
    name=standard slots,
    description={means the command station slots that are accessed and manipulated by protocol 1 messages}
}

\newglossaryentry{expanded slots}{
    name=expanded slots,
    description={means the command station slots that are accessed and manipulated by protocol 2 messages}
}

