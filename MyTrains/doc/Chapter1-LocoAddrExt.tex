\subsubsection{OPC\_LOCO\_ADR\_EXT}
\underline{Operation:} Request an extended slot for a locomotive.

\underline{Group:} \hspace{0.5cm} 4-Byte Message

\underline{Direction:} \hspace{0.05cm} $\rightarrow$ Command Station

\underline{Encoding:} 

Byte 0:

\begin{tabular}{p{0.4\linewidth} p{0.15\linewidth} p{0.38\linewidth}} 

\begin{tabular}{|p{0.3cm}|p{0.3cm}|p{0.3cm}|p{0.3cm}|p{0.3cm}|p{0.3cm}|p{0.3cm}|p{0.3cm}|}
\hline
1 & 0 & 1 & 1 & 1 & 1 & 1 & 0\\
\hline
\end{tabular}
& 0xBE & Opcode.\\
\end{tabular}

Byte 1:

\begin{tabular}{p{0.4\linewidth} p{0.15\linewidth} p{0.38\linewidth}} 

\begin{tabular}{|p{0.3cm}|p{0.3cm}|p{0.3cm}|p{0.3cm}|p{0.3cm}|p{0.3cm}|p{0.3cm}|p{0.3cm}|}
\hline
0 & n & n & n & n & n & n & n\\
\hline
\end{tabular}
& $<$ADR2$>$ & High address.\\
\end{tabular}

Byte 2:

\begin{tabular}{p{0.4\linewidth} p{0.15\linewidth} p{0.38\linewidth}} 

\begin{tabular}{|p{0.3cm}|p{0.3cm}|p{0.3cm}|p{0.3cm}|p{0.3cm}|p{0.3cm}|p{0.3cm}|p{0.3cm}|}
\hline
0 & n & n & n & n & n & n & n\\
\hline
\end{tabular}
& $<$ADR$>$ & Low address.\\
\end{tabular}

Byte 3:

\begin{tabular}{p{0.4\linewidth} p{0.15\linewidth} p{0.38\linewidth}} 

\begin{tabular}{|p{0.3cm}|p{0.3cm}|p{0.3cm}|p{0.3cm}|p{0.3cm}|p{0.3cm}|p{0.3cm}|p{0.3cm}|}
\hline
0 & n & n & n & n & n & n & n\\
\hline
\end{tabular}
& $<$CHK$>$ & Checksum.

\end{tabular}

\underline{Description:}

This message requests the slot number for the selected locomotive address. If the locomotive is found in the slot table then the command station returns an OPC\_SL\_RD\_DATA\_EXT message with the slot information. If it is not found then the command station will put the locomotive into a free slot and then return an OPC\_SL\_RD\_DATA\_EXT message with the slot information. If there are no free slots then the command station returns an OPC\_LONG\_ACK error code.

Note that regular short 7 bit NMRA addresses are denoted by $<$ADR2$>$ = 0. The Analog, zero stretched, locomotive is selected when both $<$ADR2$>$ = 0 and $<$ADR$>$ = 0. $<$ADR$>$ is always a 7 bit value. If $<$ADR2$>$ is non-zero then the master will generate NMRA type 14 bit or long address packets using all 14 bits from $<$ADR2$>$ and $<$ADR$>$ with $<$ADR2$>$ being the most significant address bits. Note that a DT200 Master does not process 14 bit address requests and will consider the $<$ADR2$>$ to be zero. You can check the $<$TRK$>$ return bits to see if the master is a DT200.

\underline{Response:} 

OPC\_SL\_RD\_DATA\_EXT if success, otherwise OPC\_LONG\_ACK.

\underline{Notes:} 

None.

\rule{15.1cm}{0.4pt}
